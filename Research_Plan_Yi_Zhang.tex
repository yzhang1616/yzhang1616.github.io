%\documentclass{article}[12pt]
%\usepackage{amssymb,amsmath}
\documentclass[10pt,a4paper]{article}
%\usepackage[T1]{fontenc}
%\usepackage[utf8]{inputenc}
\usepackage[utf8]{inputenc}
\usepackage{enumitem}
\usepackage{authblk}
%\usepackage{fullpage}
\usepackage[colorlinks = true,
            linkcolor = blue,
            urlcolor  = blue,
            citecolor = blue,
            anchorcolor = blue]{hyperref}

\usepackage{bbm}
\usepackage{amsmath,amsthm}
\usepackage{alltt, amssymb}
\usepackage[]{algorithm2e}
\usepackage{url}
\usepackage{graphicx}

\newcommand{\bigO}{{\mathcal{O}}}
\newcommand{\bigOsoft}{\tilde{\mathcal{O}}}
\newcommand{\bA}{ {\mathbb A}}
\newcommand{\bN}{ {\mathbb  N}}
\newcommand{\bZ} { {\mathbb Z}}
\newcommand{\bQ}{ {\mathbb  Q}}
\newcommand{\bC}{ {\mathbb  C}}
\newcommand{\bK}{ {\mathbb  K}}
\newcommand{\bD}{ {\mathbb D}}
\newcommand{\bE}{ {\mathbb E}}
\newcommand{\bR}{ {\mathbf R}}
\newcommand{\bx}{ {\mathbf x}}
\newcommand{\by}{ {\mathbf y}}
\newcommand{\bz}{ {\mathbf z}}
\newcommand{\bs}{ {\mathbf s}}
\newcommand{\bt}{ {\mathbf t}}
\newcommand{\bff}{ {\mathbf f}}
\newcommand{\bpa}{ {\boldsymbol \partial}}
\newcommand{\balpha}{ {\boldsymbol \alpha}}
\newcommand{\bbeta}{ {\boldsymbol \beta}}
\newcommand{\bdelta}{ {\boldsymbol \delta}}
\newcommand{\bgamma}{ {\boldsymbol \gamma}}
\newcommand{\bu}{ {\mathbf u}}
\newcommand{\bw}{ {\mathbf w}}
\newcommand{\cP}{ {\mathcal P}}
\newcommand{\cG}{ {\mathcal G}}
\newcommand{\bv}{ {\mathbf v}}
\newcommand{\bU}{ {\mathbf U}}
\newcommand{\bm}{ {\mathbf m}}
\newcommand{\bn}{ {\mathbf n}}
\newcommand{\bB}{ {\mathbf B}}
\newcommand{\ie}{{\it i.e.}}
\newcommand{\si} { {\sigma}}
\newcommand{\dd}{ {\rm d}}
\newcommand{\tor}{ {\rm tor}}
\newcommand{\den}{ {\rm den}}
\newcommand{\lc}{ \operatorname{lc}}
\newcommand{\lm}{ \operatorname{lm}}
\newcommand{\mm}{ \operatorname{M}}
\newcommand{\TT}{ \operatorname{T}}
\newcommand{\lt}{ \operatorname{lt}}
\newcommand{\HM}{{\operatorname{HM}}}
\newcommand{\HT}{{\operatorname{HT}}}
\newcommand{\PT}{{\operatorname{PT}}}
\newcommand{\PE}{{\operatorname{PE}}}
\newcommand{\HC}{{\operatorname{HC}}}
\newcommand{\lcm}{ \operatorname{lcm}}
\newcommand{\qlcm}{ \operatorname{qlcm}}
\newcommand{\spol}{ \operatorname{spol}}
\newcommand{\gpol}{ \operatorname{gpol}}
\newcommand{\rank}{ \operatorname{rank}}
\newcommand{\pa}{ {\partial}}
\newcommand{\lder}{ \operatorname{lder}}
\newcommand{\pder}{ \operatorname{pder}}
\newcommand{\ind}{ \operatorname{ind}}
\newcommand{\In}{ \operatorname{in}}
\newcommand{\sol}{ \operatorname{sol}}
\newcommand{\ann}{ \operatorname{ann}}
\newcommand{\dis}{ \operatorname{dis}}
\newcommand{\Drat}{ {\bK(\bx)[\bpa]}}
\newcommand{\Dpol}{ {\bK[\bx][\bpa]}}
\newcommand{\qpol}{ {\bK(q)[x][\pa]}}
\newcommand{\qrat}{ {\bK(q, x)[\pa]}}
\newcommand{\pf} {{\rm {\bf Proof.} } }
\newcommand{\cont}{\operatorname{Cont}}
\newcommand{\red}{\color{red}}

\newtheorem{thm}{Theorem}[section]
\newtheorem{cor}[thm]{Corollary}
\newtheorem{lemma}[thm]{Lemma}
\newtheorem{prop}[thm]{Proposition}
\newtheorem{defn}[thm]{Definition}
\newtheorem{ex}[thm]{Example}
\newtheorem{algo}[thm]{Algorithm}
\newtheorem{remark}[thm]{Remark}
\newtheorem{conj}[thm]{Conjecture}
\newcommand{\AODE}{{AO{$\Delta$}E}}
\newcommand{\AODEs}{{AO{$\Delta$}Es}}

\renewcommand{\labelenumi}{(\arabic{enumi})}

\begin{document}

\title{\bf{\Huge{Research Plan}}}

\author{Yi Zhang}
% \thanks{Supported by the Austrian Science Fund (FWF): P29467-N32.\\
% E-mail addresses:  zhangy@amss.ac.cn}\ }
% % \author[2]{Ziming Li\thanks{Supported by the NSFC grants (91118001, 60821002/F02)
% % and a 973 project (2011CB302401). Email: zmli@mmrc.iss.ac.cn}}
% % \author{Yi Zhang$^{\ast}$}
% %\thanks{Supported by the Austrian Science Fund (FWF): P29467-N32. E-mail: zhangy@amss.ac.cn}}
% %\author[1]{Author C\thanks{C.C@university.edu}}
% %\author[2]{Author D\thanks{D.D@university.edu}}
% %\author[2]{Author E\thanks{E.E@university.edu}}
% % \affil[2]{KLMM, AMSS, Chinese Academy of Sciences, Beijing, China}
% % \affil[1, 3]{Institute for Algebra, Johannes Kepler University Linz, Austria}
% \affil{Johann Radon Institute for Computational and Applied Mathematics (RICAM), Austrian Academy of Sciences, Austria}
%\affil[2]{KLMM, AMSS, Chinese Academy of Sciences, Beijing 100190, China}

%\renewcommand\Authands{ and }

% \alignauthor
% \leavevmode
% \mathstrut
% Ziming Li\titlenote{Supported by the NSFC grants (91118001, 60821002/F02)
% and a 973 project (2011CB302401).\vspace{5pt}}\\[\smallskipamount]
%  \affaddr{\leavevmode\mathstrut KLMM, AMSS, Chinese Academy of Sciences, Beijing 100190, China}\\
%  \affaddr{\mathstrut 4040 Linz, Austria}\\
%  \affaddr{\mathstrut zmli@mmrc.iss.ac.cn}
% \and
%% \mathstrut
% Yi Zhang\titlenote{Supported by the Austrian Science Fund (FWF) grants Y464-N18, NSFC grants (91118001, 60821002/F02)
% and a 973 project (2011CB302401).\vspace{-4pt}}\\[\smallskipamount]
%  \affaddr{\leavevmode\mathstrut Institute for Algebra, Johannes Kepler University, Linz A-4040, Austria}\\
%  \affaddr{\leavevmode\mathstrut KLMM, AMSS, Chinese Academy of Sciences, Beijing 100190, China}\\
%  \affaddr{\leavevmode\mathstrut zhangy@amss.ac.cn}


\date{}
\maketitle

My main research interests are symbolic computation and its applications in combinatorics, knot theory, statistics, 
and cryptography.  Symbolic computation aims to give algorithmic and constructive answers 
to various problems in mathematics and computer science, such as polynomial factorization, computing solutions of systems of polynomial equations, 
and quantifier elimination. Systems of algebraic differential equations and difference equations are important research subjects in mathematics, physics, and related areas. 
The algebraic study of such systems gives useful information about their applications in physics, statistics, and other areas. 
%Computing solutions and illustrating algebraic structures of those systems give essential answers to their applications in physics, statistics and so on.  
%%Functions defined by systems of differential equations and difference equations are a principal focus of study in many areas of math and physics. Understanding the algebraic properties %of these functions is essential in many of their physical and mathematical applications. 
%Much of my work is devoted to developing algorithms that can discover these properties automatically.
Much of my work is devoted to developing algorithms for  computing solutions and illustrating algebraic structures of differential equations and difference equations
by using constructive tools (such as Gr\"obner bases and resultant theory) in computer algebra and differential algebra. 
My work has found interesting applications in the certification of integer sequences, checking special cases of a conjecture of Krattenthaler, and verifying 
several instances of the colored Jones polynomial are Laurent polynomial sequences.  
The following sections describe my research plans in the future. 

\section*{The algebraic-geometric method for solving algebraic difference equations} \label{SECT:AODE}

An algebraic ordinary difference equation (\AODE) is a difference equation of the form
\begin{equation} \label{EQ:AODE}
F(x, y(x), y(x + 1), \cdots, y(x + m))=0,
\end{equation}
where $F$ is a nonzero polynomial in $y(x), y(x + 1), \cdots, y(x + m)$ with coefficients in the field $\bK(x)$ of rational functions over an algebraically closed field $\bK$ 
of characteristic zero, and~$m \in \bN$. We call $m$ the \emph{order} of~\eqref{EQ:AODE}. 
%{\red add some motivations for studying {\AODE}s}. 
{\AODE}s naturally appear from various problems, such as symbolic summation~\cite{PWZbook1996, KoutschanThesis}, 
factorization of linear difference operators~\cite{BronsteinPetkovsek1996}, 
analysis of time or space complexity of computer programs with recursive calls~\cite{Eekelen2018}. 
We are interested in computing symbolic solutions (for instance, polynomials solutions, rational solutions) of {\AODE}s.
In particular, for a first-order {\AODE}, the corresponding algebraic equation $F(x, y, z) = 0$ defines an algebraic curve in the two dimensional affine plane 
over the field $\overline{\mathbb{K}(x)}$. A solution in a certain class of
functions, such a rational or algebraic functions, determines a parametrization of this algebraic curve. 
Thus, we may apply algebraic tools from parametrization theory of algebraic curves to study solutions of first-order {\AODE}s. 
Based on this observation, we propose an algebraic-geometric approach to solve first-order {\AODE}s: 
\begin{itemize}
 \item [1] Decide whether a given first-order {\AODE} can be parametrized with functions form a given class, 
such as rational parametrization;
 \item [2] Solve the corresponding reduced {\AODE} by using techniques from computer algebra 
and differential algebra.
\end{itemize}
This idea is inherited from the differential case and turns out to be successful for solving algebraic 
differential equations. For details, see~\cite{Winkler2019}. 
Using this method, we give an complete algorithm~\cite{VoZhang2019} to compute rational solutions of first-order autonomous {\AODE}s. 
Possible future work is as follows: 

\begin{itemize}
\item  Design algorithms to compute polynomial and rational solutions of high-order \AODEs.
\item Compute rational solutions of non-autonomous first-order \AODEs. 
\end{itemize}	
% Next, we will
% Next, we will study how to use this algebraic-geometric approach to compute other types of solutions 
% (such as algebraic solutions, polynomial solutions, series solutions) of first-order and higher-order {\AODE}s. 





%\section*{Fast computation of intersections of zero-dimensional ideals of linear partial differential operators} \label{SECT:lclm}

\section*{The improved holonomic gradient method via gauge transformation} \label{SECT:hgm}


\section{Contraction of Ore ideals with applications} \label{SECT:contraction}

\subsection{Introduction}
Let $\bK$ be a field of characteristic $0$. 
Consider the following linear recurrence equation:
\begin{equation} \label{EQ:recurrence}
 a_0(n)f(n) + \cdots + a_r(n)f(n+r)=0,
\end{equation}
where $a_i \in \bK[n]$ with $a_r \neq 0$, and $i = 0, \ldots, r$. The roots of $a_r(n)$ is called 
the singularities of~\eqref{EQ:recurrence}.  
There is a strong connection between the roots of $a_r$ 
and the singularities of a solution of~\eqref{EQ:recurrence} . 

It is well know that every singularity of a solution of~\eqref{EQ:recurrence} 
must be a root of~$a_r$. However, the converse is not true.
Generally speaking, the leading coefficient~$a_r$ may have roots at a point where no solution is singular. 
Such points are called
apparent singularities, and it is sometimes useful to
identify them. The technique for doing so is called desingularization.
For instance, consider the recurrence operator
\[
 L = (1 + 16 n)^2 \pa^2 - 32 (7 + 16 n) \pa - (1 + n)(17 + 16 n)^2,
\]
which comes from~\cite[Section 4.1]{Abramov2006}. 
In this, we use $\pa$ to denote the shift operator $f(n) \mapsto f(n + 1)$.
For any choice of two initial values
$u_0,u_1\in  \bQ$, there is a unique sequence $u \colon
\bN \to \bQ$ with $u(0)=u_0$, $u(1)=u_1$ and $L$ applied to $u$
gives the zero sequence. A priori, it is not obvious whether or
not $u$ is actually an integer sequence, if we choose $u_0,u_1$
from~$\bZ$, because the calculation of the $(n+2)$nd term
from the earlier terms via the recurrence encoded by $L$ requires
a division by $(1+16n)^2$, which could introduce fractions. In order
to show that this division never introduces a denominator, 
we note that every solution of $L$ is also a solution
of its left multiple
\begin{equation} \label{EQ:ah}
\begin{array}{ccl}
T  & = & \pa^3 +\left(128 n^3-104 n^2-11 n-3\right) \pa^2 + \\
          &   & \left(-256 n^2+127 n + 94 \right) \pa - \\
          &   & (128 n^2+24 n-131)(1 + n)^2,
\end{array}
\end{equation}
The operator $T$ has the interesting property that the factor
$(1+16n)^2$ has been ``removed'' from the leading
coefficient, which immediately certifies the integrality of its solutions. 
The process of obtaining the operator $T$ from $L$ is called
desingularization, because there is a polynomial factor in the
leading coefficient of $L$ which does not appear in the leading
coefficient of~$T$. 


In more algebraic terms, we consider the following problem. Given
an operator $L\in \bZ[x][\pa]$, where $\bZ[x][\pa]$
is an Ore algebra, we consider
the left ideal $\langle L \rangle = \bQ(x)[\pa]L$ generated by $L$ in the
extended algebra $\bQ(x)[\pa]$. The contraction of $\langle L \rangle$ to
$\bZ[x][\pa]$ is defined as $\cont(L) := \langle L \rangle \cap
\bZ[x][\pa]$. This is a left ideal of $\bZ[x][\pa]$ which
contains $\bZ[x][\pa]L$, but in general more operators.
Our goal is to compute a $\bZ[x][\pa]$-generating set of $\cont(L)$.
In the example above, such a generating set is given by $\{L, T\}$ 	.
The traditional desingularization problem corresponds to computing
a generating set of the $\bQ[x][\pa]$-left ideal $\langle L \rangle \cap \bQ[x][\pa]$.


\subsection{Main results}
Given an Ore operator~$L$ with polynomial coefficients in~$x$, it generates a left ideal~$I$ in the Ore algebra
over the field~$\bK(x)$ of rational functions. 

\begin{enumerate}
 \item We present an algorithm for computing a generating set of the contraction ideal of~$I$
in the Ore algebra over the ring~$R[x]$ of polynomials, where~$R$ may be either~$\bK$ or a domain with~$\bK$ as its fraction field.
 \item Using a generating set of the contraction ideal,
we compute a completely desingularized operator for~$L$ whose leading coefficient not only
has minimal degree in~$x$ but also has minimal content.
 \item Using completely desingularized operators, we study how to certify the integrality of a sequence and 
 check special cases of a conjecture of Krattenthaler.
\end{enumerate} 

This work is published in ISSAC'16~\cite{Zhang2016}.

\subsection{Future work}

\begin{enumerate}
 \item Our algorithms rely heavily on the computation of Gr\"{o}bner bases over a principal ideal domain~$R$.
At present, the computation of Gr\"{o}bner bases over $R$ is not fully available in a computer algebra system. 
So the algorithms are not yet implemented. We would like to implement our algorithm in Maple or Mathematica 
by using linear algebra over $R$ as much as possible.
 \item Design algorithms for determining a generating set of a contraction ideal in the multivariate Ore algebra.
\end{enumerate}

\section{Apparent singularities of D-finite systems} \label{SECT:apparent}

\subsection{Introduction}

A D-finite function is specified by a linear ordinary differential equation with polynomial 
coefficients and finitely many initial values. Each singularity of a D-finite function 
will be a root of the coefficient of the highest order derivative appearing 
in the corresponding differential equation. 
For instance, $x^{-1}$ is a solution of the equation $x f'(x) + f(x) = 0$, 
and the singularity at the origin is also the root of the polynomial $x$. 
However, the converse is not true. For instance, the solution space of 
the differential equation $x f'(x) - 3 f(x) = 0$ is spanned by $x^3$ as a vector space, 
but none of those functions has singularity at the origin.

More specifically, we consider the following ordinary differential equation 
$$p_0(x)f(x) + \cdots + p_r(x)f^{(r)}(x) = 0,,$$ 
where $p_i \in \bK[x]$ with $p_r\neq0$, and $\bK$ is a field of characteristic $0$.  
The roots of $p_r$ are called the
singularities of the equation. A root $\alpha$ of $p_r$ is call \emph{apparent} if the
differential equation admits $r$ linearly independent formal power series solutions in 
$x - \alpha$. Deciding whether a singularity is apparent is therefore the same as
checking whether the equation admits a fundamental system of formal power series
solutions at this point. This can be done by inspecting the so-called
\emph{indicial polynomial} of the equation at~$\alpha$ and solving a system of finitely many linear equations. 
If a singularity $\alpha$ of an ordinary differential is apparent, then we can always 
construct a second ordinary differential equation whose solution space contains all the solutions of the
first equation, and which does not have $\alpha$ as a singularity any more. 
This process is called \emph{desingularization}. The purpose of our work is to generalize the facts sketched above to
the multivariate setting.

\subsection{Main results}

\begin{enumerate}
 \item We generalize the notions of singularities and ordinary points from linear ordinary differential equations to D-finite systems. 
 Ordinary points of a D-finite system are characterized in terms of its formal power series solutions.
 \item We show that apparent singularities can be removed like in the univariate
case by adding suitable additional solutions to the system at hand.
 \item Several algorithms are presented for removing and detecting apparent singularities of D-finite systems.
 \item An algorithm is given for computing formal power series solutions of a D-finite system
at apparent singularities.
\end{enumerate}

This work is available in~\cite{Yi2017}.

\subsection{Future work}

\begin{enumerate}
 \item Generalize our algorithms for removing and detecting apparent singularities of D-finite systems to other singularities.
 \item Study the desingularization problem for the multivariate linear difference equations with polynomial coefficients. 
\end{enumerate}

\section{Laurent series solutions of algebraic ordinary differential equations} \label{SECT:Laurentsols}

\subsection{Introduction}

An algebraic ordinary differential equation (AODE) is of the form 
$$F(x,y,y',\ldots,y^{(n)})=0,$$ 
where $F$ is a polynomial in $y,y',\ldots,y^{(n)}$ with coefficients in $\mathbb{K}(x)$, the field $\bK$ is 
algebraically closed field of characteristic zero, and $n \in \bN$. 
Many problems from applications (such as physics, combinatorics and statistics) 
can be characterized in terms of AODEs.  
% The laws of natural world are usually modelled in the form of differential equations. 
% In many cases, these equations are AODEs. 
% Or the problem of solving these equations can be reduced to that of solving AODEs. 
Therefore, determining (closed form) solutions of an AODE is one of the central problems in mathematics and computer science.
 
Although linear ODEs~\cite{Ince1926} have been intensively studied, there are still many challenging problems for solving
(nonlinear) AODEs. As far as we know, approaches for solving AODEs are only available for very specific subclasses. 
For example, Riccati equations, 
which have the form $y'=f_0(x)+f_1(x)y+f_2(x)y^2$ for some $f_0,f_1,f_2 \in \mathbb{K}(x)$, %(like: $y' = 1 + y^2$) 
can be considered as the simplest form of nonlinear AODEs. 
In \cite{Kovacic}, Kovacic gives a complete algorithm for determining Liouvillian solutions of 
a Riccati equation with rational function coefficients. 
% The study of general solutions without movable singularities 
% can be found in \cite{Fuchs, Malmquist, Poincare} for first-order, and in \cite{Eremenko2, Ince1926} for higher-order AODEs. 

Since the problem of solving an arbitrary AODE is very difficult, 
it is natural to ask whether a given AODE admits some special kinds of solutions, 
such as polynomials, rational functions, or formal power series. 
During the last two decades, an algebraic-geometric approach for finding symbolic solutions of AODEs has been developed. 
The work by Feng and Gao in \cite{FengGao, FengGao06} for computing rational general solutions of first-order autonomous AODEs 
can be considered as the starting point. 
% In \cite{NgoWinkler11b, GraseggerThesis, VoWinkler2015, VoGraseggerWinkler2017}, 
The authors of~\cite{NgoWinkler11b, GraseggerThesis, VoWinkler2015, VoGraseggerWinkler2017} 
developed methods for finding different kinds of solutions of non-autonomous, higher-order AODEs. 
For formal power series solutions, we refer to \cite{DenefLipshitz,SingerFormalSolutions}.
As far as we know, there is few results concerning Laurent series solutions of AODEs. 
Our main purpose is to give a method for determining such solutions.

\subsection{Main results}

\begin{enumerate}
 \item We present several approaches to compute formal power series solutions of a given AODE.
 \item Given an AODE, we determine a bound for the order of its Laurent series solutions. 
 Using the order bound, one can transform a given AODE into a new one whose Laurent series solutions are only formal power series.
 \item As applications, new algorithms are presented for determining all particular polynomial and rational solutions of certain classes of AODEs.
\end{enumerate}

% In this paper, we consider Laurent series solutions of algebraic ordinary differential equations (AODEs). 
% We first present several approaches to compute formal power series solutions of a given AODE. 
% Then we determine a bound for the order of its Laurent series solutions. 
% Using the order bound, one can transform a given AODE into a new one whose Laurent series solutions are only formal power series.
% The idea is basically inherited from the Frobenius method for linear ordinary differential equations. 
% As applications, new algorithms are presented for determining all particular polynomial and rational solutions of certain classes of AODEs.

This work is available in~\cite{VoZhang2017}.

\subsection{Future work}

\begin{enumerate}
 \item Design algorithms for computing formal power series solutions of AODEs, which extends the classic Implicit Function Theorem of AODEs.
 \item Compute rational solutions of first-order algebraic difference equations by using the parametrization of algebraic curves.  
\end{enumerate}


\section{Desingularization in the \texorpdfstring{$q$}{q}-Weyl algebra} \label{SECT:qdesing}

\subsection{Introduction}

Prof.\ Stavros Garoufalidis, who is an expert for knot theory, 
presented the following conjecture in an email with the author:

\begin{conj} \label{CONJ:stavros}
(Garoufalidis): Let $J_{K,n}(q)$ denote the Jones polynomial of a knot colored by the $n$-dimensional irreducible 
representation of $\mathfrak{sl}_2$ and normalized by $J_{\text{Unknot},n}(q)=1$. 
Then, (a) $(1-q^n)*J_{K,n}(q)$ satisfies a bimonic recursion relation. (b) $J_{K,n}(q)$ does not satisfy a monic recursion relation.
\end{conj}

Using $q$-holonomic summation methods (as implemented in the \texttt{qMultiSum}
package~\cite{Riese03} or \texttt{HolonomicFunctions}
package~\cite{Christoph2010}) or by guessing (as implemented in the
\texttt{Guess} package~\cite{Kauers2009a}), 
we can always compute $q$-holonomic recurrence equations for $(1-q^n)*J_{K,n}(q)$ and $J_{K,n}(q)$, respectively. 
However, the equation for $(1-q^n)*J_{K,n}(q)$ usually does not satisfy the property in Conjecture~\ref{CONJ:stavros}. 
Furthermore, we can not see immediately that $J_{K,n}(q)$ does not satisfy a monic recursion relation.

In order to certify Conjecture~\ref{CONJ:stavros} for some specific $J_{K,n}(q)$, 
we develop the desingularization technique in the $q$-Weyl algebra. 

As an example, consider the $q$-holonomic sequence
\[
  f(n) = [n]_q := \frac{q^n-1}{q-1}
\]
that is a $q$-analog of the natural numbers.  The minimal-order homogeneous
$q$-recurrence satisfied by $f(n)$ is
\[
  (q^n-1)f(n+1) - (q^{n+1}-1)f(n) = 0,
\]
in operator notation:
\begin{equation} \label{EQ:qrecurrence}
 \bigl((q^n-1)\pa - q^{n+1} + 1\bigr) \cdot f(n) = 0.
\end{equation}

When we multiply this operator by a suitable left factor, we obtain
a monic (and hence: desingularized) operator of order~$2$:
\begin{equation} \label{EQ:qdesin}
  \frac{1}{q^{n+1}-1}\bigl(\pa - q\bigr)\bigl((q^n-1)\pa - q^{n+1} + 1\bigr) =
  \pa^2 - (q+1)\pa + q.
\end{equation}
The process of deriving~\eqref{EQ:qdesin} from~\eqref{EQ:qrecurrence} to called desingularization in the $q$-Weyl algebra.

% From this representation it is a routine task (but possibly computationally
% expensive) to compute a $q$-holonomic recurrence equation for
% $J^{\mathrm{twist}}_p(n)$ when $p$ is a fixed integer. This can be done either
% by $q$-holonomic summation methods (as implemented in the \texttt{qMultiSum}
% package~\cite{Riese03} or \texttt{HolonomicFunctions}
% package~\cite{Christoph2010}) or by guessing (as implemented in the
% \texttt{Guess} package~\cite{Kauers2009a}).

\subsection{Main results}

\begin{enumerate}
 \item We give an order bound for desingularized operators, and thus derive
an algorithm for computing desingularized operators in the first $q$-Weyl
algebra.
 \item An algorithm is presented for computing a generating set
of the first $q$-Weyl closure of a given $q$-difference operator.
 \item As an application, we certify that several instances of $J_{K,n}(q)$ always 
 satisfy the properties specified in Conjecture~\ref{CONJ:stavros}.  
\end{enumerate}

% In this paper, we study the desingularization problem in the first $q$-Weyl
% algebra.  We give an order bound for desingularized operators, and thus derive
% an algorithm for computing desingularized operators in the first $q$-Weyl
% algebra.  Moreover, an algorithm is presented for computing a generating set
% of the first $q$-Weyl closure of a given $q$-difference operator.  As an
% application, we certify that several instances of the colored Jones polynomial
% are Laurent polynomial sequences by computing the corresponding desingularized
% operator.

This work is available in~\cite{KZ2018}.

\subsection{Future work}

\begin{enumerate}
 \item Study the desingularization problem in the multivariate $q$-Weyl algebra.
 \item Develop the desingularization technique for linear Mahler equations. 
\end{enumerate}

\section{An enhanced holonomic gradient method with algebraic and numerical analysis of differential equations} \label{SECT:HGM}

\subsection{Introduction}

Studying problems in differential equations which arose in statistics will lead us a remarkable advances in the algebraic and algorithmic study of differential equations and the combination of algebraic algorithms and numerical algorithms for differential equations. An important approach in the algebraic analysis of differential equation is the holonomic gradient method. 

  Let us first recall the idea of the holonomic gradient method~\cite{Nakayama2011}. A holonomic function with n variables is a function which satisfies n linear ordinary differential equations with multivariate polynomial coefficients for each independent variable. Those differential equations satisfied by a holonomic function is called a holonomic system. The holonomic gradient method (HGM) is an approach to evaluate numerically normalizing constants and their derivatives of holonomic probability distributions. HGM consists of three steps: 
\begin{enumerate}
  \item Finding a holonomic system satisfied by the normalizing constant. We may use the restriction algorithm from D-module theory and related methods to compute it. 
  \item Finding an initial value vector for the holonomic system. It is equivalent to evaluating the normalizing constant and its derivatives at a point. This step is usually performed by numerical integration.
  \item Solving the holonomic system numerically. We can use classical methods in numerical analysis such as the Runge-Kutta method of solving ordinary differential equations and efficient solvers of systems of linear equations.
  \end{enumerate}

For the first step of HGM, there are efficient algorithms (such as the creative telescoping method) to derive a holonomic system for the target normalizing constant. The holonomic system can be translated into a linear ODE system (Pfaffian system) for the normalizing constant and its derivatives. However, if the normalizing constant is not the dominant~\cite{Danufane2018} solution among all the solutions of the corresponding linear ODE system as the independent variable goes to infinity, then the usual methods involved in the third step of HGM only works for a small interval. Besides, the evaluation step relies on the precision of initial values of the target normalizing constant and its derivatives. The current methods for evaluating initial values with high-precision are also not satisfactory.
  
  We want to design an enhanced HGM by combining theoretical study of ODEs, algebraic algorithms, and numerical algorithms for ODEs to give a more efficient numerical evaluator in the second and third step of HGM. Furthermore, we will apply the improved HGM to study problems in differential equations which arose in statistics, combinatorics and so on.  


\subsection{Main results}

We give an approximate formula of the distribution of the largest
eigenvalue of real Wishart matrices by the expected Euler characteristic
method for the general dimension.
The formula is expressed in terms of a definite integral with parameters.
We derive a differential equation satisfied by the integral
for the $2 \times 2$ matrix case and perform a numerical analysis of it.

This work is available in~\cite{Takayama2019}.

\subsection{Future work}


\bibliographystyle{abbrv}

\begin{thebibliography}{10}

\bibitem{Abramov2006}
S.~A. Abramov, M.~Barkatou, and M.~van Hoeij.
\newblock Apparent singularities of linear difference equations with polynomial
  coefficients.
\newblock {\em AAECC}, 117--133, 2006.


\bibitem{BronsteinPetkovsek1996}
M.~Bronstein and M.~Petkov{\v{s}}ek.
\newblock An introduction to pseudo-linear algebra.
\newblock {\em Theoretical Computer Science}, 157:3--33, 1996.
  
\bibitem{Yi2017}
S.~Chen, M.~Kauers, Z.~Li, and Y.~Zhang.
\newblock Apparent singularities of {D}-finite systems.
\newblock {\em Journal of Symbolic Computation}, 95:217--237, 2019.
  
\bibitem{Danufane2018}
F.~H.~Danufane, C.~Siriteanu, K.~Ohara, N.~Takayama,
\newblock Holonomic gradient method-based CDF evaluation for the largest eigenvalue
of a complex noncentral Wishart matrix.
\newblock {\em arXiv:1707.02564}, 1--29, 2018 .

\bibitem{DenefLipshitz}
J.~Denef and L.~Lipshitz.
\newblock Power series solutions of algebraic differential equations.
\newblock {\em Mathematische Annalen}, 267:213--238, 1984.

\bibitem{FengGao}
R.~Feng and X.-S. Gao.
\newblock Rational general solutions of algebraic ordinary differential
  equations.
\newblock In {\em Proceedings of the 2004 International Symposium on Symbolic
  and Algebraic Computation}, ISSAC'04, pages 155--162, New York, NY, USA,
  2004. ACM.

\bibitem{FengGao06}
R.~Feng and X.-S. Gao.
\newblock A polynomial time algorithm for finding rational general solutions of
  first order autonomous {ODEs}.
\newblock {\em Journal of Symbolic Computation}, 41(7):739--762, 2006.



\bibitem{GraseggerThesis}
G.~Grasegger.
\newblock {\em {Symbolic solutions of first-order algebraic differential
  equations}}.
\newblock PhD thesis, Johannes Kepler University Linz, 06 2015.



\bibitem{Ince1926}
E.~Ince.
\newblock Ordinary differential equations.
\newblock Dover, 1926.

\bibitem{Kauers2009a}
M.~Kauers.
\newblock Guessing handbook.
\newblock Tech. Report 09-07, RISC Report Series, Johannes Kepler University Linz, Austria, 2009. \\
\href{http://www.risc.jku.at/publications/download/risc_3814/demo.nb.pdf}{http:/$\!$/www.risc.jku.at/publications/download/risc\_3814/demo.nb.pdf}


\bibitem{KoutschanThesis}
C.~Koutschan.
\newblock {\em Advanced applications of the holonomic systems approach}.
\newblock PhD thesis, Johannes Kepler University Linz, 2009.

\bibitem{Christoph2010}
C.~Koutschan.
\newblock {\em HolonomicFunctions user's guide}.
\newblock Tech. Reprot 10-01, RISC Report Series, Johannes Kepler University Linz, Austria, 2010. \\
\href{http://www.risc.jku.at/publications/download/risc_3934/hf.pdf}{http:/$\!$/www.risc.jku.at/publications/download/risc\_3934/hf.pdf}



\bibitem{KZ2018}
C.~Koutschan and Y.~Zhang.
\newblock Desingularization in the $q$-Weyl algebra. \\
\newblock {\em Advances in Applied Mathematics}, 97, pp.\ 80–101, 2018. 

\bibitem{Kovacic}
J.~J. Kovacic.
\newblock An algorithm for solving second order linear homogeneous differential
  equations.
\newblock {\em Journal of Symbolic Computation}, 2(1):3--43, 1986.

\bibitem{Nakayama2011}
H.~Nakayama, K.~Nishiyama, M.~Noro, K.~Ohara,T.~Sei, N.~Takayama, and A.~Takemura.
\newblock {Holonomic gradient descent and its application to the Fisher–Bingham integral.}
\newblock {\em Advances in Applied Mathematics}, 47(3): 639--658, 2011. 




\bibitem{NgoWinkler11b}
L.~X.~C. {Ng\^o} and F.~Winkler.
\newblock {Rational general solutions of parametrizable AODEs.}
\newblock {\em {Publicationes Mathematicae}}, 79(3-4):573--587, 2011.



\bibitem{PWZbook1996}
M.~Petkov{\v{s}}ek, H.~S.~Wilf, and D.~Zeilberger.
\newblock {\em {$A=B$}}.
\newblock A K Peters Ltd., Wellesley, MA, 1996.
\newblock With a foreword by Donald E. Knuth.

\bibitem{Riese03}
A.~Riese.
\newblock qMultiSum---A Package for Proving $q$-Hypergeometric Multiple Summation Identities.
\newblock {\em Journal of Symbolic Computation}, 35:349--376, 2003.


\bibitem{Eekelen2018}
O.~Shkaravska and M.~van Eekelen.
\newblock Polynomial solutions of algebraic difference equations and
  homogeneous symmetric polynomials.
\newblock 2018.

\bibitem{SingerFormalSolutions}
M.~F. Singer.
\newblock Formal solutions of differential equations.
\newblock {\em Journal of Symbolic Computation}, 10(1):59 -- 94, 1990.

\bibitem{Takayama2019}
N.~Takayama, J.~Lin, S.~Kuriki, and Y.~Zhang.
\newblock Computations of the Expected Euler Characteristic for the Largest Eigenvalue of a Real non-central Wishart Matrix. 
\newblock {\em arXiv:1903.10099}, 1--24, 2019. 


\bibitem{VoGraseggerWinkler2017}
N.~T. Vo, G.~Grasegger, and F.~Winkler.
\newblock Deciding the existence of rational general solutions for first-order
  algebraic {ODEs}.
\newblock {\em Journal of Symbolic Computation}, 2017.

\bibitem{VoWinkler2015}
N.~T. Vo and F.~Winkler.
\newblock {Algebraic general solutions of first order algebraic ODEs}.
\newblock In V.~P.~G. et. al., editor, {\em {Computer Algebra in Scientific
  Computing}}, volume 9301 of {\em Lecture Notes in Computer Science}, pages
  479--492. Springer International Publishing, 2015.
  
  

\bibitem{VoZhang2017}
N.T.~Vo and Y.~Zhang.
\newblock {Laurent Series Solutions of Algebraic Ordinary Differential Equations}.
\newblock {\em arXiv: 1709.04174}, pages 1--21,  2017.

\bibitem{VoZhang2019}
N.T.~Vo and Y.~Zhang.
\newblock {Rational Solutions of First-Order Algebraic Ordinary Difference Equations}.
\newblock {\em arXiv: 1901.11048}, pages 1--25,  2019.

\bibitem{Winkler2019}
F.~Winkler.
\newblock {The Algebraic-Geometric Method for Solving Algebraic Differential Equations}.
\newblock {to appear in Journal of System Sciences and Complexity}, 2019. 

\bibitem{Zhang2016}
Y.~Zhang.
\newblock {\em Contraction of {O}re ideals with applications}.
\newblock In {\em Proc.\ of ISSAC'16},  413--420, New York, NY, USA, 2016, ACM.



\end{thebibliography}
\end{document}
