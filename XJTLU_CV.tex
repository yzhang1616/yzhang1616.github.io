\documentclass[a4paper,12pt]{article}
\usepackage[utf8]{inputenc}
\usepackage{enumitem}
\usepackage[colorlinks = true,
            linkcolor = blue,
            urlcolor  = blue,
            citecolor = blue,
            anchorcolor = blue]{hyperref}
\usepackage{url}
\usepackage{graphicx,wrapfig,lipsum}
\usepackage{longtable}
\usepackage{fancyhdr}
% \usepackage[paperwidth=5.5in, paperheight=8.5in]{geometry}
\usepackage[margin=1in]{geometry}
% \usepackage[letterpaper, landscape, margin=1in]{geometry}


 
\pagestyle{fancy}
\fancyhf{}
\rhead{Page \thepage}
\lhead{Yi Zhang}
\chead{Curriculum Vitae}
\rhead{\thepage}
\newcommand{\blue}{\color{blue}}

\title{\bf{\Huge{Curriculum Vitae}}}
\author{}
\date{}

\begin{document}

\maketitle
\thispagestyle{empty}

\begin{picture}(2,2)
 \put(300,-150){\includegraphics[width=3cm]{Yi_Zhang}}
\end{picture}


\section*{\Large{Personal Information}}
\begin{tabular}{@{}p{1.2in}p{4in}}
Full name            & Yi Zhang \\
Date of birth        & 12.12.1988 \\
Place of birth       & Changzhou, Jiangsu Province, China \\
Nationality          & Chinese \\
%Marital Status       & Single 
\end{tabular}

\section*{\Large{Contact}}
\begin{tabular}{@{}p{1.2in}p{4in}}
E-mail           & \href{mailto:Yi.Zhang03@xjtlu.edu.cn}{Yi.Zhang03@xjtlu.edu.cn}  \\
Address          & Department of Foundational Mathematics \\ 
                 & Xi'an Jiaotong-Liverpool University \\
                 & 111 Ren'ai Road, Suzhou Dushu Lake Science and Education Innovation District \\
                 & Suzhou Industrial Park, Suzhou, China, 215123 \\
Office           & MB 251\\                
Phone            & +86 0512 8188 9056\\
Homepage         & \url{https://www.xjtlu.edu.cn/en/departments/academic-departments/foundational-mathematics/staff/yi-zhang03}
\end{tabular}

\section*{\Large{Research Interests}}
\begin{itemize}
\item  Computer Algebra
\item Algorithmic Combinatorics
\item Algebraic Theory of Differential and Difference Equations
\item Algebraic Statistics
\item Coding and Cryptography
\end{itemize}

\section*{\Large{Education}}
\begin{tabular}{@{}p{1.4in}p{4in}}
09/2013 -- 02/2017    & Ph.D. in Mathematics with distinction, 
                        \href{http://www.jku.at/algebra/content}{Institute for Algebra}, 
                        \href{http://www.jku.at/content}{Johannes Kepler University Linz}, Linz, Austria. 
                        (Co-supervisors: Prof. \href{http://www.kauers.de/}{Manuel Kauers} and 
                        Prof. \href{http://mmrc.iss.ac.cn/~zmli/}{Ziming Li})\\
09/2011 -- 07/2016    & Ph.D. in Applied Mathematics, 
                        \href{http://english.mmrc.amss.cas.cn/}{Key Laboratory of Mathematics Mechanization}, 
                        \href{http://english.amss.cas.cn/}{Academy of Mathematics and Systems Science}, 
                        \href{http://english.ucas.ac.cn/}{University of Academy of Sciences}, Beijing, China. 
                        (Co-supervisors: Prof. Manuel Kauers and Prof. Ziming Li)\\
09/2007 -- 07/2011    & B.Sc. in Mathematics, \href{http://math.suda.edu.cn/}{School of Mathematical Sciences}, 
                        \href{http://eng.suda.edu.cn/}{Soochow University}, Suzhou, China.  
\end{tabular} \\


\section*{\Large{Work Experience}}
\begin{tabular}{@{}p{1.4in}p{4in}}
02/2020 --   present              & Assistant Professor, \href{https://www.xjtlu.edu.cn/en/study/departments/academic-departments/foundational-mathematics/}{Department of Foundational Mathematics}, \href{https://www.xjtlu.edu.cn/en/}{Xi'an Jiaotong-Liverpool University}, Suzhou, China. \\
09/2018 -- 01/2020    & Research Associate, \href{https://www.utdallas.edu/math/}{Department of Mathematical Sciences}, 
                        \href{https://www.utdallas.edu/}{The University of Texas at Dallas}, Dallas, USA. 
                        (Advisor: Prof. \href{https://www.utdallas.edu/~arreche/}{Carlos E. Arreche}) \\    
03/2017 -- 08/2018    & Postdoctoral researcher, 
                        \href{https://www.ricam.oeaw.ac.at/}{Johann Radon Institute for Computational and Applied Mathematics} (RICAM),
                        \href{http://www.oeaw.ac.at/en/austrian-academy-of-sciences/}{Austrian Academy of Sciences}, Linz, Austria. 
                        (Advisor: Prof. \href{http://www.koutschan.de/}{Christoph Koutschan})\\
\end{tabular}

\section*{\Large{Visiting Experience}}
\begin{tabular}{@{}p{1.4in}p{4in}}
06/2019 -- 07/2019               & Visiting scholar, 
                        \href{https://fms.tdtu.edu.vn/}{Faculty of Mathematics and Statistics} ,
                        \href{https://tdtu.edu.vn/}{Ton Duc Thang University}, Ho Chi Minh City, Vietnam. 
                        (Host researcher: Dr. \href{https://sites.google.com/tdtu.edu.vn/vongocthieu}{Thieu N. Vo})\\
05/2017               & Visiting scholar, 
                        \href{http://www.math.kobe-u.ac.jp/}{Department of Mathematics} ,
                        \href{http://www.kobe-u.ac.jp/en/}{Kobe University}, Kobe, Japan. 
                        (Host researcher: Prof. \href{http://www.math.kobe-u.ac.jp/home-j/takayama-e.html}{Nobuki Takayama})\\
\end{tabular}


\section*{\Large{Teaching Experience}}
\begin{tabular}{@{}p{1.4in}p{4in}} 
Spring 2023 & Lecturer. (Multivariate Calculus) \\
             & Xi'an Jiaotong - Liverpool University, Suzhou, China.   \\
Fall 2022 & Lecturer. (Linear Algebra) \\
             & Xi'an Jiaotong - Liverpool University, Suzhou, China.   \\
Spring 2022 & Lecturer. (Multivariate Calculus) \\
             & Xi'an Jiaotong - Liverpool University, Suzhou, China.   \\
Fall 2021 & Lecturer. (Analysis 1) \\
               & Xi'an Jiaotong - Liverpool University, Suzhou, China.   \\
Spring 2021 & Lecturer. (Multivariate Calculus) \\
             & Xi'an Jiaotong - Liverpool University, Suzhou, China.   \\
Fall 2020 & Lecturer. (Analysis 1) \\
               & Xi'an Jiaotong - Liverpool University, Suzhou, China.   \\
Spring 2020 & Lecturer. (Multivariate Calculus) \\
             & Xi'an Jiaotong - Liverpool University, Suzhou, China.   \\
Fall 2019 & Instructor. (\href{https://yzhang1616.github.io/calculus19fall/calculus.html}{Integral Calculus}) \\
             & The University of Texas at Dallas, Dallas, USA.   \\
Spring 2019           & Instructor. (\href{https://yzhang1616.github.io/algebra19spring/algebra.html}{Linear Algebra}) \\
                      & The University of Texas at Dallas, Dallas, USA. \\
Fall 2010             & Teaching internship. (High School Mathematics) \\
                      & Suzhou High School Affiliated to Xi'an Jiaotong University, Suzhou, China.                        
\end{tabular}

\section*{\Large{Professional Service Activities}}
\begin{tabular}{@{}p{1.4in}p{4in}} 
2023 -- present & UoL (University of Liverpool) Coordinator. \\
                           & Xi'an Jiaotong - Liverpool University, Suzhou, China.  \\
2021 -- 2023 & PGRS (Postgraduate Research Scholarship) Panel Member. \\
                           & Xi'an Jiaotong - Liverpool University, Suzhou, China.  \\
2020 -- 2021 & Global Engagement Officer. \\
             & Xi'an Jiaotong - Liverpool University, Suzhou, China.                  
\end{tabular}

\section*{\Large{Professional Memberships}}
\begin{tabular}{@{}p{1.4in}p{4in}} 
2021 -- present & Fellow of The Higher Education Academy. 
In recognition of attainment against the UK
Professional Standards Framework for teaching and learning support in higher education.
Fellowship reference: PR227162. Date of Fellowship: August 24th, 2021. \\
2018 -- present &  Reviewer for Mathematical Reviews/MathSciNet.            
\end{tabular}


\section*{\Large{Awards}}
\begin{tabular}{@{}p{1.4in}p{4in}}
12/2022 &  2022 Backbone Scientific and Education Talents (RMB 40, 000), Suzhou, China.  \\
07/2021         & 2021 Jiangsu Province Innovation \& Entrepreneurship Doctor-Talent Program (RMB 150, 000), 
                       Jiangsu, China. \\
07/2016               & \href{https://www.sigsam.org/Awards/ISSACAwards.html}{ISSAC'16 Distinguished Student Author Award}, \\
                      & SIGSAM, Association for Computing Machinery (ACM). \\
09/2009 -- 07/2010    & The Second Prize Scholarship of Soochow University, \\ 
                      & Suzhou, China.\\
09/2008 -- 07/2009    & The First Prize Scholarship of Soochow University, \\ 
                      & Suzhou, China.\\
09/2007 -- 07/2008    & The First Prize Scholarship of Soochow University, \\ 
                      & Suzhou, China.\\
09/2007 -- 07/2008    & The Zhu Jingwen Scholarship of Soochow University, \\ 
                      & Suzhou, China.\\
09/2007 -- 07/2008    & The Merit Student of Soochow Universtiy, \\ 
                      & Suzhou, China.\\
\end{tabular}

\section*{\Large{Grants}}
\begin{tabular}{@{}p{1.4in}p{4in}}
10/2021 -- 09/2023 & Natural Science Foundation of the Jiangsu Higher Education Institutions of China Programme-General Programme, 
                                       \textbf{Principle Investigator}. No.\ 21KJB110032. \\
                                   & ``Algebraic and Numerical Analysis of Differential Equations based on Computer Algebra'', RMB 30, 000. \\     
01/2022 -- 12/2024 & National Science Foundation of China Young Scientist Fund, \textbf{Principle Investigator}. No. 12101506. \\
                                 & ``Theory and Algorithms of Algebraic Differential and Difference Equations based on Symbolic Computation'',
                                 RMB 300, 000. \\
03/2021 -- 02/2024    & XJTLU Research Development Fund, \textbf{Principle Investigator}. No. RDF-20-01-12. RMB 60, 000.

\end{tabular}

\section*{\Large{PhD Thesis}}
\begin{itemize}
 \item Yi Zhang. \href{https://yzhang1616.github.io/yzhang_PhDthesis_final.pdf}{{\em Univarite 
                Contraction and Multivariate Desingularization of Ore Ideals}}. 
                PhD thesis, Institute for Algebra, Johannes Kepler University Linz, 2017. 
                arXiv:\href{https://arxiv.org/abs/1710.07445}{1710.07445}
\end{itemize}

\section*{\Large{Publications}}

\subsection*{Published}
\begin{enumerate}
\item Carlos E.\ Arreche and Yi Zhang (corresponding author). 
{\em Mahler Discrete Residues and Summability for Rational Functions}. In {\em Proceedings of the 2022 International Symposium on Symbolic and Algebraic Computation (ISSAC)}, 
pp.\ 525–533, ACM Press, 2022. DOI:\href{https://dl.acm.org/doi/10.1145/3476446.3536186}{10.1145/3476446.3536186},
arXiv:\href{https://arxiv.org/abs/2202.09805}{2202.09805}.  (EI, ISSAC is a top international conference in the field ``Algorithms and Theory'', NUS evaluation: Rank 1, AUS evaluation: A+, MSAR field rating: Rank 1, conference ranks: A) 
\item Alin Bostan, Jordan Tirrell, Bruce W.\ Westbury and Yi Zhang (corresponding author). 
{\em On Some Combinatorial Sequences Associated to Invariant Theory}, European Journal of Combinatorics, 105, 103554, 2022. 
DOI:\href{https://doi.org/10.1016/j.ejc.2022.103554}{10.1016/j.ejc.2022.103554}, 
 arXiv:\href{https://arxiv.org/abs/2110.13753}{2110.13753}.  (SCI)
\item Sebastian Falkensteiner, Yi Zhang (co-first author) and Thieu N. Vo. 
{\em On Existence and Uniqueness of Formal Power Series Solutions of Algebraic Ordinary
     Differential Equations},  Mediterranean Journal of Mathematics, 19, 74, 2022. 
DOI:\href{https://doi.org/10.1007/s00009-022-01984-w}{10.1007/s00009-022-01984-w}, 
arXiv:\href{https://arxiv.org/abs/1803.09646}{1803.09646}. (SCI) 
\item Carlos E.\ Arreche and Yi Zhang (corresponding author). 
{\em Computing Differential Galois Groups of Second-order Linear q-Difference Equations}, Advances in Applied Mathematics, 132, 102273, 2022. 
DOI:\href{https://doi.org/10.1016/j.aam.2021.102273}{10.1016/j.aam.2021.102273}, 
arXiv:\href{https://arxiv.org/abs/2009.14026}{2009.14026}. (SCI)
\item Zhimin Sun,  Xiangyong Zeng, Lin Yi and Yi Zhang (corresponding author). 
{\em The Expansion Complexity of Ultimately Periodic Sequences over Finite Fields},  IEEE Transactions on Information Theory, 67 (11), pp.\ 7550-7560, 2021. \\ 
DOI:\href{https://doi.org/10.1109/TIT.2021.3112824}{10.1109/TIT.2021.3112824}. (SCI, JCR Q1)
 \item Nobuki Takayama, Lin Jiu, Satoshi Kuriki and Yi Zhang (corresponding author). 
 {\em Computation of the Expected Euler Characteristic for the Largest Eigenvalue of a Real Wishart Matrix}, Journal of Multivariate Analysis, 179, 104642, 2020. \\
 DOI:\href{https://doi.org/10.1016/j.jmva.2020.104642}{10.1016/j.jmva.2020.104642}, 
 arXiv:\href{http://arxiv.org/abs/1903.10099}{1903.10099}. (SCI)
 \item Thieu N. Vo and Yi Zhang (corresponding author). 
{\em Rational Solutions of First-Order Algebraic Ordinary Difference Equations},  Advances in Applied Mathematics, 117, 102018, 2020. 
DOI:\href{https://doi.org/10.1016/j.aam.2020.102018}{10.1016/j.aam.2020.102018},
arXiv:\href{http://arxiv.org/abs/1901.11048}{1901.11048}. (SCI)
 \item Thieu N. Vo and Yi Zhang (corresponding author). {\em Rational Solutions of High-Order Algebraic Ordinary Differential Equations},  {\em Journal of Systems Science and Complexity}, 33, pp.\ 821-835, 2020. DOI:\href{https://link.springer.com/article/10.1007/s11424-019-8133-0}{10.1007/s11424-019-8133-0}, \\
 arXiv:\href{https://arxiv.org/abs/1709.04174}{1709.04174}. (SCI)
  \item Shaoshi Chen, Manuel Kauers, Ziming Li and Yi Zhang (corresponding author). {\em Apparent Singularities of D-finite Systems}, 
 {\em  Journal of Symbolic Computation},  95, pp.\ 217-237, 2019. DOI:\href{https://doi.org/10.1016/j.jsc.2019.02.009}{10.1016/j.jsc.2019.02.009}, arXiv:\href{http://arxiv.org/abs/1705.00838}{1705.00838}. (SCI)
\item Ting Guo, Christian Krattenthaler and Yi Zhang (corresponding author).
{\em On (shape-)Wilf-equivalence for Words}, 
{\em  Advances in Applied Mathematics} , 100, pp.\ 87-100, 2018. 
DOI:\href{https://doi.org/10.1016/j.aam.2018.05.006}{10.1016/j.aam.2018.05.006}, 
arXiv:\href{https://arxiv.org/pdf/1802.09856.pdf}{1802.09856}. (SCI)
\item Christoph Koutschan and Yi Zhang (corresponding author). {\em Desingularization in the $q$-Weyl Algebra}. 
{\em Advances in Applied Mathematics}, 97, pp.\ 80–101, 2018. 
DOI: \href{http://dx.doi.org/10.1016/j.aam.2018.02.005}{10.1016/j.aam.2018.02.005},
arXiv:\href{https://arxiv.org/abs/1801.04160}{1801.04160}. (SCI) 
\item Yi Zhang. {\em Contraction of Ore Ideals with Applications}. 
In {\em Proceedings of the 2016 International Symposium on Symbolic and Algebraic Computation (ISSAC)}, 
pp.\ 413-420, ACM Press, 2016. DOI:\href{http://dl.acm.org/citation.cfm?id=2930890}{10.1145/2930889.2930890},
arXiv:\href{https://arxiv.org/abs/1511.07922}{1511.07922}. 
\href{https://www.sigsam.org/Awards/ISSACAwards.html}{[Distinguished Student Author Award]} 
(EI, ISSAC is a top international conference in the field ``Algorithms and Theory'', NUS evaluation: Rank 1, AUS evaluation: A+, MSAR field rating: Rank 1, conference ranks: A) 
\end{enumerate}

\subsection*{Submitted/in preparation}
\begin{enumerate}
\item Dmitrii Karp and Yi Zhang. {\em Asymptotic Approximations and Bounds for the Incomplete Elliptic Integral of the Second Kind near the Logarithmic Singularity}, August, 2022, submitted. 
arXiv:\href{https://arxiv.org/abs/2208.05242}{2208.05242}. 
\item Yi Zhang. 
{\em Desingularization of Linear Mahler Operators}, 2021, in preparation.
%\item Yi Zhang, Sebastian Falkensteiner and Thieu N. Vo. 
%{\em On Formal Power Series Solutions of Algebraic Ordinary
%     Differential Equations}, 2019, arXiv:\href{https://arxiv.org/abs/1803.09646}{1803.09646}, submitted. 
\end{enumerate}

\section*{\Large{Research Note}}
\begin{itemize}
%\item Yi Zhang. {\em Testing q-shift Equivalence of Polynomials}, July, 2017.
%\item Yi Zhang. {\em Integer Vectors of a Fundamental Parallelepiped}, 2016.
 \item Ziming Li and Yi Zhang. {\em A Note on Gr\"{o}bner Bases of Ore Polynomials over a PID}, 2016. 
 \url{https://yzhang1616.github.io/GB.pdf} 
\end{itemize}

\section*{\Large{Software}}
\begin{itemize}
\item \href{https://yzhang1616.github.io/lch/Discrete_Log_Concavity_Hyper.nb}{Discrete\_Log\_Concavity\_Hyper.nb}, a Mathematica notebook for proving the discrete log-concavity of certain hypergeometric functions by using Cylindrical Algebraic Decomposition. It is based on joint work with Dmitrii Karp.
\item \href{https://yzhang1616.github.io/fd/Fake_Degree_Sequence.nb}{Fake\_Degree\_Sequence.nb},  a Mathematica notebook for finding a linear~$q$-difference equation satisfied by the fake degree sequence associated to a given representation and a given simple complex Lie algebra by the method of creative telescoping and the closure properties of the class of q-holonomic sequences.  It is based on joint work
    with Bruce W.\ Westbury.  The notebook requires the availability of Koutschan's package 
 \href{http://www.risc.jku.at/research/combinat/software/ergosum/RISC/HolonomicFunctions.html}{HolonomicFunctions.m}. 
\item \href{https://yzhang1616.github.io/ct/Mihailovs_Conjecture.nb}{Mihailovs\_Conjecture.nb},  a Mathematica notebook for 
proving Mihailovs' conjecure by the method of creative telescoping. It is based on joint work
with Jordan Tirrell and Bruce W.\ Westbury.   The notebook requires the availability of Koutschan's package 
 \href{http://www.risc.jku.at/research/combinat/software/ergosum/RISC/HolonomicFunctions.html}{HolonomicFunctions.m}.
\item \href{https://yzhang1616.github.io/complexity/ansatz.m}{ansatz.m}, 
a Mathematica package for computing the expansion complexity of a given finite length sequences. 
It is based on joint work with Zhimin Sun and Xiangyong Zeng. 
  \item \href{https://yzhang1616.github.io/TestNonvanishing.nb}{TestNonvanishing.nb}, 
    a Mathematica notebook for checking the nonvanishing property of algebraic ordinary
    differential equations in Kamke's collection. It is based on joint work
    with Sebastian Falkensteiner and N.\ Thieu Vo. 
    The notebook requires the availability of the Mathematica package \href{https://yzhang1616.github.io/Kamke_ODE.m}{Kamke\_ODE.m}.
  \item \href{https://yzhang1616.github.io/zof/zof.m}{zof.m}, a Mathematica package for generating $0$-$1$-fillings 
  of a Ferrers board (shape), checking the number of
    sigma-avoiding $0$-$1$-fillings of a Ferrers board, 
     generating generalized $0$-$1$-fillings of a Ferrers board, 
     and checking the number of generalized $0$-$1$-fillings of a Ferrers board with weight $n$
    such that the longest ne-chain has length $u$ 
    and the longest se-chain has length $v$. It is based on joint work with Ting
    Guo and Christian Krattenthaler. For a demonstration of the package,
    see the \href{https://yzhang1616.github.io/zof/zof.nb}{zof.nb} notebook. 
  \item \href{https://yzhang1616.github.io/ec1/Example1_HGM.nb}{Example1\_HGM.nb}, a Mathematica notebook for
    the demonstration of the holonomic gradient method for the evaluation of
    expection of an Euler characteristic number. It is based on joint work
    with Satoshi Kuriki and Nobuki Takayama. 
    The notebook requires the availability of Koutschan's package 
     \href{http://www.risc.jku.at/research/combinat/software/ergosum/RISC/HolonomicFunctions.html}{HolonomicFunctions.m}.
  \item \href{https://yzhang1616.github.io/KamkeODEs.mw}{KamkeODEs.mw}, a Maple worksheet for 
     checking the (completely) maximal comparability and noncriticality of algebraic
     ordinary differential equations in
     Kample's collection. It is based on joint work with Dr. Thieu Vo Ngoc. 
     The worksheet requires the availability of the Maple package \href{https://yzhang1616.github.io/KamkeODEs.mpl}{KamkeODEs.mpl}.
 \item \href{https://yzhang1616.github.io/qDesingularization.m}{qDesingularization.m}, a Mathematica
     package for computing desingularized operators and the $q$-Weyl closure of
     a given $q$-difference operator in
     the first $q$-Weyl algebra. It is based on joint work with Dr. Christoph
     Koutschan. The package requires the availability of Koutschan's package
     \href{http://www.risc.jku.at/research/combinat/software/ergosum/RISC/HolonomicFunctions.html}{HolonomicFunctions.m}
     and Kauer's pacakge \href{https://www.risc.jku.at/research/combinat/risc/software/Singular/index.html}{Singular.m}.
     For a description of the usage of the package, see the \href{https://yzhang1616.github.io/Example.nb}{Example.nb} notebook.
\end{itemize}

\section*{\Large{Talks}}
\begin{enumerate}
\item {\em Mahler Discrete Residues and Summability for Rational Functions}. 
Invited talk at at the 6th Conference on Combinatorics and Symbolic Computation, School of Mathematical Sciences, Luoyang Normal University, Luoyang, China, May, 2023. 
\item {\em Mahler Discrete Residues and Summability for Rational Functions}. 
Invited talk at School of Mathematical Sciences, Shang Jiao Tong University, Shanghai, China, December, 2022. 
\item {\em Mahler Discrete Residues and Summability for Rational Functions}. 
Contributed talk at the 3rd Suzhou Area Youth Mathematicians Workshop, School of Mathematics and Physics, Xi'an Jiaotong-Liverpool  University, Suzhou, China, November, 2022. 
\item {\em Mahler Discrete Residues and Summability for Rational Functions}. 
Invited talk at School of Mathematics, Beihang University, Beijing, China, October, 2022. 
\item {\em Mahler Discrete Residues and Summability for Rational Functions}. 
Invited talk at School of Mathematics, Shandong University, Jinan, China, September, 2022. 
\item {\em Apparent Singularities of D-finite Systems}.
Invited talk at Key Laboratory of Mathematics Mechanization, Academy of Mathematics and Systems Sciences,
 Chinese Academy of Sciences, Beijing, China, October, 2021. 
 \item {\em Desingularization in the q-Weyl algebra}. 
 Invited talk at the Workshop on Combinatorics, $q$-series, and Symbolic Computation in Suzhou Area, School of Mathematical Sciences, Soochow University, Suzhou, China, October, 2021.
\item {\em On Sequences Associated to the Invariant Theory of Rank Two Lie Algebras}.
Invited talk at School of Mathematical Sciences, Soochow University, Suzhou, 
China, September, 2021.
\item {\em On Sequences Associated to the Invariant Theory of Rank Two Lie Algebras}.
Contributed talk at the 26th International Conference on Applications of Computer Algebra (ACA'21), Virtual, Online, July, 2021.
\item {\em On Sequences Associated to the Invariant Theory of Rank Two Lie Algebras}.
Contributed talk at the 12th Annual Conference on Computer Mathematics (CM'21), Guilin, China, June, 2021.
\item {\em On Sequences Associated to the Invariant Theory of Rank Two Lie Algebras}.
Invited talk at School of Mathematics, Beihang University, Beijing, China, October, 2020.
\item {\em On Sequences Associated to the Invariant Theory of Rank Two Lie Algebras}.
Invited talk at Mahidol University International College, Bangkok, Thailand, January, 2020.
\item {\em Apparent Singularities of D-finite Systems}.
Invited talk at Kolchin Seminar, CUNY Graduate Center, New York, USA, December, 2019. 
\item {\em Computations of the Expected Euler Characteristic for the Largest Eigenvalue of a Real Wishart Matrix}.
Contributed talk at SIAM TX/LA Section (The 2nd Annual Meeting of the SIAM Texas Louisiana Section), Southern Methodist University, Dallas, USA, November, 2019.
\item {\em Computations of the Expected Euler Characteristic for the Largest Eigenvalue of a Real Wishart Matrix}.
Invited talk at Key Laboratory of Mathematics Mechanization, Academy of Mathematics and Systems Sciences,
 Chinese Academy of Sciences, Beijing, China, July, 2019. 
\item {\em Computations of the Expected Euler Characteristic for the Largest Eigenvalue of a Real Wishart Matrix}.
Invited talk at Johann Radon Institute for Computational and Applied Mathematics (RICAM), Austrian Academy of Sciences, 
Austria, May, 2019.
 \item {\em Desingularization in the q-Weyl algebra}. 
 Invited talk at Key Laboratory of Mathematics Mechanization, Academy of Mathematics and Systems Sciences,
 Chinese Academy of Sciences, Beijing, China, July, 2018. 
 \item {\em Desingularization in the $q$-Weyl algebra}. 
 Contributed talk at at ACA'18 
 \ (the 24th Conference on Applications of Computer Algebra), the Faculty of Mathematics, 
 The University of Santiago de Compostela, Santiago, Spain, June, 2018.
 \item {\em Laurent Series Solutions of Algebraic Ordinary Differential Equations}. 
 Invited talk at Computer Algebra Seminar, Research Institute for Symbolic Computation (RISC), Johannes Kepler University Linz, 
 Austria, November, 2017.
 \item {\em Apparent Singularities of D-finite Systems}. Contributed talk at ACA'17 
 \ (the 23rd Conference on Applications of Computer Algebra), Jerusalem College of Technology, Jerusalem, Israel, July, 2017.
 \item {\em Contraction of Linear Difference and Differential Operators}. Contributed talk at ISSAC'16 
 \ (the 41st International Symposium on Symbolic and Algebraic Computation), Wilfrid Laurier University, Waterloo, Canada, July, 2016.
 \item {\em Contraction of Linear Difference and Differential Operators}.
       Invited talk at the seminar of Center for Combinatorics, Nankai University, Tianjin, China, June, 2016.
 \item {\em An Algorithm for Contraction of an Ore Ideal}. Invited talk at the seminar of Institute of Discrete Mathematics and Geometry, 
       Vienna University of Technology, Vienna, Austria, October, 2015.
 \item {\em The Restriction Problem for D-finite Functions}. 
       Contributed talk at the Workshop on Computational and Algebraic Methods in Statistics,
       The University of Tokyo, Tokyo, Japan, March, 2015.
 \item {\em An Algorithm for Decomposing Multivariate Hypergeometric Terms}. Contributed talk at CM'13
       \ (the 5th National Conference of Computer Mathematics), Jilin University, Changchun, China, August, 2013.
\end{enumerate}

\section*{\Large Peer-Reviewing Activities}
For each journal and conference the number of completed reviews in given in parentheses.
\begin{itemize}
\item Journal of Difference Equations and Applications (1)
\item Mathematical Reviews (1)
\item Maple Conference (1)
\item Mediterranean Journal of Mathematics (1)
\item Conferences on Applications of Computer Algebra (1)
 \item Journal of Systems Science and Complexity (4)
 \item Journal of Computational and Applied Mathematics (1)
 \item Advances in Applied Mathematics (1)
 \item International Symposiums on Symbolic and Algebraic Computation (5)
 \item Journal of Symbolic Computation (3)
\end{itemize}

\section*{\Large{Further Skills}}
\begin{itemize}
 \item Programming Skills: C, Matlab, Maple, Mathematica, Macaulay2, Sage, and Python
 \item Spoken Language: Chinese (native), English (fluent), and German (basic)
\end{itemize}

%\section*{\Large{Research Plan}}

%\section*{\Large{Teaching Philosophy}}
%
%As Socrates said, “Education is not the filling of a vessel, but the kinding of a flame”. 
%A good teacher not only teach students key ideas and techniques of a course, but also motivate their interests so that they will learn and 
%study by themselves in the future. 
%Meanwhile, a teacher shall also show the general methodology for learning a course so that they know how to learn it by themselves. 
%There are general goals I want to achieve for my teaching. 
%Under this philosophy, I teach my courses with the following schemes: 
%
%\begin{itemize}
%\item Design lecture notes for audience. 
%Before my lectures, I investigate mathematical backgrounds of my students by checking their previous course lists online. 
%On the other hand, I also ask some of my senior colleagues about their experience in teaching the same course. 
%This is very helpful in preparing my lecture notes because with those information 
%I know how to organize the material so that it is not so hard for my students. 
%When I design my lecture notes, I first recall some key results in the previous lecture so that students 
%can have a  retrospect about what they learned before at the very beginning. 
%In the final part of my notes, I give a summary about all the ingredients of the current lecture. 
%After my lectures, I scan and submit my notes online so that students can download and read them for the convenience of their study.
%
%\item General guides in the first lecture. In my first lecture, I give an outline of the content of the course, such as: 
%what are objects we are going to study;
%some historical backgrounds of the topic; main problems and results in this area; possible applications in related areas. 
%Besides, I also give some general suggestions for studying: 1.\ read the textbook and attend my lectures regularly; 
%2.\ discuss with classmates and me (after lectures and during my office hours); 
%3.\ use library properly; 4.\ finish homework independently as much as possible (discussion is encouraged, 
%but plagiarism is definitely prohibited).  
%
%\item Balance between time and material. Each lecture has fixed time and is scheduled to teach one section 
%(sometimes a half one) from the designated textbook. 
%This requires the instructor or lecturer to prepare lecture notes with balance between time and material. 
%For instance, sometimes one section contains five subsections (or even more). 
%It would be very time-consuming to go through every detail of each subsection. In that case, 
%I always try to 
%summarize key ideas of each subsection and show students how to apply them to do concrete examples. 
%%If the time is still very tight, I simply give some remarks in the lecture 
%%and suggest students to learn that part by self-reading.  
%On the other hand, if the time is quite abundant, 
%I would give more concrete examples, details and some historical backgrounds of the topic during the class.  
%%After my lectures, I will scan and submit my lecture notes online so that some students can download and read them for the convenience of their study.  
%
%\item Teaching with concrete examples. 
%Mathematics is abstract because it contains a lot of definitions, lemmas, propositions and theorems. 
%I find that it is an excellent idea to teach with concrete examples, especially for those students who are not from 
%the mathematical department. Actually, classical examples not only illustrate abstract concepts and important ideas in mathematics, 
%but also show students how to apply them to solve practical problems. 
%Last but not least, examples help students understand, review and memorize 
%key points of materials in lectures.  
%
%\item Communicating with students. Roughly speaking, teaching a course is all about communicating with students, 
%exchanges ideas and thoughts with them. During my lectures, I always speak loudly, clearly and slowly so that my students 
%can hear what I am talking about. Besides, I also write in the backboard with large enough handwriting and 
%try to have suitable eye contact with students. Sometimes, they ask me some questions during the class. 
%I always try to answer their questions concisely and clearly. On the other hand, their questions, comments and suggestions also help me 
%improve the quality of my lectures and learn how to explain certain materials of the course in a more proper way. 
%Moreover, I help them as possible as I can through office hours and emails.     
%\end{itemize}
%
%\subsection*{\large{Teaching Experience}}
%\begin{tabular}{@{}p{1.4in}p{4in}} 
%Spring 2019           & Instructor. (\href{https://yzhang1616.github.io/algebra19spring/algebra.html}{Linear Algebra}) \\
%                      & The University of Texas at Dallas, USA. \\
%Fall 2010             & Teaching internship. (High School Mathematics) \\
%                      & Suzhou High School Affiliated to Xi'an Jiaotong University, China.   
%                  
%\end{tabular}

%\section*{\Large{Reference Providers}}
%
%\begin{itemize}
%\item Prof. \href{http://www.kauers.de/}{Manuel Kauers}, \href{http://www.jku.at/algebra/content}{Institute for Algebra},  
%         \href{http://www.jku.at/content}{Johannes Kepler University Linz}, Austria. Phone: +43 (0)732 2468 6850 (secretary), +43 (0)732 2468 9958 (direct). Email: \href{mailto:manuel@kauers.de}{manuel@kauers.de}
%
%\item Prof. \href{http://www.koutschan.de/}{Christoph Koutschan}, \href{https://www.ricam.oeaw.ac.at/}{Johann Radon Institute for Computational and Applied Mathematics} (RICAM),
%                        \href{http://www.oeaw.ac.at/en/austrian-academy-of-sciences/}{Austrian Academy of Sciences}, Austria. Phone: +43 732 2468 5254. 
%                        Email: \href{mailto:christoph@koutschan.de}{christoph@koutschan.de}
%
%\item Prof. \href{https://www.utdallas.edu/~arreche/}{Carlos E. Arreche}, \href{https://www.utdallas.edu/math/}{Department of Mathematical Sciences}, 
%                        \href{https://www.utdallas.edu/}{The University of Texas at Dallas}, USA. Phone:  +1 (972) 883 6594. 
%                        Email: \href{mailto:arreche@utdallas.edu}{arreche@utdallas.edu}
%\end{itemize}
%
%Prof. Carlos E. Arreche can comment on my teaching at UT Dallas. 

% C, Matlab, Maple, Mathematica, Macaulay2 and Sage
% 
% \section*{\Large{Hobbies and Interests}}
% \begin{itemize}
%  \item Sports: Table Tennis, Billiards, Tennis.
%  \item Reading: Philosophy, History, Literature.
%  \item Language: Chinese (native), English (fluent), German (basic).
% \end{itemize}

\end{document}
%\documentclass[a4paper,12pt]{article}
%\usepackage[utf8]{inputenc}
%\usepackage{enumitem}
%\usepackage[colorlinks = true,
%            linkcolor = blue,
%            urlcolor  = blue,
%            citecolor = blue,
%            anchorcolor = blue]{hyperref}
%\usepackage{url}
%\usepackage{graphicx,wrapfig,lipsum}
%\usepackage{longtable}
%\usepackage{fancyhdr}
%% \usepackage[paperwidth=5.5in, paperheight=8.5in]{geometry}
%\usepackage[margin=1in]{geometry}
%% \usepackage[letterpaper, landscape, margin=1in]{geometry}
%
%
% 
%\pagestyle{fancy}
%\fancyhf{}
%\rhead{Page \thepage}
%\lhead{Yi Zhang}
%\chead{Curriculum Vitae}
%\rhead{\thepage}
%\newcommand{\red}{\color{red}}
%
%\title{\bf{\Huge{Curriculum Vitae}}}
%\author{}
%\date{}
%
%\begin{document}
%
%\maketitle
%\thispagestyle{empty}
%
%\begin{picture}(2,2)
% \put(300,-150){\includegraphics[width=3cm]{Yi_Zhang}}
%\end{picture}
%
%
%\section*{\Large{Personal Data}}
%\begin{tabular}{@{}p{1.2in}p{4in}}
%Full name            & Yi Zhang \\
%Date of birth        & 12.12.1988 \\
%Place of birth       & Changzhou, Jiangsu Province, China \\
%Nationality          & Chinese \\
%Marital Status       & Single 
%\end{tabular}
%
%\section*{\Large{Contact}}
%\begin{tabular}{@{}p{1.2in}p{4in}}
%E-mail           & \href{mailto:Yi.Zhang03@xjtlu.edu.cn}{Yi.Zhang03@xjtlu.edu.cn}  \\
%Address          & Department of Applied Mathematics \\ 
%                 & Xi'an Jiaotong-Liverpool University \\
%                 & 111 Ren'ai Road, Suzhou Dushu Lake Science and Education Innovation District \\
%                 & Suzhou Industrial Park, Suzhou, China, 215123 \\
%Office           & MB 421B\\                
%Phone            & +86 0512 8188 9056\\
%Homepage         & \url{https://www.xjtlu.edu.cn/zh/departments/academic-departments/mathematical-sciences/staff/yi-zhang}
%\end{tabular}
%
%\section*{\Large{Research Interests}}
%\begin{itemize}
%\item  Computer Algebra
%\item Algorithmic Combinatorics
%\item Algebraic Theory of Differential and Difference Equations
%\item Algebraic Statistics
%\end{itemize}
%
%\section*{\Large{Education}}
%\begin{tabular}{@{}p{1.4in}p{4in}}
%09/2013 -- 02/2017    & Ph.D. in Mathematics with distinction, 
%                        \href{http://www.jku.at/algebra/content}{Institute for Algebra}, 
%                        \href{http://www.jku.at/content}{Johannes Kepler University Linz}, Austria. 
%                        (Co-supervisors: Prof. \href{http://www.kauers.de/}{Manuel Kauers} and 
%                        Prof. \href{http://mmrc.iss.ac.cn/~zmli/}{Ziming Li})\\
%09/2011 -- 07/2016    & Ph.D. in Applied Mathematics, 
%                        \href{http://english.mmrc.amss.cas.cn/}{Key Laboratory of Mathematics Mechanization}, 
%                        \href{http://english.amss.cas.cn/}{Academy of Mathematics and Systems Science}, 
%                        \href{http://english.ucas.ac.cn/}{University of Academy of Sciences}, Beijing, China. 
%                        (Co-supervisors: Prof. Manuel Kauers and Prof. Ziming Li)\\
%09/2007 -- 07/2011    & B.Sc. in Mathematics, \href{http://math.suda.edu.cn/}{School of Mathematical Sciences}, 
%                        \href{http://eng.suda.edu.cn/}{Soochow University}, Suzhou, China.  
%\end{tabular} \\
%
%
%\section*{\Large{Work Experience}}
%\begin{tabular}{@{}p{1.4in}p{4in}}
%02/2020 --   present              & Assistant Professor, \href{https://www.xjtlu.edu.cn/en/study/departments/academic-departments/applied-mathematics/}{Department of Applied Mathematics}, \href{https://www.xjtlu.edu.cn/en/}{Xi'an Jiaotong-Liverpool University}, Suzhou, China. \\
%09/2018 -- 01/2020    & Research Associate, \href{https://www.utdallas.edu/math/}{Department of Mathematical Sciences}, 
%                        \href{https://www.utdallas.edu/}{The University of Texas at Dallas}, Dallas, USA. 
%                        (Advisor: Prof. \href{https://www.utdallas.edu/~arreche/}{Carlos E. Arreche}) \\    
%03/2017 -- 08/2018    & Postdoctoral researcher, 
%                        \href{https://www.ricam.oeaw.ac.at/}{Johann Radon Institute for Computational and Applied Mathematics} (RICAM),
%                        \href{http://www.oeaw.ac.at/en/austrian-academy-of-sciences/}{Austrian Academy of Sciences}, Linz, Austria. 
%                        (Advisor: Prof. \href{http://www.koutschan.de/}{Christoph Koutschan})\\
%\end{tabular}
%
%\section*{\Large{Visiting Experience}}
%\begin{tabular}{@{}p{1.4in}p{4.5in}}
%06/2019 -- 07/2019               & Visiting scholar, 
%                        \href{https://fms.tdtu.edu.vn/}{Faculty of Mathematics and Statistics} ,
%                        \href{https://tdtu.edu.vn/}{Ton Duc Thang University}, Ho Chi Minh City, Vietnam. 
%                        (Host researcher: Dr. \href{https://sites.google.com/tdtu.edu.vn/vongocthieu}{Thieu N. Vo})\\
%05/2017               & Visiting scholar, 
%                        \href{http://www.math.kobe-u.ac.jp/}{Department of Mathematics} ,
%                        \href{http://www.kobe-u.ac.jp/en/}{Kobe University}, Kobe, Japan. 
%                        (Host researcher: Prof. \href{http://www.math.kobe-u.ac.jp/home-j/takayama-e.html}{Nobuki Takayama})\\
%\end{tabular}
%
%
%\section*{\Large{Teaching Experience}}
%\begin{tabular}{@{}p{1.4in}p{4in}} 
%Fall 2021 & Lecturer. (Analysis 1) \\
%               & Xi'an Jiaotong - Liverpool University, Suzhou, China.   \\
%Spring 2021 & Lecturer. (Multivariate Calculus) \\
%             & Xi'an Jiaotong - Liverpool University, Suzhou, China.   \\
%Fall 2020 & Lecturer. (Analysis 1) \\
%               & Xi'an Jiaotong - Liverpool University, Suzhou, China.   \\
%Spring 2020 & Lecturer. (Multivariate Calculus) \\
%             & Xi'an Jiaotong - Liverpool University, Suzhou, China.   \\
%Fall 2019 & Instructor. (\href{https://yzhang1616.github.io/calculus19fall/calculus.html}{Integral Calculus}) \\
%             & The University of Texas at Dallas, Dallas, USA.   \\
%Spring 2019           & Instructor. (\href{https://yzhang1616.github.io/algebra19spring/algebra.html}{Linear Algebra}) \\
%                      & The University of Texas at Dallas, Dallas, USA. \\
%Fall 2010             & Teaching internship. (High School Mathematics) \\
%                      & Suzhou High School Affiliated to Xi'an Jiaotong University, Suzhou, China.                        
%\end{tabular}
%
%\section*{\Large{Professional Service Activities}}
%\begin{tabular}{@{}p{1.4in}p{4in}} 
%2020 -- 2021 & Global Engagement Officer. \\
%             & Xi'an Jiaotong - Liverpool University, Suzhou, China.                  
%\end{tabular}
%
%\section*{\Large{Professional Memberships}}
%\begin{tabular}{@{}p{1.4in}p{4in}} 
%2018 -- present &  Reviewer for Mathematical Reviews/MathSciNet.            
%\end{tabular}
%
%
%\section*{\Large{Awards}}
%\begin{longtable}{@{}p{1.4in}p{4in}}
%07/2021         & 2021 Jiangsu Province Innovation & Entrepreneurship Doctor-Talent Program (RMB 150,000), \\
%                       & Jiangsu, China. \\
%07/2016               & \href{https://www.sigsam.org/Awards/ISSACAwards.html}{ISSAC'16 Distinguished Student Author Award}, \\
%                      & SIGSAM, Association for Computing Machinery (ACM). \\
%09/2009 -- 07/2010    & The Second Prize Scholarship of Soochow University, \\ 
%                      & Suzhou, China.\\
%09/2008 -- 07/2009    & The First Prize Scholarship of Soochow University, \\ 
%                      & Suzhou, China.\\
%09/2007 -- 07/2008    & The First Prize Scholarship of Soochow University, \\ 
%                      & Suzhou, China.\\
%09/2007 -- 07/2008    & The Zhu Jingwen Scholarship of Soochow University, \\ 
%                      & Suzhou, China.\\
%09/2007 -- 07/2008    & The Merit Student of Soochow Universtiy, \\ 
%                      & Suzhou, China.\\
%\end{longtable}
%
%\section*{\Large{Grants}}
%\begin{longtable}{@{}p{1.4in}p{4in}}
%03/2021 -- 02/2024    & XJTLU Research Development Fund, Principle Investigator. No. RDF-20-01-12.\\ 
%                      & ``Algebraic and Numerical Analysis of Differential Equations based on Computer Algebra'', RMB 60,000.\\
%
%\end{longtable}
%
%\section*{\Large{PhD Thesis}}
%\begin{itemize}
% \item Yi Zhang. \href{https://yzhang1616.github.io/yzhang_PhDthesis_final.pdf}{{\em Univarite 
%                Contraction and Multivariate Desingularization of Ore Ideals}}. 
%                PhD thesis, Institute for Algebra, Johannes Kepler University Linz, 2017. 
%                arXiv:\href{https://arxiv.org/abs/1710.07445}{1710.07445}
%\end{itemize}
%
%\section*{\Large{Publications}}
%
%\subsection*{Published}
%\begin{enumerate}
%\item Sebastian Falkensteiner, Yi Zhang and Thieu N. Vo. 
%{\em On Existence and Uniqueness of Formal Power Series Solutions of Algebraic Ordinary
%     Differential Equations}, 2020, accepted by Mediterranean Journal of Mathematics. arXiv:\href{https://arxiv.org/abs/1803.09646}{1803.09646}. (SCI) 
%\item Carlos E.\ Arreche and Yi Zhang. 
%{\em Computing Differential Galois Groups of Second-order Linear q-Difference Equations}, 2020, accepted by Advances in Applied Mathematics. 
%arXiv:\href{https://arxiv.org/abs/2009.14026}{2009.14026}. (SCI)
% \item Nobuki Takayama, Lin Jiu, Satoshi Kuriki and Yi Zhang (corresponding author). 
% {\em Computation of the Expected Euler Characteristic for the Largest Eigenvalue of a Real Wishart Matrix}, Journal of Multivariate Analysis, 179, 104642, 2020. \\
% DOI:\href{https://doi.org/10.1016/j.jmva.2020.104642}{10.1016/j.jmva.2020.104642}, 
% arXiv:\href{http://arxiv.org/abs/1903.10099}{1903.10099}. (SCI)
% \item Thieu N. Vo and Yi Zhang (corresponding author). 
%{\em Rational Solutions of First-Order Algebraic Ordinary Difference Equations},  Advances in Applied Mathematics, 117, 102018, 2020. 
%DOI:\href{https://doi.org/10.1016/j.aam.2020.102018}{10.1016/j.aam.2020.102018},
%arXiv:\href{http://arxiv.org/abs/1901.11048}{1901.11048}. (SCI)
% \item Thieu N. Vo and Yi Zhang (corresponding author). {\em Rational Solutions of High-Order Algebraic Ordinary Differential Equations},  {\em Journal of Systems Science and Complexity}, 33, pp.\ 821-835, 2020. DOI:\href{https://link.springer.com/article/10.1007/s11424-019-8133-0}{10.1007/s11424-019-8133-0}, \\
% arXiv:\href{https://arxiv.org/abs/1709.04174}{1709.04174}. (SCI)
%  \item Shaoshi Chen, Manuel Kauers, Ziming Li and Yi Zhang (corresponding author). {\em Apparent Singularities of D-finite Systems}, 
% {\em  Journal of Symbolic Computation},  95, pp.\ 217-237, 2019. DOI:\href{https://doi.org/10.1016/j.jsc.2019.02.009}{10.1016/j.jsc.2019.02.009}, arXiv:\href{http://arxiv.org/abs/1705.00838}{1705.00838}. (SCI)
%\item Ting Guo, Christian Krattenthaler and Yi Zhang (corresponding author).
%{\em On (shape-)Wilf-equivalence for words}, 
%{\em  Advances in Applied Mathematics} , 100, pp.\ 87-100, 2018. 
%DOI:\href{https://doi.org/10.1016/j.aam.2018.05.006}{10.1016/j.aam.2018.05.006}, 
%arXiv:\href{https://arxiv.org/pdf/1802.09856.pdf}{1802.09856}. (SCI)
%\item Christoph Koutschan and Yi Zhang (corresponding author). {\em Desingularization in the $q$-Weyl Algebra}. 
%{\em Advances in Applied Mathematics}, 97, pp.\ 80–101, 2018. 
%DOI: \href{http://dx.doi.org/10.1016/j.aam.2018.02.005}{10.1016/j.aam.2018.02.005},
%arXiv:\href{https://arxiv.org/abs/1801.04160}{1801.04160}. (SCI) 
%\item Yi Zhang (corresponding author). {\em Contraction of Ore Ideals with Applications}. 
%In {\em Proceedings of the 2016 International Symposium on Symbolic and Algebraic Computation (ISSAC)}, 
%pp.\ 413-420, ACM Press, 2016. DOI:\href{http://dl.acm.org/citation.cfm?id=2930890}{10.1145/2930889.2930890},
%arXiv:\href{https://arxiv.org/abs/1511.07922}{1511.07922}. 
%\href{https://www.sigsam.org/Awards/ISSACAwards.html}{[Distinguished Student Author Award]} 
%(EI, ISSAC is a top international conference in the field ``Algorithms and Theory'', NUS evaluation: Rank 1, AUS evaluation: A+) 
%\end{enumerate}
%
%\subsection*{Submitted/in preparation}
%\begin{enumerate}
%\item Carlos E.\ Arreche and Yi Zhang. 
%{\em Mahler Residues for Rational Functions}, 2020, in preparation. 
%\item Zhimin Sun,  Xiangyong Zeng, Lin Yi and Yi Zhang. 
%{\em The Expansion Complexity of Ultimately Periodic Sequences over Finite Fields}, 2020, submitted.
%%\item Yi Zhang, Sebastian Falkensteiner and Thieu N. Vo. 
%%{\em On Formal Power Series Solutions of Algebraic Ordinary
%%     Differential Equations}, 2019, arXiv:\href{https://arxiv.org/abs/1803.09646}{1803.09646}, submitted. 
%\item Alin Bostan, Jordan Tirrell, Bruce W.\ Westbury and Yi Zhang. 
%{\em On Sequences Associated to the Invariant Theory of Rank Two Simple Lie Algebras}, 2019, \\
% arXiv:\href{https://arxiv.org/abs/1911.10288}{1911.10288}, submitted. 
%\end{enumerate}
%
%\section*{\Large{Research Notes}}
%\begin{itemize}
%\item Yi Zhang. {\em Testing q-shift Equivalence of Polynomials}, July, 2017.
%\item Yi Zhang. {\em Integer Vectors of a Fundamental Parallelepiped}, 2016.
% \item Ziming Li and Yi Zhang. {\em A Note on Gr\"{o}bner Bases of Ore Polynomials over a PID}, 2016. 
% \url{https://yzhang1616.github.io/GB.pdf} 
%\end{itemize}
%
%\section*{\Large{Software}}
%\begin{itemize}
%\item \href{https://yzhang1616.github.io/ct/Mihailovs_Conjecture.nb}{Mihailovs\_Conjecture.nb},  a Mathematica notebook for 
%proving Mihailovs' conjecure by the method of creative telescoping. It is based on joint work
%with Jordan Tirrell and Bruce W.\ Westbury.   The notebook requires the availability of Koutschan's package 
% \href{http://www.risc.jku.at/research/combinat/software/ergosum/RISC/HolonomicFunctions.html}{HolonomicFunctions.m}.
%\item \href{https://yzhang1616.github.io/complexity/ansatz.m}{ansatz.m}, 
%a Mathematica package for computing the expansion complexity of a given finite length sequences. 
%It is based on joint work with Zhimin Sun and Xiangyong Zeng. 
%  \item \href{https://yzhang1616.github.io/TestNonvanishing.nb}{TestNonvanishing.nb}, 
%    a Mathematica notebook for checking the nonvanishing property of algebraic ordinary
%    differential equations in Kamke's collection. It is based on joint work
%    with Sebastian Falkensteiner and N.\ Thieu Vo. 
%    The notebook requires the availability of the Mathematica package \href{https://yzhang1616.github.io/Kamke_ODE.m}{Kamke\_ODE.m}.
%  \item \href{https://yzhang1616.github.io/zof/zof.m}{zof.m}, a Mathematica package for generating $0$-$1$-fillings 
%  of a Ferrers board (shape), checking the number of
%    sigma-avoiding $0$-$1$-fillings of a Ferrers board, 
%     generating generalized $0$-$1$-fillings of a Ferrers board, 
%     and checking the number of generalized $0$-$1$-fillings of a Ferrers board with weight $n$
%    such that the longest ne-chain has length $u$ 
%    and the longest se-chain has length $v$. It is based on joint work with Ting
%    Guo and Christian Krattenthaler. For a demonstration of the package,
%    see the \href{https://yzhang1616.github.io/zof/zof.nb}{zof.nb} notebook. 
%  \item \href{https://yzhang1616.github.io/ec1/Example1_HGM.nb}{Example1\_HGM.nb}, a Mathematica notebook for
%    the demonstration of the holonomic gradient method for the evaluation of
%    expection of an Euler characteristic number. It is based on joint work
%    with Satoshi Kuriki and Nobuki Takayama. 
%    The notebook requires the availability of Koutschan's package 
%     \href{http://www.risc.jku.at/research/combinat/software/ergosum/RISC/HolonomicFunctions.html}{HolonomicFunctions.m}.
%  \item \href{https://yzhang1616.github.io/KamkeODEs.mw}{KamkeODEs.mw}, a Maple worksheet for 
%     checking the (completely) maximal comparability and noncriticality of algebraic
%     ordinary differential equations in
%     Kample's collection. It is based on joint work with Dr. Thieu Vo Ngoc. 
%     The worksheet requires the availability of the Maple package \href{https://yzhang1616.github.io/KamkeODEs.mpl}{KamkeODEs.mpl}.
% \item \href{https://yzhang1616.github.io/qDesingularization.m}{qDesingularization.m}, a Mathematica
%     package for computing desingularized operators and the $q$-Weyl closure of
%     a given $q$-difference operator in
%     the first $q$-Weyl algebra. It is based on joint work with Dr. Christoph
%     Koutschan. The package requires the availability of Koutschan's package
%     \href{http://www.risc.jku.at/research/combinat/software/ergosum/RISC/HolonomicFunctions.html}{HolonomicFunctions.m}
%     and Kauer's pacakge \href{https://www.risc.jku.at/research/combinat/risc/software/Singular/index.html}{Singular.m}.
%     For a description of the usage of the package, see the \href{https://yzhang1616.github.io/Example.nb}{Example.nb} notebook.
%\end{itemize}
%
%\section*{\Large{Talks}}
%\begin{enumerate}
%\item {\em On Sequences Associated to the Invariant Theory of Rank Two Lie Algebras}.
%Contributed talk at the 12th Annual Conference on Computer Mathematics (CM2021), Guilin, China, June, 2021.
%\item {\em On Sequences Associated to the Invariant Theory of Rank Two Lie Algebras}.
%Invited talk at Beihang University, Beijing, China, October, 2020.
%\item {\em On Sequences Associated to the Invariant Theory of Rank Two Lie Algebras}.
%Invited talk at Mahidol University International College, Bangkok, Thailand, January, 2020.
%\item {\em Apparent Singularities of D-finite Systems}.
%Invited talk at Kolchin Seminar, CUNY Graduate Center, New York, USA, December, 2019. 
%\item {\em Computations of the Expected Euler Characteristic for the Largest Eigenvalue of a Real Wishart Matrix}.
%Contributed talk at SIAM TX/LA Section (The 2nd Annual Meeting of the SIAM Texas Louisiana Section), Southern Methodist University, Dallas, USA, November, 2019.
%\item {\em Computations of the Expected Euler Characteristic for the Largest Eigenvalue of a Real Wishart Matrix}.
%Invited talk at Key Laboratory of Mathematics Mechanization, Academy of Mathematics and Systems Sciences,
% Chinese Academy of Sciences, Beijing, China, July, 2019. 
%\item {\em Computations of the Expected Euler Characteristic for the Largest Eigenvalue of a Real Wishart Matrix}.
%Invited talk at Johann Radon Institute for Computational and Applied Mathematics (RICAM), Austrian Academy of Sciences, 
%Austria, May, 2019.
% \item {\em Desingularization in the q-Weyl algebra}. 
% Invited talk at Key Laboratory of Mathematics Mechanization, Academy of Mathematics and Systems Sciences,
% Chinese Academy of Sciences, Beijing, China, July, 2018. 
% \item {\em Desingularization in the $q$-Weyl algebra}. 
% Contributed talk at at ACA'18 
% \ (the 24th Conference on Applications of Computer Algebra), the Faculty of Mathematics, 
% The University of Santiago de Compostela, Santiago, Spain, June, 2018.
% \item {\em Laurent Series Solutions of Algebraic Ordinary Differential Equations}. 
% Invited talk at Computer Algebra Seminar, Research Institute for Symbolic Computation (RISC), Johannes Kepler University Linz, 
% Austria, November, 2017.
% \item {\em Apparent Singularities of D-finite Systems}. Contributed talk at ACA'17 
% \ (the 23rd Conference on Applications of Computer Algebra), Jerusalem College of Technology, Jerusalem, Israel, July, 2017.
% \item {\em Contraction of Linear Difference and Differential Operators}. Contributed talk at ISSAC'16 
% \ (the 41st International Symposium on Symbolic and Algebraic Computation), Wilfrid Laurier University, Waterloo, Canada, July, 2016.
% \item {\em Contraction of Linear Difference and Differential Operators}.
%       Invited talk at the seminar of Center for Combinatorics, Nankai University, Tianjin, China, June, 2016.
% \item {\em An Algorithm for Contraction of an Ore Ideal}. Invited talk at the seminar of Institute of Discrete Mathematics and Geometry, 
%       Vienna University of Technology, Vienna, Austria, October, 2015.
% \item {\em The Restriction Problem for D-finite Functions}. 
%       Contributed talk at the Workshop on Computational and Algebraic Methods in Statistics,
%       The University of Tokyo, Tokyo, Japan, March, 2015.
% \item {\em An Algorithm for Decomposing Multivariate Hypergeometric Terms}. Contributed talk at CM'13
%       \ (the 5th National Conference of Computer Mathematics), Jilin University, Changchun, China, August, 2013.
%\end{enumerate}
%
%\section*{\Large Peer-Reviewing Activities}
%For each journal and conference the number of completed reviews in given in parentheses.
%\begin{itemize}
%\item Mathematical Reviews (1)
%\item Maple Conference (1)
%\item Mediterranean Journal of Mathematics (1)
%\item Conferences on Applications of Computer Algebra (1)
% \item Journal of Systems Science and Complexity (1)
% \item Journal of Computational and Applied Mathematics (1)
% \item Advances in Applied Mathematics (1)
% \item International Symposiums on Symbolic and Algebraic Computation (4)
% \item Journal of Symbolic Computation (3)
%\end{itemize}
%
%\section*{\Large{Further Skills}}
%\begin{itemize}
% \item Programming Skills: C, Matlab, Maple, Mathematica, Macaulay2, Sage and Python
% \item Spoken Language: Chinese (native), English (fluent), German (basic)
%\end{itemize}
%
%%\section*{\Large{Research Plan}}
%
%%\section*{\Large{Teaching Philosophy}}
%%
%%As Socrates said, “Education is not the filling of a vessel, but the kinding of a flame”. 
%%A good teacher not only teach students key ideas and techniques of a course, but also motivate their interests so that they will learn and 
%%study by themselves in the future. 
%%Meanwhile, a teacher shall also show the general methodology for learning a course so that they know how to learn it by themselves. 
%%There are general goals I want to achieve for my teaching. 
%%Under this philosophy, I teach my courses with the following schemes: 
%%
%%\begin{itemize}
%%\item Design lecture notes for audience. 
%%Before my lectures, I investigate mathematical backgrounds of my students by checking their previous course lists online. 
%%On the other hand, I also ask some of my senior colleagues about their experience in teaching the same course. 
%%This is very helpful in preparing my lecture notes because with those information 
%%I know how to organize the material so that it is not so hard for my students. 
%%When I design my lecture notes, I first recall some key results in the previous lecture so that students 
%%can have a  retrospect about what they learned before at the very beginning. 
%%In the final part of my notes, I give a summary about all the ingredients of the current lecture. 
%%After my lectures, I scan and submit my notes online so that students can download and read them for the convenience of their study.
%%
%%\item General guides in the first lecture. In my first lecture, I give an outline of the content of the course, such as: 
%%what are objects we are going to study;
%%some historical backgrounds of the topic; main problems and results in this area; possible applications in related areas. 
%%Besides, I also give some general suggestions for studying: 1.\ read the textbook and attend my lectures regularly; 
%%2.\ discuss with classmates and me (after lectures and during my office hours); 
%%3.\ use library properly; 4.\ finish homework independently as much as possible (discussion is encouraged, 
%%but plagiarism is definitely prohibited).  
%%
%%\item Balance between time and material. Each lecture has fixed time and is scheduled to teach one section 
%%(sometimes a half one) from the designated textbook. 
%%This requires the instructor or lecturer to prepare lecture notes with balance between time and material. 
%%For instance, sometimes one section contains five subsections (or even more). 
%%It would be very time-consuming to go through every detail of each subsection. In that case, 
%%I always try to 
%%summarize key ideas of each subsection and show students how to apply them to do concrete examples. 
%%%If the time is still very tight, I simply give some remarks in the lecture 
%%%and suggest students to learn that part by self-reading.  
%%On the other hand, if the time is quite abundant, 
%%I would give more concrete examples, details and some historical backgrounds of the topic during the class.  
%%%After my lectures, I will scan and submit my lecture notes online so that some students can download and read them for the convenience of their study.  
%%
%%\item Teaching with concrete examples. 
%%Mathematics is abstract because it contains a lot of definitions, lemmas, propositions and theorems. 
%%I find that it is an excellent idea to teach with concrete examples, especially for those students who are not from 
%%the mathematical department. Actually, classical examples not only illustrate abstract concepts and important ideas in mathematics, 
%%but also show students how to apply them to solve practical problems. 
%%Last but not least, examples help students understand, review and memorize 
%%key points of materials in lectures.  
%%
%%\item Communicating with students. Roughly speaking, teaching a course is all about communicating with students, 
%%exchanges ideas and thoughts with them. During my lectures, I always speak loudly, clearly and slowly so that my students 
%%can hear what I am talking about. Besides, I also write in the backboard with large enough handwriting and 
%%try to have suitable eye contact with students. Sometimes, they ask me some questions during the class. 
%%I always try to answer their questions concisely and clearly. On the other hand, their questions, comments and suggestions also help me 
%%improve the quality of my lectures and learn how to explain certain materials of the course in a more proper way. 
%%Moreover, I help them as possible as I can through office hours and emails.     
%%\end{itemize}
%%
%%\subsection*{\large{Teaching Experience}}
%%\begin{tabular}{@{}p{1.4in}p{4in}} 
%%Spring 2019           & Instructor. (\href{https://yzhang1616.github.io/algebra19spring/algebra.html}{Linear Algebra}) \\
%%                      & The University of Texas at Dallas, USA. \\
%%Fall 2010             & Teaching internship. (High School Mathematics) \\
%%                      & Suzhou High School Affiliated to Xi'an Jiaotong University, China.   
%%                  
%%\end{tabular}
%
%%\section*{\Large{Reference Providers}}
%%
%%\begin{itemize}
%%\item Prof. \href{http://www.kauers.de/}{Manuel Kauers}, \href{http://www.jku.at/algebra/content}{Institute for Algebra},  
%%         \href{http://www.jku.at/content}{Johannes Kepler University Linz}, Austria. Phone: +43 (0)732 2468 6850 (secretary), +43 (0)732 2468 9958 (direct). Email: \href{mailto:manuel@kauers.de}{manuel@kauers.de}
%%
%%\item Prof. \href{http://www.koutschan.de/}{Christoph Koutschan}, \href{https://www.ricam.oeaw.ac.at/}{Johann Radon Institute for Computational and Applied Mathematics} (RICAM),
%%                        \href{http://www.oeaw.ac.at/en/austrian-academy-of-sciences/}{Austrian Academy of Sciences}, Austria. Phone: +43 732 2468 5254. 
%%                        Email: \href{mailto:christoph@koutschan.de}{christoph@koutschan.de}
%%
%%\item Prof. \href{https://www.utdallas.edu/~arreche/}{Carlos E. Arreche}, \href{https://www.utdallas.edu/math/}{Department of Mathematical Sciences}, 
%%                        \href{https://www.utdallas.edu/}{The University of Texas at Dallas}, USA. Phone:  +1 (972) 883 6594. 
%%                        Email: \href{mailto:arreche@utdallas.edu}{arreche@utdallas.edu}
%%\end{itemize}
%%
%%Prof. Carlos E. Arreche can comment on my teaching at UT Dallas. 
%
%% C, Matlab, Maple, Mathematica, Macaulay2 and Sage
%% 
%% \section*{\Large{Hobbies and Interests}}
%% \begin{itemize}
%%  \item Sports: Table Tennis, Billiards, Tennis.
%%  \item Reading: Philosophy, History, Literature.
%%  \item Language: Chinese (native), English (fluent), German (basic).
%% \end{itemize}
%
%\end{document}
