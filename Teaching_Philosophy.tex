%\documentclass{article}[12pt]
%\usepackage{amssymb,amsmath}
\documentclass[10pt,a4paper]{article}
%\usepackage[T1]{fontenc}
%\usepackage[utf8]{inputenc}
\usepackage[utf8]{inputenc}
\usepackage{enumitem}
\usepackage{authblk}
%\usepackage{fullpage}
\usepackage[colorlinks = true,
            linkcolor = blue,
            urlcolor  = blue,
            citecolor = blue,
            anchorcolor = blue]{hyperref}

\usepackage{bbm}
\usepackage{amsmath,amsthm}
\usepackage{alltt, amssymb}
\usepackage[]{algorithm2e}
\usepackage{url}
\usepackage{graphicx}

\newcommand{\bigO}{{\mathcal{O}}}
\newcommand{\bigOsoft}{\tilde{\mathcal{O}}}
\newcommand{\bA}{ {\mathbb A}}
\newcommand{\bN}{ {\mathbb  N}}
\newcommand{\bZ} { {\mathbb Z}}
\newcommand{\bQ}{ {\mathbb  Q}}
\newcommand{\bC}{ {\mathbb  C}}
\newcommand{\bK}{ {\mathbb  K}}
\newcommand{\bD}{ {\mathbb D}}
\newcommand{\bE}{ {\mathbb E}}
\newcommand{\bR}{ {\mathbf R}}
\newcommand{\bx}{ {\mathbf x}}
\newcommand{\by}{ {\mathbf y}}
\newcommand{\bz}{ {\mathbf z}}
\newcommand{\bs}{ {\mathbf s}}
\newcommand{\bt}{ {\mathbf t}}
\newcommand{\bff}{ {\mathbf f}}
\newcommand{\bpa}{ {\boldsymbol \partial}}
\newcommand{\balpha}{ {\boldsymbol \alpha}}
\newcommand{\bbeta}{ {\boldsymbol \beta}}
\newcommand{\bdelta}{ {\boldsymbol \delta}}
\newcommand{\bgamma}{ {\boldsymbol \gamma}}
\newcommand{\bu}{ {\mathbf u}}
\newcommand{\bw}{ {\mathbf w}}
\newcommand{\cP}{ {\mathcal P}}
\newcommand{\cG}{ {\mathcal G}}
\newcommand{\bv}{ {\mathbf v}}
\newcommand{\bU}{ {\mathbf U}}
\newcommand{\bm}{ {\mathbf m}}
\newcommand{\bn}{ {\mathbf n}}
\newcommand{\bB}{ {\mathbf B}}
\newcommand{\ie}{{\it i.e.}}
\newcommand{\si} { {\sigma}}
\newcommand{\dd}{ {\rm d}}
\newcommand{\tor}{ {\rm tor}}
\newcommand{\den}{ {\rm den}}
\newcommand{\lc}{ \operatorname{lc}}
\newcommand{\lm}{ \operatorname{lm}}
\newcommand{\mm}{ \operatorname{M}}
\newcommand{\TT}{ \operatorname{T}}
\newcommand{\lt}{ \operatorname{lt}}
\newcommand{\HM}{{\operatorname{HM}}}
\newcommand{\HT}{{\operatorname{HT}}}
\newcommand{\PT}{{\operatorname{PT}}}
\newcommand{\PE}{{\operatorname{PE}}}
\newcommand{\HC}{{\operatorname{HC}}}
\newcommand{\lcm}{ \operatorname{lcm}}
\newcommand{\qlcm}{ \operatorname{qlcm}}
\newcommand{\spol}{ \operatorname{spol}}
\newcommand{\gpol}{ \operatorname{gpol}}
\newcommand{\rank}{ \operatorname{rank}}
\newcommand{\pa}{ {\partial}}
\newcommand{\lder}{ \operatorname{lder}}
\newcommand{\pder}{ \operatorname{pder}}
\newcommand{\ind}{ \operatorname{ind}}
\newcommand{\In}{ \operatorname{in}}
\newcommand{\sol}{ \operatorname{sol}}
\newcommand{\ann}{ \operatorname{ann}}
\newcommand{\dis}{ \operatorname{dis}}
\newcommand{\Drat}{ {\bK(\bx)[\bpa]}}
\newcommand{\Dpol}{ {\bK[\bx][\bpa]}}
\newcommand{\qpol}{ {\bK(q)[x][\pa]}}
\newcommand{\qrat}{ {\bK(q, x)[\pa]}}
\newcommand{\pf} {{\rm {\bf Proof.} } }
\newcommand{\cont}{\operatorname{Cont}}

\newtheorem{thm}{Theorem}[section]
\newtheorem{cor}[thm]{Corollary}
\newtheorem{lemma}[thm]{Lemma}
\newtheorem{prop}[thm]{Proposition}
\newtheorem{defn}[thm]{Definition}
\newtheorem{ex}[thm]{Example}
\newtheorem{algo}[thm]{Algorithm}
\newtheorem{remark}[thm]{Remark}
\newtheorem{conj}[thm]{Conjecture}

\renewcommand{\labelenumi}{(\arabic{enumi})}

\begin{document}

\title{\bf{\Huge{Teaching Philosophy}}}

\author{Yi Zhang}


\date{}
\maketitle

As Socrates said, “Education is not the filling of a vessel, but the kinding of a flame”. 
A good teacher not only teach students key ideas and techniques of a course, but also motivate their interests so that they will learn and 
study by themselves in the future. 
Meanwhile, a teacher shall also show the general methodology for learning a course so that they know how to learn it by themselves. 
There are general goals I want to achieve for my teaching. 
Under this philosophy, I teach my courses with the following schemes: 

\begin{itemize}
\item Design lecture notes for audience. 
Before my lectures, I investigate mathematical backgrounds of my students by checking their previous course lists online. 
On the other hand, I also ask some of my senior colleagues about their experience in teaching the same course. 
This is very helpful in preparing my lecture notes because with those information 
I know how to organize the material so that it is not so hard for my students. 
When I design my lecture notes, I first recall some key results in the previous lecture so that students 
can have a  retrospect about what they learned before at the very beginning. 
In the final part of my notes, I give a summary about all the ingredients of the current lecture. 
After my lectures, I scan and submit my notes online so that students can download and read them for the convenience of their study.

\item General guides in the first lecture. In my first lecture, I give an outline of the content of the course, such as: 
what are objects we are going to study;
some historical backgrounds of the topic; main problems and results in this area; possible applications in related areas. 
Besides, I also give some general suggestions for studying: 1.\ read the textbook and attend my lectures regularly; 
2.\ discuss with classmates and me (after lectures and during my office hours); 
3.\ use library properly; 4.\ finish homework independently as much as possible (discussion is encouraged, 
but plagiarism is definitely prohibited).  

\item Balance between time and material. Each lecture has fixed time and is scheduled to teach one section 
(sometimes a half one) from the designated textbook. 
This requires the instructor or lecturer to prepare lecture notes with balance between time and material. 
For instance, sometimes one section contains five subsections (or even more). 
It would be very time-consuming to go through every detail of each subsection. In that case, 
I always try to 
summarize key ideas of each subsection and show students how to apply them to do concrete examples. 
%If the time is still very tight, I simply give some remarks in the lecture 
%and suggest students to learn that part by self-reading.  
On the other hand, if the time is quite abundant, 
I would give more concrete examples, details and some historical backgrounds of the topic during the class.  
%After my lectures, I will scan and submit my lecture notes online so that some students can download and read them for the convenience of their study.  

\item Teaching with concrete examples. 
Mathematics is abstract because it contains a lot of definitions, lemmas, propositions and theorems. 
I find that it is an excellent idea to teach with concrete examples, especially for those students who are not from 
the mathematical department. Actually, classical examples not only illustrate abstract concepts and important ideas in mathematics, 
but also show students how to apply them to solve practical problems. 
Last but not least, examples help students understand, review and memorize 
key points of materials in lectures.  

\item Communicating with students. Roughly speaking, teaching a course is all about communicating with students, 
exchanges ideas and thoughts with them. During my lectures, I always speak loudly, clearly and slowly so that my students 
can hear what I am talking about. Besides, I also write in the backboard with large enough handwriting and 
try to have suitable eye contact with students. Sometimes, they ask me some questions during the class. 
I always try to answer their questions concisely and clearly. On the other hand, their questions, comments and suggestions also help me 
improve the quality of my lectures and learn how to explain certain materials of the course in a more proper way. 
Moreover, I help them as possible as I can through office hours and emails.     
\end{itemize}

\section*{\Large{Teaching Experience}}
\begin{tabular}{@{}p{1.4in}p{4in}} 
Spring 2019           & Instructor. (\href{https://yzhang1616.github.io/algebra19spring/algebra.html}{Linear Algebra}) \\
                      & The University of Texas at Dallas, USA. \\
Fall 2010             & Teaching internship. (High School Mathematics) \\
                      & Suzhou High School Affiliated to Xi'an Jiaotong \\
                      & University, China.   
                  
\end{tabular}





\end{document}
