\documentclass[a4paper,12pt]{article}
\usepackage{CJKutf8}
\usepackage{enumitem}
\usepackage[colorlinks = true,
            linkcolor = blue,
            urlcolor  = blue,
            citecolor = blue,
            anchorcolor = blue]{hyperref}
\usepackage{url}
\usepackage{graphicx,wrapfig,lipsum}
\usepackage{longtable}
\usepackage{fancyhdr}
 
\pagestyle{fancy}
\fancyhf{}
\rhead{Page \thepage}
\begin{CJK*}{UTF8}{gbsn}
\lhead{Yi Zhang}
\chead{Curriculum Vitae}
\end{CJK*}
\rhead{\thepage}

\title{\bf{\Huge{个人简历}}}
\author{}
\date{}

\begin{document}
\begin{CJK*}{UTF8}{gbsn}
\lhead{Yi Zhang}
\chead{Curriculum Vitae}


\maketitle
\thispagestyle{empty}

\begin{picture}(2,2)
 \put(300,-150){\includegraphics[width=3cm]{Yi_Zhang}}
\end{picture}

\section*{\Large{个人资料}}

\vspace{.05in}
\begin{tabular}{@{}p{1.2in}p{4in}}
姓名            & 张熠 \\
出生日期        & 1988年12月12日 \\
籍贯与户口       & 江苏省常州市 \\
国籍          & 中国 \\
婚姻状况       & 未婚 
\end{tabular}

\section*{\Large{联系方式}}

\vspace{.05in}
\begin{tabular}{@{}p{1.2in}p{4in}}
电子邮箱          & \href{mailto:zhangy@amss.ac.cn}{zhangy@amss.ac.cn}  \\
地址          & \href{https://www.ricam.oeaw.ac.at/}{Johann Radon Institute for Computational and Applied Mathematics (RICAM)} \\ 
                 & Austrian Academy of Sciences \\
                 & Altenbergerstra{\ss}e 69, A-4040 Linz, Austria \\
办公室           & S2 0435 (Science Park II) \\                
电话号码           & +43 732 2468 5235\\
个人网页         & \url{https://yzhang1616.github.io/}
\end{tabular}

\section*{\Large{教育经历}}

\vspace{.05in}
\begin{tabular}{@{}p{1.4in}p{4in}}
09/2013 -- 02/2017    & Ph.D. in Mathematics with distinction, 
                        Institute for Algebra, 
                        Johannes Kepler University Linz, Austria. 
                        (Co-supervisors: Prof. \href{http://www.kauers.de/}{Manuel Kauers} and 
                        Prof. \href{http://mmrc.iss.ac.cn/~zmli/}{Ziming Li})\\
09/2011 -- 07/2016    & Ph.D. in Applied Mathematics, 
                        Key Laboratory of Mathematics Mechanization, 
                        Academy of Mathematics and Systems Science, 
                        Chinese Academy of Sciences, Beijing, China. 
                        (Co-supervisors: Prof. Manuel Kauers and Prof. Ziming Li)\\
\end{tabular}
\begin{tabular}{@{}p{1.4in}p{4in}}
09/2007 -- 07/2011    & B.Sc. in Mathematics, School of Mathematical Sciences, 
                        Soochow University, Suzhou, China.  
\end{tabular} \\

\noindent I also studied as a Ph.D. student in Research Institute for Symbolic Computation, Johannes Kepler University Linz 
from 09/2013 to 06/2015 under the supervision of Prof. Manuel Kauers. 


\section*{\Large{工作经历}}

\vspace{.05in}
\begin{tabular}{@{}p{1.4in}p{4in}}
03/2017 -- 02/2018    & Postdoc researcher, 
                        Johann Radon Institute for Computational and Applied Mathematics (RICAM),
                        Austrian Acedemy of Sciences. 
                        (Supervisor: Dr. \href{http://www.koutschan.de/}{Christoph Koutschan})\\
\end{tabular}
\section*{\Large{获奖经历}}

\begin{tabular}{@{}p{1.4in}p{4in}}
07/2016               & ACM Distinguished Student Author Award at ISSAC'16, SIGSAM, Association for Computing Machinery. \\
09/2009 -- 07/2010    & The Second Prize Scholarship of Soochow University, Suzhou, China.\\
09/2008 -- 07/2009    & The First Prize Scholarship of Soochow University, Suzhou, China. \\
09/2007 -- 07/2009    & The First Prize Scholarship of Soochow University, Suzhou, China. \\ 
09/2007 -- 07/2008    & The Zhu Jingwen Scholarship of Soochow University, Suzhou, China. \\
09/2007 -- 07/2008    & The Merit Student of Soochow Universtiy, Suzhou, China. 
\end{tabular}

\section*{\Large{研究领域}}
{\bf Computer Algebra, Ore Algebras, Gr\"{o}bner Bases, Algorithmic Combinatorics and Experimental Mathematics}

\section*{\Large{学术论文}}
\begin{itemize}
 \item Yi Zhang. {\em Contraction of Ore Ideals with Applications}. 
       In {\em Proceedings of the 2016 International Symposium on Symbolic and Algebraic Computation}, 
       pp.\ 413-420, ACM Press, 2016. DOI:\href{http://dl.acm.org/citation.cfm?id=2930890}{10.1145/2930889.2930890.}
\end{itemize}

\section*{\Large{研究注解 (Research Notes)}}
\begin{itemize}
 \item Ziming Li and Yi Zhang. {\em A Note on Gr\"{o}bner Bases of Ore Polynomials over a PID}, 2016. 
 \url{https://yzhang1616.github.io/GB.pdf} 
 \item Yi Zhang. {\em Integer Vectors of a Fundamental Parallelepiped}, 2016.
\end{itemize}

\section*{\Large{待发表论文}}
\begin{itemize}
 \item Manuel Kauers, Ziming Li and Yi Zhang. {\em Apparent Singularities of D-finite Systems}, 2016. 
\end{itemize}

\section*{\Large{学术报告}}
\begin{itemize}
 \item[5.] {\em Contraction of Linear Difference and Differential Operators}. Contributed talk at ISSAC'16 
 (the 41st International Symposium on Symbolic and Algebraic Computation), Wilfrid Laurier University, Waterloo, Canada, July, 2016.
 \item[4.] {\em Contraction of Linear Difference and Differential Operators}.
       Invited talk at the seminar of Center for Combinatorics, Nankai University, Tianjin, China, June, 2016.
 \item[3.] {\em An Algorithm for Contraction of an Ore Ideal}. Invited talk at the seminar of Institute of Discrete Mathematics and Geometry, 
       Vienna University of Technology, Vienna, Austria, October, 2015.
 \item[2.] {\em The Restriction Problem for D-finite Functions}. 
       Contributed talk at the Workshop on Computational and Algebraic Methods in Statistics,
       The University of Tokyo, Tokyo, Japan, March, 2015.
 \item[1.] {\em An Algorithm for Decomposing Multivariate Hypergeometric Terms}. Contributed talk at CM'13
       (the 5th National Conference of Computer Mathematics), Jilin University, Changchun, China, August, 2013.
\end{itemize}

\section*{\Large 学术期刊评审工作}
For each journal and conference the number of completed reviews in given in parentheses.
\begin{itemize}
 \item International Symposiums on Symbolic and Algebraic Computation (1)
 \item Journal of Symbolic Computation (2)
\end{itemize}

\section*{\Large{编程语言}}
\vspace{.05in}
{\bf C, Matlab, Maple, Mathematica, Macaulay2 and Sage}

\section*{\Large{业余爱好及兴趣}}
\begin{itemize}
 \item 体育运动: 乒乓球, 桌球(台球), 网球
 \item 阅读: 文学, 历史, 哲学
 \item 语言: 中文 (母语), 英文 (流利), 德文 (基础)
\end{itemize}

\end{CJK*}
\end{document}
