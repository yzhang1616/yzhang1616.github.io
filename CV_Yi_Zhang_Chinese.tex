\documentclass[a4paper,12pt]{article}
\usepackage{CJKutf8}
\usepackage{enumitem}
\usepackage[colorlinks = true,
            linkcolor = blue,
            urlcolor  = blue,
            citecolor = blue,
            anchorcolor = blue]{hyperref}
\usepackage{url}
\usepackage{graphicx,wrapfig,lipsum}
\usepackage{longtable}
\usepackage{fancyhdr}
\usepackage[margin=1in]{geometry}
 
\pagestyle{fancy}
\fancyhf{}
\rhead{Page \thepage}
\lhead{Yi Zhang}
\chead{Curriculum Vitae}
\rhead{\thepage}
\newcommand{\red}{\color{red}}

\title{\bf{\Huge{个人简历}}}
\author{}
\date{}

\begin{document}
\begin{CJK*}{UTF8}{gbsn}
\lhead{Yi Zhang}
\chead{Curriculum Vitae}


\maketitle
\thispagestyle{empty}

\begin{picture}(2,2)
 \put(300,-150){\includegraphics[width=3cm]{Yi_Zhang}}
\end{picture}

\section*{\Large{基本信息}}
\begin{tabular}{@{}p{1.4in}p{4in}}
姓名            & 张熠 \\
性别            & 男 \\
国籍          & 中国 \\
出生年月       & 1988年12月12日 \\
现工作单位 \textbackslash 职称  & 得克萨斯大学达拉斯分校 \textbackslash 博士后\\
联系地址 & 800 West Campbell Road, \\
       &  Richardson, TX 75080-3021\\
%电话   & +43 677 6289 1475\\ 
邮箱 & \href{mailto:zhangy@amss.ac.cn}{zhangy@amss.ac.cn}
\end{tabular}

% \section*{\Large{联系方式}}
% \begin{tabular}{@{}p{1.2in}p{4in}}
% 电子邮箱          & \href{mailto:yi.zhang@ricam.oeaw.ac.at}{yi.zhang@ricam.oeaw.ac.at}  \\
% 地址          & \href{https://www.ricam.oeaw.ac.at/}{Johann Radon Institute for Computational and Applied Mathematics (RICAM)} \\ 
%                  & Austrian Academy of Sciences \\
%                  & Altenbergerstra{\ss}e 69, A-4040 Linz, Austria \\
% 办公室           & S2 0435 (Science Park II) \\                
% 电话号码           & +43 732 2468 5235\\
% 个人网页         & \url{https://www.ricam.oeaw.ac.at/people/member/?firstname=Yi&lastname=Zhang}
% \end{tabular}

% \section*{\Large{研究领域}}
% 计算机代数, 计算代数几何, 以及 算法组合学  

\section*{\Large{教育背景(自本科起)}}
\begin{tabular}{@{}p{1.4in}p{4in}}
09/2013 -- 02/2017    & 理学博士 \ (优秀毕业生), 数学,  
                        \href{http://www.jku.at/algebra/content}{代数研究所}, 
                        \href{http://www.jku.at/content}{约翰·开普勒林茨大学}, 林茨, 奥地利 \\                        
                      & 导师: \href{http://www.kauers.de/}{Manuel Kauers} 教授 与  
                         \ \href{http://mmrc.iss.ac.cn/~zmli/}{李子明} 研究员\\
09/2011 -- 07/2016    & 理学博士,  应用数学, 
                        \href{http://english.mmrc.amss.cas.cn/}{数学机械化重点实验室}, 
                        \href{http://www.amss.ac.cn/}{中国科学院数学与系统科学研究院}, 
                        \href{http://www.gucas.ac.cn/}{中国科学院大学}, 北京, 中国 
                        \ 导师: \href{http://www.kauers.de/}{Manuel Kauers} 教授 与  
                         \ \href{http://mmrc.iss.ac.cn/~zmli/}{李子明} 研究员\\
09/2007 -- 07/2011    & 理学学士, 数学与应用数学 \ (师范), \href{http://math.suda.edu.cn/}{数学科学学院}, 
                        \href{http://www.suda.edu.cn/}{苏州大学}, 苏州, 中国
\end{tabular} \\

% \noindent 我于 09/2013 到 06/2015 在 Manuel Kauers 教授的指导下在\href{http://www.risc.jku.at/}{奥地利符号计算研究中心}, 
% 约翰开普勒林茨大学进行博士阶段的学习与研究。 

\section*{\Large{科研 \slash 工作背景}}
\begin{tabular}{@{}p{1.4in}p{4in}}
09/2018 -- 至今        & 博士后, \href{https://www.utdallas.edu/math/}{数学系}, 
                        \href{https://www.utdallas.edu/}{得克萨斯大学达拉斯分校 \ (UTD)}, 美国 \\
                       & 合作导师: \href{https://www.utdallas.edu/~arreche/}{Carlos E. Arreche} 教授 \\ 
03/2017 -- 08/2018        & 博士后, 
                        \href{https://www.ricam.oeaw.ac.at/}{Johann Radon Institute for Computational and Applied Mathematics} (RICAM),
                        \href{http://www.oeaw.ac.at/en/austrian-academy-of-sciences/}{奥地利科学院}, 奥地利 \\                       
                       & 合作导师: \href{http://www.koutschan.de/}{Christoph Koutschan} 教授\\
\end{tabular}

\section*{\Large{主要研究背景}}
符号计算, 计算机代数, 计算代数几何, 算法组合学, 微分及差分方程的代数理论

\section*{\Large{代表性论文(*通讯作者)}}
\begin{enumerate}
\item Jordan Tirrell, Bruce W.\ Westbury and Yi Zhang. 
{\em Set partitions and $G_2$ webs}, 2019, in preparation.
\item Thieu N. Vo and Yi Zhang. 
{\em Rational Solutions of High-Order Algebraic Ordinary Difference Equations}, 2019, in preparation.
\item Carlos Arreche and Yi Zhang.
{\em Computation of the unipotent radical of the differential Galois group for a parameterized linear differential equation}, 2019, in preparation. 
\item Zhimin Sun,  Xiangyong Zeng and Yi Zhang. 
{\em The relation between the maximum order
omplexity and expansion complexity of finite length sequences}, 2019, in preparation.
\item Maximilian Jaroschek and Yi Zhang. 
{\em Desingularization for Linear Mahler Operators}, 2019, in preparation.
\item Nobuki Takayama, Lin Jiu, Satoshi Kuriki and Yi Zhang. 
 {\em Computations of the Expected Euler Characteristic for the Largest Eigenvalue of a Real Wishart Matrix}, 2019, 
 arXiv:\href{http://arxiv.org/abs/1903.10099}{1903.10099}, submitted.
% \item Nobuki Takayama, Lin Jiu, Satoshi Kuriki and Yi Zhang. 
%  {\em Euler Characteristic Method for the Largest Eigenvalue of a Random Matrix}, 2019, in preparation.
  \item Thieu N. Vo and Yi Zhang. 
{\em Rational Solutions of First-Order Algebraic Ordinary Difference Equations}, 2019,  
arXiv:\href{http://arxiv.org/abs/1901.11048}{1901.11048}, submitted.
  \item N. Thieu Vo and Yi Zhang (*通讯作者). {\em Rational Solutions of High-Order Algebraic Ordinary Differential Equations}, 2018, 
  arXiv:\href{https://arxiv.org/abs/1709.04174}{1709.04174}, accepted by Journal of Systems Science and Complexity (SCI).
   \item Shaoshi Chen, Manuel Kauers, Ziming Li and Yi Zhang  (*通讯作者). {\em Apparent Singularities of D-finite Systems}, 2019. 
 {\em  Journal of Symbolic Computation}, 95, pp.\ 217-237, 2019. arXiv:\href{http://arxiv.org/abs/1705.00838}{1705.00838},  
 DOI:\href{https://doi.org/10.1016/j.jsc.2019.02.009}{10.1016/j.jsc.2019.02.009}. (SCI)
\item Ting Guo, Christian Krattenthaler and Yi Zhang (*通讯作者).
{\em On (shape-)Wilf-equivalence for words}, 2018.
{\em  Advances in Applied Mathematics}, 100, pp.\ 87-100, 2018. 
DOI:\href{https://doi.org/10.1016/j.aam.2018.05.006}{10.1016/j.aam.2018.05.006}, 
arXiv:\href{https://arxiv.org/pdf/1802.09856.pdf}{1802.09856}. (SCI)
\item Christoph Koutschan and Yi Zhang (*通讯作者). {\em Desingularization in the $q$-Weyl Algebra}. 
{\em Advances in Applied Mathematics}, 97, pp.\ 80–101, 2018. \\
DOI:\href{http://dx.doi.org/10.1016/j.aam.2018.02.005}{10.1016/j.aam.2018.02.005},
arXiv:\href{https://arxiv.org/abs/1801.04160}{1801.04160}. (SCI) 
%\item Shaoshi Chen, Manuel Kauers, Ziming Li and Yi Zhang (*通讯作者). {\em Apparent Singularities of D-finite Systems}, 2017. 
% arXiv:\href{http://arxiv.org/abs/1705.00838}{1705.00838}, accepted by Journal of Symbolic Computation (SCI).
\item Yi Zhang (*通讯作者). {\em Contraction of Ore Ideals with Applications}. 
In {\em Proceedings of the 2016 International Symposium on Symbolic and Algebraic Computation}, 
pp.\ 413-420, ACM Press, 2016. DOI:\href{http://dl.acm.org/citation.cfm?id=2930890}{10.1145/2930889.2930890.} 
{\red (ISSAC为计算机科学 
“Algorithms and Theory”
领域的国际顶级会议,NUS评价为Rank 1,AUS评价为A+) (EI, 该文获得ISSAC2016杰出学生论文奖)} 
\end{enumerate}

\section*{\Large{获奖情况}}
\begin{tabular}{@{}p{1.4in}p{4in}}
07/2016               & \href{https://www.sigsam.org/Awards/ISSACAwards.html}{ACM SIGSAM颁发的ISSAC2016杰出学生论文奖} \\
09/2009 -- 07/2010    & 苏州大学人民综合二等奖学金\\
09/2008 -- 07/2009    & 苏州大学人民综合一等奖学金 \\
09/2007 -- 07/2008    & 苏州大学人民综合一等奖学金 \\ 
09/2007 -- 07/2008    & 苏州大学朱敬文奖学金 \\
09/2007 -- 07/2008    & 苏州大学校三好学生
\end{tabular}

\section*{学术任职}
\begin{itemize}
 \item 《Mathematical Reviews》评论员
\end{itemize}

\section*{教学背景}
\begin{tabular}{@{}p{1.4in}p{4in}}
2019春季           & 教师 (Instructor), \href{https://yzhang1616.github.io/algebra19spring/algebra.html}{线性代数} \\
                      & 得克萨斯大学达拉斯分校
\end{tabular}


\section*{\Large{访问背景}}
\begin{tabular}{@{}p{1.0in}p{4.5in}}
05/2017               & 访问学者, 
                        \href{http://www.math.kobe-u.ac.jp/}{数学系},
                        \href{http://www.kobe-u.ac.jp/en/}{神户大学 \ (Kobe University)}, 日本. \\                       
                        & 合作导师: \href{http://www.math.kobe-u.ac.jp/home-j/takayama-e.html}{Nobuki Takayama \ (高山信毅)} 教授\\
\end{tabular}

% \section*{学术任职}
% \begin{itemize}
%  \item 《Mathematical Reviews》评论员
% \end{itemize}

% \section*{\Large{获奖情况}}
% \begin{tabular}{@{}p{1.4in}p{4in}}
% 07/2016               & \href{https://www.sigsam.org/Awards/ISSACAwards.html}{ACM SIGSAM颁发的ISSAC2016杰出学生论文奖} \\
% 09/2009 -- 07/2010    & 苏州大学人民综合二等奖学金\\
% 09/2008 -- 07/2009    & 苏州大学人民综合一等奖学金 \\
% 09/2007 -- 07/2008    & 苏州大学人民综合一等奖学金 \\ 
% 09/2007 -- 07/2008    & 苏州大学朱敬文奖学金 \\
% 09/2007 -- 07/2008    & 苏州大学校三好学生
% \end{tabular}

% \section*{\Large{研究领域}}
% {\bf 计算机代数 (Computer Algebra), 计算代数几何 (Computational Algebraic Geometry), Ore代数 (Ore Algebras), 
% Gr\"{o}bner基 (Gr\"{o}bner Bases), 算法组合学 (Algorithmic Combinatorics) 以及 \ 实验数学 \\ (Experimental Mathematics)}

\section*{\Large{博士论文}}
\begin{itemize}
 \item Yi Zhang. \href{https://yzhang1616.github.io/yzhang_PhDthesis_final.pdf}{{\em Univarite 
                Contraction and Multivariate Desingularization of Ore Ideals}}. 
                PhD thesis, Institute for Algebra, Johannes Kepler University Linz, 2017. 
                arXiv:\href{https://arxiv.org/abs/1710.07445}{1710.07445}
\end{itemize}

% \section*{\Large{代表性论文(*通讯作者)}}
% \begin{enumerate}
% \item Maximilian Jaroschek and Yi Zhang. 
% {\em Desingularization in General Shift Algebras}, 2018, in preparation.
% \item Georg Grasegger, N. Thieu Vo and Yi Zhang. 
% {\em Rational Solutions of Algebraic Difference Equations}, 2018, in preparation.
% \item Lin Jiu, Satoshi Kuriki, Nobuki Takayama and Yi Zhang. 
%  {\em Euler Characteristic Method for the Largest Eigenvalue of a Random Matrix}, 2018, in preparation.
%   \item N. Thieu Vo and Yi Zhang. {\em Rational Solutions of High-Order Algebraic Ordinary Differential Equations}, 2018, 
%   arXiv:\href{https://arxiv.org/abs/1709.04174}{1709.04174}, submitted.
% \item Ting Guo, Christian Krattenthaler and Yi Zhang (*通讯作者).
% {\em On (shape-)Wilf-equivalence for words}, 2018.
% {\em  Advances in Applied Mathematics}, 100, pp.\ 87-100, 2018. 
% DOI:\href{https://doi.org/10.1016/j.aam.2018.05.006}{10.1016/j.aam.2018.05.006}, 
% arXiv:\href{https://arxiv.org/pdf/1802.09856.pdf}{1802.09856}. (SCI)
% \item Christoph Koutschan and Yi Zhang (*通讯作者). {\em Desingularization in the $q$-Weyl Algebra}. 
% {\em Advances in Applied Mathematics}, 97, pp.\ 80–101, 2018. \\
% DOI:\href{http://dx.doi.org/10.1016/j.aam.2018.02.005}{10.1016/j.aam.2018.02.005},
% arXiv:\href{https://arxiv.org/abs/1801.04160}{1801.04160}. (SCI) 
% \item Shaoshi Chen, Manuel Kauers, Ziming Li and Yi Zhang (*通讯作者). {\em Apparent Singularities of D-finite Systems}, 2017. 
%  arXiv:\href{http://arxiv.org/abs/1705.00838}{1705.00838}, submitted.
% \item Yi Zhang (*通讯作者). {\em Contraction of Ore Ideals with Applications}. 
% In {\em Proceedings of the 2016 International Symposium on Symbolic and Algebraic Computation}, 
% pp.\ 413-420, ACM Press, 2016. DOI:\href{http://dl.acm.org/citation.cfm?id=2930890}{10.1145/2930889.2930890.} 
% {\red (ISSAC为计算机科学 \\
% “Algorithms and Theory”
% 领域的国际顶级会议,NUS评价为Rank 1,AUS评价为A+) (EI, 该文获得ISSAC2016杰出学生论文奖)} 
% \end{enumerate}


% \section*{\Large{已发表论文}}
% \begin{itemize}
%  \item \textbf{Yi Zhang}. {\em Contraction of Ore Ideals with Applications}. 
%        In {\em Proceedings of the 2016 International Symposium on Symbolic and Algebraic Computation}, 
%        pp.\ 413-420, ACM Press, 2016. DOI:\href{http://dl.acm.org/citation.cfm?id=2930890}{10.1145/2930889.2930890.}
%        {\red (ISSAC为计算机科学“Algorithms and Theory”领域的国际顶级会议,NUS评价为Rank 1,AUS评价为A+) (EI, 该文获得ACM最佳学生论文, ISSAC 2016)}
% \end{itemize}
% 
% \section*{\Large{待发表论文}}
% \begin{itemize}
%  \item Manuel Kauers, Ziming Li and \textbf{Yi Zhang}. {\em Apparent Singularities of D-finite Systems}, 2017. 
%  arXiv:\href{http://arxiv.org/abs/1705.00838}{705.00838}, submitted to Journal of Symbolic Computation (SCI).
%  \item \textbf{Yi Zhang}. {\em Desingularization in the $q$-Weyl Algebra}, 2017.
%  \item Thieu Vo Ngoc and \textbf{Yi Zhang}. {\em Laurent Series Solutions of Algebraic Ordinary Differential Equations}, 2017. 
% \end{itemize}

\section*{\Large{研究注记 \ (Research Notes)}}
\begin{itemize}
  \item N. Thieu Vo, Sebastian Falkensteiner and Yi Zhang.
 {\em Formal Power Series Solutions of Algebraic Ordinary Differential Equations}, 
 2018, arXiv:\href{https://arxiv.org/abs/1803.09646}{1803.09646}.
 \item Yi Zhang. {\em Testing q-shift Equivalence of Polynomials}, July, 2017.
 \item Yi Zhang. {\em Integer Vectors of a Fundamental Parallelepiped}, 2016.
 \item Ziming Li and Yi Zhang. {\em A Note on Gr\"{o}bner Bases of Ore Polynomials over a PID}, 2016. 
 \url{https://yzhang1616.github.io/GB.pdf} 
%  \item Yi Zhang. {\em Integer Vectors of a Fundamental Parallelepiped}, 2016.
%  \item Yi Zhang. {\em Testing q-shift Equivalence of Polynomials}, July, 2017.
\end{itemize}

% \section*{\Large{其它出版物}}

\section*{\Large{软件包}}
\begin{itemize}
\item \href{https://yzhang1616.github.io/ct/Mihailovs_Conjecture.nb}{Mihailovs\_Conjecture.nb},  a Mathematica notebook for 
proving Mihailovs' conjecure by the method of creative telescoping. It is based on joint work
with Jordan Tirrell and Bruce W.\ Westbury.   The notebook requires the availability of Koutschan's package 
 \href{http://www.risc.jku.at/research/combinat/software/ergosum/RISC/HolonomicFunctions.html}{HolonomicFunctions.m}.
\item \href{https://yzhang1616.github.io/complexity/ansatz.m}{ansatz.m}, 
a Mathematica package for computing the expansion complexity of a given finite length sequences. 
It is based on joint work with Zhimin Sun and Xiangyong Zeng.
  \item \href{https://yzhang1616.github.io/TestNonvanishing.nb}{TestNonvanishing.nb}, 
    a Mathematica notebook for checking the nonvanishing property of algebraic ordinary
    differential equations in Kamke's collection. It is based on joint work
    with Sebastian Falkensteiner and N.\ Thieu Vo. 
    The notebook requires the availability of the Mathematica package \href{https://yzhang1616.github.io/Kamke_ODE.m}{Kamke\_ODE.m}.
  \item \href{https://yzhang1616.github.io/zof/zof.m}{zof.m}, a Mathematica package for generating $0$-$1$-fillings 
  of a Ferrers board (shape), checking the number of
    sigma-avoiding $0$-$1$-fillings of a Ferrers board, 
     generating generalized $0$-$1$-fillings of a Ferrers board, 
     and checking the number of generalized $0$-$1$-fillings of a Ferrers board with weight $n$
    such that the longest ne-chain has length $u$ 
    and the longest se-chain has length $v$. It is based on joint work with Ting
    Guo and Christian Krattenthaler. For a demonstration of the package,
    see the \href{https://yzhang1616.github.io/zof/zof.nb}{zof.nb} notebook. 
\item \href{https://yzhang1616.github.io/ec1/Example1_HGM.nb}{Example1\_HGM.nb}, a Mathematica notebook for
    the demonstration of the holonomic gradient method for the evaluation of
    expection of an Euler characteristic number. It is based on joint work
    with Satoshi Kuriki and Nobuki Takayama. 
    The notebook requires the availability of Koutschan's package 
     \href{http://www.risc.jku.at/research/combinat/software/ergosum/RISC/HolonomicFunctions.html}{HolonomicFunctions.m}.
 \item \href{https://yzhang1616.github.io/KamkeODEs.mw}{KamkeODEs.mw}, a Maple worksheet for 
     checking the (completely) maximal comparability and noncriticality of algebraic
     ordinary differential equations in
     Kample's collection. It is based on joint work with Dr. Thieu Vo Ngoc. 
     The worksheet requires the availability of the Maple package \href{https://yzhang1616.github.io/KamkeODEs.mpl}{KamkeODEs.mpl}.
 \item \href{https://yzhang1616.github.io/qDesingularization.m}{qDesingularization.m}, a Mathematica
     package for computing desingularized operators and the $q$-Weyl closure of
     a given $q$-difference operator in
     the first $q$-Weyl algebra. It is based on joint work with Dr. Christoph
     Koutschan. The package requires the availability of Koutschan's package
     \href{http://www.risc.jku.at/research/combinat/software/ergosum/RISC/HolonomicFunctions.html}{HolonomicFunctions.m}
     and Kauer's pacakge \href{https://www.risc.jku.at/research/combinat/risc/software/Singular/index.html}{Singular.m}.
     For a description of the usage of the package, see the \href{https://yzhang1616.github.io/Example.nb}{Example.nb} notebook.
\end{itemize}

\section*{\Large{学术报告}}
\begin{enumerate}
\item {\em Computations of the Expected Euler Characteristic for the Largest Eigenvalue of a Real Wishart Matrix}.
Invited talk at Johann Radon Institute for Computational and Applied Mathematics (RICAM), Austrian Academy of Sciences, 
Austria, May, 2019.
 \item {\em Desingularization in the q-Weyl algebra}. 
 Invited talk at Key Laboratory of Mathematics Mechanization, Academy of Mathematics and Systems Sciences,
 Chinese Academy of Sciences, Beijing, China, July, 2018. 
\item {\em Desingularization in the $q$-Weyl algebra}. 
 Contributed talk at at ACA'18 
 \ (the 24th Conference on Applications of Computer Algebra), the Faculty of Mathematics, 
 The University of Santiago de Compostela, Santiago, Spain, June, 2018.
 \item {\em Laurent Series Solutions of Algebraic Ordinary Differential Equations}. 
 Invited talk at Computer Algebra Seminar, Research Institute for Symbolic Computation (RISC), Johannes Kepler University Linz, 
 Austria, November, 2017.
 \item {\em Apparent Singularities of D-finite Systems}. Contributed talk at ACA'17 
 \ (the 23rd Conference on Applications of Computer Algebra), Jerusalem College of Technology, Jerusalem, Israel, July, 2017.
 \item {\em Contraction of Linear Difference and Differential Operators}. Contributed talk at ISSAC'16 
 \ (the 41st International Symposium on Symbolic and Algebraic Computation), Wilfrid Laurier University, Waterloo, Canada, July, 2016.
 \item {\em Contraction of Linear Difference and Differential Operators}.
       Invited talk at the seminar of Center for Combinatorics, Nankai University, Tianjin, China, June, 2016.
 \item {\em An Algorithm for Contraction of an Ore Ideal}. Invited talk at the seminar of Institute of Discrete Mathematics and Geometry, 
       Vienna University of Technology, Vienna, Austria, October, 2015.
 \item {\em The Restriction Problem for D-finite Functions}. 
       Contributed talk at the Workshop on Computational and Algebraic Methods in Statistics,
       The University of Tokyo, Tokyo, Japan, March, 2015.
 \item {\em An Algorithm for Decomposing Multivariate Hypergeometric Terms}. Contributed talk at CM'13
       \ (the 5th National Conference of Computer Mathematics), Jilin University, Changchun, China, August, 2013.
\end{enumerate}

\section*{\Large 学术期刊评审工作}
对于以下的学术期刊及会议论文,括号内给出了相应的评审次数。
\begin{itemize}
\item Conferences on Applications of Computer Algebra (1)
\item Journal of Systems Science and Complexity (1)
 \item Journal of Computational and Applied Mathematics (1)
 \item Advances in Applied Mathematics (1)
 \item International Symposiums on Symbolic and Algebraic Computation (2)
 \item Journal of Symbolic Computation (3)
\end{itemize}

\section*{\Large{其他技能}}
\begin{itemize}
 \item 编程技能: C, Matlab, Maple, Mathematica, Macaulay2, Sage 以及 \ Python
 \item 口语: 中文 \ (母语), 英文 \ (流利), 德文 \ (基础)
\end{itemize}

% \section*{\Large{业余爱好及兴趣}}
% \begin{itemize}
%  \item 体育运动: 乒乓球, 桌球(台球), 网球
%  \item 阅读: 文学, 历史, 哲学
%  \item 语言: 中文 (母语), 英文 (流利), 德文 (基础)
% \end{itemize}

\end{CJK*}
\end{document}
