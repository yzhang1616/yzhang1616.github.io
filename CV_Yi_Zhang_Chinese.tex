\documentclass[a4paper,12pt]{article}
\usepackage{CJKutf8}
\usepackage{enumitem}
\usepackage[colorlinks = true,
            linkcolor = blue,
            urlcolor  = blue,
            citecolor = blue,
            anchorcolor = blue]{hyperref}
\usepackage{url}
\usepackage{graphicx,wrapfig,lipsum}
\usepackage{longtable}
\usepackage{fancyhdr}
 
\pagestyle{fancy}
\fancyhf{}
\rhead{Page \thepage}
\lhead{Yi Zhang}
\chead{Curriculum Vitae}
\rhead{\thepage}

\title{\bf{\Huge{个人简历}}}
\author{}
\date{}

\begin{document}
\begin{CJK*}{UTF8}{gbsn}
\lhead{Yi Zhang}
\chead{Curriculum Vitae}


\maketitle
\thispagestyle{empty}

\begin{picture}(2,2)
 \put(300,-150){\includegraphics[width=3cm]{Yi_Zhang}}
\end{picture}

\section*{\Large{个人资料}}

\vspace{.05in}
\begin{tabular}{@{}p{1.2in}p{4in}}
姓名            & 张熠 \\
出生日期        & 1988年12月12日 \\
籍贯与户口       & 江苏省常州市 \\
国籍          & 中国 \\
婚姻状况       & 未婚 
\end{tabular}

\section*{\Large{联系方式}}

\vspace{.05in}
\begin{tabular}{@{}p{1.2in}p{4in}}
电子邮箱          & \href{mailto:zhangy@amss.ac.cn}{zhangy@amss.ac.cn}  \\
地址          & \href{https://www.ricam.oeaw.ac.at/}{Johann Radon Institute for Computational and Applied Mathematics (RICAM)} \\ 
                 & Austrian Academy of Sciences \\
                 & Altenbergerstra{\ss}e 69, A-4040 Linz, Austria \\
办公室           & S2 0435 (Science Park II) \\                
电话号码           & +43 732 2468 5235\\
个人网页         & \url{https://yzhang1616.github.io/}
\end{tabular}

\section*{\Large{教育经历}}

\vspace{.05in}
\begin{tabular}{@{}p{1.4in}p{4in}}
09/2013 -- 02/2017    & 理学博士 (优秀毕业生), 数学,  
                        \href{http://www.jku.at/algebra/content}{代数研究所}, 
                        \href{http://www.jku.at/content}{约翰开普勒林茨大学}, 奥地利 
                        (导师 (Supervisor): \href{http://www.kauers.de/}{Manuel Kauers} 教授 与  
                         \ \href{http://mmrc.iss.ac.cn/~zmli/}{李子明} 研究员)\\
09/2011 -- 07/2016    & 理学博士,  应用数学, 
                        \href{http://english.mmrc.amss.cas.cn/}{数学机械化重点实验室}, 
                        \href{http://www.amss.ac.cn/}{中国科学院数学与系统科学研究院}, 
                        \href{http://www.gucas.ac.cn/}{中国科学院大学}, 北京, 中国 
                        \ (导师 (Supervisor): \href{http://www.kauers.de/}{Manuel Kauers} 教授 与  
                         \ \href{http://mmrc.iss.ac.cn/~zmli/}{李子明} 研究员)\\
\end{tabular}
\begin{tabular}{@{}p{1.4in}p{4in}}
09/2007 -- 07/2011    & 理学学士, 数学与应用数学 (师范), \href{http://math.suda.edu.cn/}{数学科学学院}, 
                        \href{http://www.suda.edu.cn/}{苏州大学}, 苏州, 中国
\end{tabular} \\

\noindent 我于 09/2013 到 06/2015 在 Manuel Kauers 教授的指导下在\href{http://www.risc.jku.at/}{奥地利符号计算研究中心}, 
约翰开普勒林茨大学进行博士阶段的学习与研究。 

\section*{\Large{工作经历}}

\vspace{.05in}
\begin{tabular}{@{}p{1.4in}p{4in}}
03/2017 -- 02/2018    & 博士后 (Postdoc researcher), 
                        \href{https://www.ricam.oeaw.ac.at/}{Johann Radon Institute for Computational and Applied Mathematics} (RICAM),
                        \href{http://www.oeaw.ac.at/en/austrian-academy-of-sciences/}{奥地利科学院}. 
                        (合作导师 (Supervisor): \href{http://www.koutschan.de/}{Christoph Koutschan} 博士)\\
\end{tabular}

\section*{\Large{访问经历}}

\vspace{.05in}
\begin{tabular}{@{}p{1.0in}p{4.5in}}
2017年五月               & 访问学者 (Visiting scholar), 
                        \href{http://www.math.kobe-u.ac.jp/}{数学系},
                        \href{http://www.kobe-u.ac.jp/en/}{神户大学 (Kobe University)}, 日本. 
                        (合作导师 (Host researcher): \href{http://www.math.kobe-u.ac.jp/home-j/takayama-e.html}{Nobuki Takayama (高山信毅)} 教授)\\
\end{tabular}

\section*{\Large{获奖情况}}

\begin{tabular}{@{}p{1.4in}p{4in}}
07/2016               & \href{https://www.sigsam.org/Awards/ISSACAwards.html}{ACM最佳学生论文}, 
                        ISSAC'16, SIGSAM, Association for Computing Machinery \\
09/2009 -- 07/2010    & 苏州大学人民综合二等奖学金\\
09/2008 -- 07/2009    & 苏州大学人民综合一等奖学金 \\
09/2007 -- 07/2008    & 苏州大学人民综合一等奖学金 \\ 
09/2007 -- 07/2008    & 苏州大学朱敬文奖学金 \\
09/2007 -- 07/2008    & 苏州大学校三好学生
\end{tabular}

\section*{\Large{研究领域}}
{\bf 计算机代数 (Computer Algebra), 计算代数几何 (Computational Algebraic Geometry), Ore代数 (Ore Algebras), 
Gr\"{o}bner基 (Gr\"{o}bner Bases), 算法组合学 (Algorithmic Combinatorics) 以及 \ 实验数学 \\ (Experimental Mathematics)}

\section*{\Large{博士论文}}
\begin{itemize}
 \item Yi Zhang. \href{https://yzhang1616.github.io/yzhang_PhDthesis_final.pdf}{{\em Univarite Contraction and Multivariate Desingularization of Ore Ideals}}. 
                PhD thesis, Institute for Algebra, Johannes Kepler University Linz, 2017.
\end{itemize}

\section*{\Large{学术论文}}
\begin{itemize}
 \item Yi Zhang. {\em Contraction of Ore Ideals with Applications}. 
       In {\em Proceedings of the 2016 International Symposium on Symbolic and Algebraic Computation}, 
       pp.\ 413-420, ACM Press, 2016. DOI:\href{http://dl.acm.org/citation.cfm?id=2930890}{10.1145/2930889.2930890.}
\end{itemize}

\section*{\Large{待发表论文}}
\begin{itemize}
 \item Manuel Kauers, Ziming Li and Yi Zhang. {\em Apparent Singularities of D-finite Systems}, 2016. 
\end{itemize}

\section*{\Large{研究注解 (Research Notes)}}
\begin{itemize}
 \item Ziming Li and Yi Zhang. {\em A Note on Gr\"{o}bner Bases of Ore Polynomials over a PID}, 2016. 
 \url{https://yzhang1616.github.io/GB.pdf} 
 \item Yi Zhang. {\em Integer Vectors of a Fundamental Parallelepiped}, 2016.
\end{itemize}

\section*{\Large{其它出版物}}

\section*{\Large{学术报告}}
\begin{itemize}
 \item[5.] {\em Contraction of Linear Difference and Differential Operators}. Contributed talk at ISSAC'16 
 (the 41st International Symposium on Symbolic and Algebraic Computation), Wilfrid Laurier University, Waterloo, Canada, July, 2016.
 \item[4.] {\em Contraction of Linear Difference and Differential Operators}.
       Invited talk at the seminar of Center for Combinatorics, Nankai University, Tianjin, China, June, 2016.
 \item[3.] {\em An Algorithm for Contraction of an Ore Ideal}. Invited talk at the seminar of Institute of Discrete Mathematics and Geometry, 
       Vienna University of Technology, Vienna, Austria, October, 2015.
 \item[2.] {\em The Restriction Problem for D-finite Functions}. 
       Contributed talk at the Workshop on Computational and Algebraic Methods in Statistics,
       The University of Tokyo, Tokyo, Japan, March, 2015.
 \item[1.] {\em An Algorithm for Decomposing Multivariate Hypergeometric Terms}. Contributed talk at CM'13
       (the 5th National Conference of Computer Mathematics), Jilin University, Changchun, China, August, 2013.
\end{itemize}

\section*{\Large 学术期刊评审工作}
对于以下的学术期刊及会议论文,括号内给出了相应的评审次数。
\begin{itemize}
 \item International Symposiums on Symbolic and Algebraic Computation (1)
 \item Journal of Symbolic Computation (2)
\end{itemize}

\section*{\Large{编程语言}}
\vspace{.05in}
{\bf C, Matlab, Maple, Mathematica, Macaulay2 以及 \ Sage}

\section*{\Large{业余爱好及兴趣}}
\begin{itemize}
 \item 体育运动: 乒乓球, 桌球(台球), 网球
 \item 阅读: 文学, 历史, 哲学
 \item 语言: 中文 (母语), 英文 (流利), 德文 (基础)
\end{itemize}

\end{CJK*}
\end{document}
