\documentclass[a4paper,12pt]{article}
\usepackage{CJKutf8}
\usepackage{enumitem}
\usepackage[colorlinks = true,
            linkcolor = blue,
            urlcolor  = blue,
            citecolor = blue,
            anchorcolor = blue]{hyperref}
\usepackage{url}
\usepackage{graphicx,wrapfig,lipsum}
\usepackage{longtable}
\usepackage{fancyhdr}
\usepackage[margin=1in]{geometry}
 
\pagestyle{fancy}
\fancyhf{}
\rhead{Page \thepage}
\lhead{Yi Zhang}
\chead{Curriculum Vitae}
\rhead{\thepage}
\newcommand{\red}{\color{red}}

\title{\bf{\Huge{个人简历}}}
\author{}
\date{}

\begin{document}
\begin{CJK*}{UTF8}{gbsn}
\lhead{Yi Zhang}
\chead{Curriculum Vitae}


\maketitle
\thispagestyle{empty}

\begin{picture}(2,2)
 \put(300,-150){\includegraphics[width=3cm]{Yi_Zhang}}
\end{picture}

\section*{\Large{个人资料}}
\begin{tabular}{@{}p{1.2in}p{4in}}
姓名            & 张熠 \\
出生日期        & 1988年12月12日 \\
籍贯与户口       & 江苏省常州市 \\
国籍          & 中国 \\
婚姻状况       & 未婚 
\end{tabular}

\section*{\Large{联系方式}}
\begin{tabular}{@{}p{1.2in}p{4in}}
电子邮箱          & \href{mailto:yi.zhang@ricam.oeaw.ac.at}{yi.zhang@ricam.oeaw.ac.at}  \\
地址          & \href{https://www.ricam.oeaw.ac.at/}{Johann Radon Institute for Computational and Applied Mathematics (RICAM)} \\ 
                 & Austrian Academy of Sciences \\
                 & Altenbergerstra{\ss}e 69, A-4040 Linz, Austria \\
办公室           & S2 0435 (Science Park II) \\                
电话号码           & +43 732 2468 5235\\
个人网页         & \url{https://www.ricam.oeaw.ac.at/people/member/?firstname=Yi&lastname=Zhang}
\end{tabular}

\section*{\Large{研究领域}}
计算机代数, 计算代数几何, 以及 算法组合学  

\section*{\Large{教育经历}}
\begin{tabular}{@{}p{1.4in}p{4in}}
09/2013 -- 02/2017    & 理学博士 \ (优秀毕业生), 数学,  
                        \href{http://www.jku.at/algebra/content}{代数研究所}, 
                        \href{http://www.jku.at/content}{约翰·开普勒林茨大学}, 林茨, 奥地利 \\                        
                      & 导师: \href{http://www.kauers.de/}{Manuel Kauers} 教授 与  
                         \ \href{http://mmrc.iss.ac.cn/~zmli/}{李子明} 研究员\\
09/2011 -- 07/2016    & 理学博士,  应用数学, 
                        \href{http://english.mmrc.amss.cas.cn/}{数学机械化重点实验室}, 
                        \href{http://www.amss.ac.cn/}{中国科学院数学与系统科学研究院}, 
                        \href{http://www.gucas.ac.cn/}{中国科学院大学}, 北京, 中国 
                        \ 导师: \href{http://www.kauers.de/}{Manuel Kauers} 教授 与  
                         \ \href{http://mmrc.iss.ac.cn/~zmli/}{李子明} 研究员\\
09/2007 -- 07/2011    & 理学学士, 数学与应用数学 \ (师范), \href{http://math.suda.edu.cn/}{数学科学学院}, 
                        \href{http://www.suda.edu.cn/}{苏州大学}, 苏州, 中国
\end{tabular} \\

% \noindent 我于 09/2013 到 06/2015 在 Manuel Kauers 教授的指导下在\href{http://www.risc.jku.at/}{奥地利符号计算研究中心}, 
% 约翰开普勒林茨大学进行博士阶段的学习与研究。 

\section*{\Large{工作经历}}
\begin{tabular}{@{}p{1.4in}p{4in}}
03/2017 -- 至今        & 博士后, 
                        \href{https://www.ricam.oeaw.ac.at/}{Johann Radon Institute for Computational and Applied Mathematics} (RICAM),
                        \href{http://www.oeaw.ac.at/en/austrian-academy-of-sciences/}{奥地利科学院}. \\                       
                       & 合作导师: \href{http://www.koutschan.de/}{Christoph Koutschan} 博士\\
\end{tabular}

\section*{\Large{访问经历}}
\begin{tabular}{@{}p{1.0in}p{4.5in}}
2017年 5月               & 访问学者, 
                        \href{http://www.math.kobe-u.ac.jp/}{数学系},
                        \href{http://www.kobe-u.ac.jp/en/}{神户大学 \ (Kobe University)}, 日本. \\                       
                        & 合作导师: \href{http://www.math.kobe-u.ac.jp/home-j/takayama-e.html}{Nobuki Takayama \ (高山信毅)} 教授\\
\end{tabular}

\section*{\Large{获奖情况}}
\begin{tabular}{@{}p{1.4in}p{4in}}
07/2016               & \href{https://www.sigsam.org/Awards/ISSACAwards.html}{ACM SIGSAM颁发的ISSAC2016杰出学生论文奖} \\
09/2009 -- 07/2010    & 苏州大学人民综合二等奖学金\\
09/2008 -- 07/2009    & 苏州大学人民综合一等奖学金 \\
09/2007 -- 07/2008    & 苏州大学人民综合一等奖学金 \\ 
09/2007 -- 07/2008    & 苏州大学朱敬文奖学金 \\
09/2007 -- 07/2008    & 苏州大学校三好学生
\end{tabular}

% \section*{\Large{研究领域}}
% {\bf 计算机代数 (Computer Algebra), 计算代数几何 (Computational Algebraic Geometry), Ore代数 (Ore Algebras), 
% Gr\"{o}bner基 (Gr\"{o}bner Bases), 算法组合学 (Algorithmic Combinatorics) 以及 \ 实验数学 \\ (Experimental Mathematics)}

\section*{\Large{博士论文}}
\begin{itemize}
 \item Yi Zhang. \href{https://yzhang1616.github.io/yzhang_PhDthesis_final.pdf}{{\em Univarite Contraction and Multivariate Desingularization of Ore Ideals}}. 
                PhD thesis, Institute for Algebra, Johannes Kepler University Linz, 2017.
\end{itemize}

\section*{\Large{学术论文}}
\begin{enumerate}
\item Lin Jiu, Christoph Koutschan, Satoshi Kuriki, Nobuki Takayama, Akimichi Takemura and Yi Zhang. 
 {\em Euler Characteristic Method for the Largest Eigenvalue of a Random Matrix}, 2017, in preparation. 
\item Yi Zhang. {\em Desingularization in the $q$-Weyl Algebra}, 2017, in preparation.
\item N. Thieu Vo and Yi Zhang. {\em Laurent Series Solutions of Algebraic Ordinary Differential Equations}, 2017. 
 arXiv:\href{https://arxiv.org/abs/1709.04174}{1709.04174}, submitted. 
\item Shaoshi Chen, Manuel Kauers, Ziming Li and Yi Zhang. {\em Apparent Singularities of D-finite Systems}, 2017. 
 arXiv:\href{http://arxiv.org/abs/1705.00838}{1705.00838}, submitted.
\item Yi Zhang. {\em Contraction of Ore Ideals with Applications}. 
In {\em Proceedings of the 2016 International Symposium on Symbolic and Algebraic Computation}, 
pp.\ 413-420, ACM Press, 2016. DOI:\href{http://dl.acm.org/citation.cfm?id=2930890}{10.1145/2930889.2930890.} 
{\red (ISSAC为计算机科学 \\
“Algorithms and Theory”
领域的国际顶级会议,NUS评价为Rank 1,AUS评价为A+) (EI, 该文获得ISSAC2016杰出学生论文奖)} 
\end{enumerate}


% \section*{\Large{已发表论文}}
% \begin{itemize}
%  \item \textbf{Yi Zhang}. {\em Contraction of Ore Ideals with Applications}. 
%        In {\em Proceedings of the 2016 International Symposium on Symbolic and Algebraic Computation}, 
%        pp.\ 413-420, ACM Press, 2016. DOI:\href{http://dl.acm.org/citation.cfm?id=2930890}{10.1145/2930889.2930890.}
%        {\red (ISSAC为计算机科学“Algorithms and Theory”领域的国际顶级会议,NUS评价为Rank 1,AUS评价为A+) (EI, 该文获得ACM最佳学生论文, ISSAC 2016)}
% \end{itemize}
% 
% \section*{\Large{待发表论文}}
% \begin{itemize}
%  \item Manuel Kauers, Ziming Li and \textbf{Yi Zhang}. {\em Apparent Singularities of D-finite Systems}, 2017. 
%  arXiv:\href{http://arxiv.org/abs/1705.00838}{705.00838}, submitted to Journal of Symbolic Computation (SCI).
%  \item \textbf{Yi Zhang}. {\em Desingularization in the $q$-Weyl Algebra}, 2017.
%  \item Thieu Vo Ngoc and \textbf{Yi Zhang}. {\em Laurent Series Solutions of Algebraic Ordinary Differential Equations}, 2017. 
% \end{itemize}

\section*{\Large{研究注记 \ (Research Notes)}}
\begin{itemize}
 \item Yi Zhang. {\em Testing q-shift Equivalence of Polynomials}, July, 2017.
 \item Yi Zhang. {\em Integer Vectors of a Fundamental Parallelepiped}, 2016.
 \item Ziming Li and Yi Zhang. {\em A Note on Gr\"{o}bner Bases of Ore Polynomials over a PID}, 2016. 
 \url{https://yzhang1616.github.io/GB.pdf} 
%  \item Yi Zhang. {\em Integer Vectors of a Fundamental Parallelepiped}, 2016.
%  \item Yi Zhang. {\em Testing q-shift Equivalence of Polynomials}, July, 2017.
\end{itemize}

% \section*{\Large{其它出版物}}

\section*{\Large{软件包}}
\begin{itemize}
 \item \href{https://yzhang1616.github.io/KamkeODEs.mw}{KamkeODEs.mw}, a Maple worksheet for 
     checking the maximal comparability and noncriticality of algebraic
     ordinary differential equations in
     Kample's collection. It is based on joint work with Dr. Thieu Vo Ngoc. 
     The worksheet requires the availability of the Maple package \href{https://yzhang1616.github.io/KamkeODEs.mpl}{KamkeODEs.mpl}.
 \item \href{https://yzhang1616.github.io/qDesingularization.m}{qDesingularization.m}, a Mathematica
     package for computing desingularized operators and the $q$-Weyl closure of
     a given $q$-difference operator in
     the first $q$-Weyl algebra. It is based on joint work with Dr. Christoph
     Koutschan. The package requires the availability of Koutschan's package
     \href{http://www.risc.jku.at/research/combinat/software/ergosum/RISC/HolonomicFunctions.html}{HolonomicFunctions.m}
     and Kauer's pacakge \href{https://www.risc.jku.at/research/combinat/risc/software/Singular/index.html}{Singular.m}.
     For a description of the usage of the package, see the \href{https://yzhang1616.github.io/Example.nb}{Example.nb} notebook.
\end{itemize}

\section*{\Large{学术报告}}
\begin{enumerate}
 \item {\em Laurent Series Solutions of Algebraic Ordinary Differential Equations}. 
 Invited talk at Computer Algebra Seminar, Research Institute for Symbolic Computation (RISC), Johannes Kepler University Linz, 
 Austria, November, 2017.
 \item {\em Apparent Singularities of D-finite Systems}. Contributed talk at ACA'17 
 \ (the 23rd Conference on Applications of Computer Algebra), Jerusalem College of Technology, Jerusalem, Israel, July, 2017.
 \item {\em Contraction of Linear Difference and Differential Operators}. Contributed talk at ISSAC'16 
 \ (the 41st International Symposium on Symbolic and Algebraic Computation), Wilfrid Laurier University, Waterloo, Canada, July, 2016.
 \item {\em Contraction of Linear Difference and Differential Operators}.
       Invited talk at the seminar of Center for Combinatorics, Nankai University, Tianjin, China, June, 2016.
 \item {\em An Algorithm for Contraction of an Ore Ideal}. Invited talk at the seminar of Institute of Discrete Mathematics and Geometry, 
       Vienna University of Technology, Vienna, Austria, October, 2015.
 \item {\em The Restriction Problem for D-finite Functions}. 
       Contributed talk at the Workshop on Computational and Algebraic Methods in Statistics,
       The University of Tokyo, Tokyo, Japan, March, 2015.
 \item {\em An Algorithm for Decomposing Multivariate Hypergeometric Terms}. Contributed talk at CM'13
       \ (the 5th National Conference of Computer Mathematics), Jilin University, Changchun, China, August, 2013.
\end{enumerate}

\section*{\Large 学术期刊评审工作}
对于以下的学术期刊及会议论文,括号内给出了相应的评审次数。
\begin{itemize}
 \item International Symposiums on Symbolic and Algebraic Computation (1)
 \item Journal of Symbolic Computation (2)
\end{itemize}

\section*{\Large{其他技能}}
\begin{itemize}
 \item 编程技能: C, Matlab, Maple, Mathematica, Macaulay2 以及 \ Sage
 \item 口语: 中文 \ (母语), 英文 \ (流利), 德文 \ (基础)
\end{itemize}

% \section*{\Large{业余爱好及兴趣}}
% \begin{itemize}
%  \item 体育运动: 乒乓球, 桌球(台球), 网球
%  \item 阅读: 文学, 历史, 哲学
%  \item 语言: 中文 (母语), 英文 (流利), 德文 (基础)
% \end{itemize}

\end{CJK*}
\end{document}
