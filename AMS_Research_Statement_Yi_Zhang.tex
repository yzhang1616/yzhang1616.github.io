%\documentclass{article}[12pt]
%\usepackage{amssymb,amsmath}
\documentclass[10pt,a4paper]{article}
%\usepackage[T1]{fontenc}
%\usepackage[utf8]{inputenc}
\usepackage[margin=1.6in]{geometry}
\usepackage[utf8]{inputenc}
\usepackage{enumitem}
\usepackage{authblk}
%\usepackage{fullpage}
\usepackage[colorlinks = true,
            linkcolor = blue,
            urlcolor  = blue,
            citecolor = blue,
            anchorcolor = blue]{hyperref}

\usepackage{bbm}
\usepackage{amsmath,amsthm}
\usepackage{alltt, amssymb}
\usepackage[]{algorithm2e}
\usepackage{url}
\usepackage{graphicx}

\newcommand{\bigO}{{\mathcal{O}}}
\newcommand{\bigOsoft}{\tilde{\mathcal{O}}}
\newcommand{\bA}{ {\mathbb A}}
\newcommand{\bN}{ {\mathbb  N}}
\newcommand{\bZ} { {\mathbb Z}}
\newcommand{\bQ}{ {\mathbb  Q}}
\newcommand{\bC}{ {\mathbb  C}}
\newcommand{\bK}{ {\mathbb  K}}
\newcommand{\bD}{ {\mathbb D}}
\newcommand{\bE}{ {\mathbb E}}
\newcommand{\bR}{ {\mathbf R}}
\newcommand{\bx}{ {\mathbf x}}
\newcommand{\by}{ {\mathbf y}}
\newcommand{\bz}{ {\mathbf z}}
\newcommand{\bs}{ {\mathbf s}}
\newcommand{\bt}{ {\mathbf t}}
\newcommand{\bff}{ {\mathbf f}}
\newcommand{\bpa}{ {\boldsymbol \partial}}
\newcommand{\balpha}{ {\boldsymbol \alpha}}
\newcommand{\bbeta}{ {\boldsymbol \beta}}
\newcommand{\bdelta}{ {\boldsymbol \delta}}
\newcommand{\bgamma}{ {\boldsymbol \gamma}}
\newcommand{\bu}{ {\mathbf u}}
\newcommand{\bw}{ {\mathbf w}}
\newcommand{\cP}{ {\mathcal P}}
\newcommand{\cG}{ {\mathcal G}}
\newcommand{\bv}{ {\mathbf v}}
\newcommand{\bU}{ {\mathbf U}}
\newcommand{\bm}{ {\mathbf m}}
\newcommand{\bn}{ {\mathbf n}}
\newcommand{\bB}{ {\mathbf B}}
\newcommand{\ie}{{\it i.e.}}
\newcommand{\si} { {\sigma}}
\newcommand{\dd}{ {\rm d}}
\newcommand{\tor}{ {\rm tor}}
\newcommand{\den}{ {\rm den}}
\newcommand{\lc}{ \operatorname{lc}}
\newcommand{\lm}{ \operatorname{lm}}
\newcommand{\mm}{ \operatorname{M}}
\newcommand{\TT}{ \operatorname{T}}
\newcommand{\lt}{ \operatorname{lt}}
\newcommand{\HM}{{\operatorname{HM}}}
\newcommand{\HT}{{\operatorname{HT}}}
\newcommand{\PT}{{\operatorname{PT}}}
\newcommand{\PE}{{\operatorname{PE}}}
\newcommand{\HC}{{\operatorname{HC}}}
\newcommand{\lcm}{ \operatorname{lcm}}
\newcommand{\qlcm}{ \operatorname{qlcm}}
\newcommand{\spol}{ \operatorname{spol}}
\newcommand{\gpol}{ \operatorname{gpol}}
\newcommand{\rank}{ \operatorname{rank}}
\newcommand{\pa}{ {\partial}}
\newcommand{\lder}{ \operatorname{lder}}
\newcommand{\pder}{ \operatorname{pder}}
\newcommand{\ind}{ \operatorname{ind}}
\newcommand{\In}{ \operatorname{in}}
\newcommand{\sol}{ \operatorname{sol}}
\newcommand{\ann}{ \operatorname{ann}}
\newcommand{\dis}{ \operatorname{dis}}
\newcommand{\Drat}{ {\bK(\bx)[\bpa]}}
\newcommand{\Dpol}{ {\bK[\bx][\bpa]}}
\newcommand{\qpol}{ {\bK(q)[x][\pa]}}
\newcommand{\qrat}{ {\bK(q, x)[\pa]}}
\newcommand{\pf} {{\rm {\bf Proof.} } }
\newcommand{\cont}{\operatorname{Cont}}
\newcommand{\red}{\color{red}} % Yi
\newcommand{\AODE}{{AO{$\Delta$}E}}
\newcommand{\AODEs}{{AO{$\Delta$}Es}}

\newtheorem{thm}{Theorem}[section]
\newtheorem{cor}[thm]{Corollary}
\newtheorem{lemma}[thm]{Lemma}
\newtheorem{prop}[thm]{Proposition}
\newtheorem{defn}[thm]{Definition}
\newtheorem{ex}[thm]{Example}
\newtheorem{algo}[thm]{Algorithm}
\newtheorem{remark}[thm]{Remark}
\newtheorem{conj}[thm]{Conjecture}

\renewcommand{\labelenumi}{(\arabic{enumi})}

\begin{document}

\title{Research Statement}

\author{Yi Zhang}
% \thanks{Supported by the Austrian Science Fund (FWF): P29467-N32.\\
% E-mail addresses:  zhangy@amss.ac.cn}\ }
% % \author[2]{Ziming Li\thanks{Supported by the NSFC grants (91118001, 60821002/F02)
% % and a 973 project (2011CB302401). Email: zmli@mmrc.iss.ac.cn}}
% % \author{Yi Zhang$^{\ast}$}
% %\thanks{Supported by the Austrian Science Fund (FWF): P29467-N32. E-mail: zhangy@amss.ac.cn}}
% %\author[1]{Author C\thanks{C.C@university.edu}}
% %\author[2]{Author D\thanks{D.D@university.edu}}
% %\author[2]{Author E\thanks{E.E@university.edu}}
% % \affil[2]{KLMM, AMSS, Chinese Academy of Sciences, Beijing, China}
% % \affil[1, 3]{Institute for Algebra, Johannes Kepler University Linz, Austria}
% \affil{Johann Radon Institute for Computational and Applied Mathematics (RICAM), Austrian Academy of Sciences, Austria}
%\affil[2]{KLMM, AMSS, Chinese Academy of Sciences, Beijing 100190, China}

%\renewcommand\Authands{ and }

% \alignauthor
% \leavevmode
% \mathstrut
% Ziming Li\titlenote{Supported by the NSFC grants (91118001, 60821002/F02)
% and a 973 project (2011CB302401).\vspace{5pt}}\\[\smallskipamount]
%  \affaddr{\leavevmode\mathstrut KLMM, AMSS, Chinese Academy of Sciences, Beijing 100190, China}\\
%  \affaddr{\mathstrut 4040 Linz, Austria}\\
%  \affaddr{\mathstrut zmli@mmrc.iss.ac.cn}
% \and
%% \mathstrut
% Yi Zhang\titlenote{Supported by the Austrian Science Fund (FWF) grants Y464-N18, NSFC grants (91118001, 60821002/F02)
% and a 973 project (2011CB302401).\vspace{-4pt}}\\[\smallskipamount]
%  \affaddr{\leavevmode\mathstrut Institute for Algebra, Johannes Kepler University, Linz A-4040, Austria}\\
%  \affaddr{\leavevmode\mathstrut KLMM, AMSS, Chinese Academy of Sciences, Beijing 100190, China}\\
%  \affaddr{\leavevmode\mathstrut zhangy@amss.ac.cn}


\date{}
\maketitle

My main research interests are symbolic computation and its applications in combinatorics, knot theory, statistics, 
and cryptography.  Symbolic computation aims to give algorithmic and constructive answers 
to various problems in mathematics and computer science, such as polynomial factorization, computing solutions of systems of polynomial equations, 
and quantifier elimination. Systems of algebraic differential equations and difference equations are important research subjects in mathematics, physics, and related areas. 
The algebraic study of such systems gives useful information about their applications in physics, statistics, and other areas. 
%Computing solutions and illustrating algebraic structures of those systems give essential answers to their applications in physics, statistics and so on.  
%%Functions defined by systems of differential equations and difference equations are a principal focus of study in many areas of math and physics. Understanding the algebraic properties %of these functions is essential in many of their physical and mathematical applications. 
%Much of my work is devoted to developing algorithms that can discover these properties automatically.
Much of my work is devoted to developing algorithms for  computing solutions and illustrating algebraic structures of differential equations and difference equations
by using constructive tools (such as Gr\"obner bases and resultant theory) in computer algebra and differential algebra. 
My work has found interesting applications in the certification of integer sequences, checking special cases of a conjecture of Krattenthaler, and verifying 
several instances of the colored Jones polynomial are Laurent polynomial sequences.  

\nopagebreak 
\section*{Desingularization of linear differential and difference operators}

A D-finite function is specified by a linear ordinary differential equation with polynomial 
coefficients and finitely many initial values. Each singularity of a D-finite function 
must be a root of the coefficient of the highest order derivative appearing 
in the corresponding differential equation. 
For instance, $x^{-1}$ is a solution of the equation $x f'(x) + f(x) = 0$, 
and the singularity at the origin is also the root of the polynomial $x$. 
However, the converse is not true. For example, the solution space of 
the differential equation $x f'(x) - 4 f(x) = 0$ is spanned by $x^4$ as a vector space, 
but none of those functions has singularity at the origin. 

More specifically, for an ordinary equation $p_0(x)f(x) + \cdots + p_r(x)f^{(r)}(x) = 0$ with
polynomial coefficients $p_0,\dots,p_r$ and $p_r\neq0$, the roots of $p_r$ are called the
singularities of the equation. A root $\alpha$ of $p_r$ is called \emph{an apparent singularity} if the
differential equation admits $r$ linearly independent formal power series solutions in 
$x - \alpha$. Deciding whether a singularity is apparent is therefore the same as
checking whether the equation admits a fundamental system of formal power series
solutions at this point. This can be done by inspecting the so-called
\emph{indicial polynomial} of the equation at~$\alpha$ and solving a system of finitely many linear equations. 
If a singularity $\alpha$ of an ordinary differential equation is apparent, then we can always 
construct a second linear differential equation whose solution space contains all the solutions of the
first equation, and which does not have $\alpha$ as a singularity any more. 
This process is called \emph{desingularization}.  There are similar techniques for the difference case. 
Our contributions in this area are as follows: 

\begin{itemize}
\item Contraction of Ore ideals with applications~\cite{Zhang2016}. Ore operators form a common algebraic abstraction of linear ordinary differential and recurrence equations.
Given an Ore operator~$L$ with polynomial coefficients in~$x$, it generates a left ideal~$I$ in the Ore algebra
over the field~$\bK(x)$ of rational functions. We present an algorithm for computing a basis of the contraction ideal of~$I$
in the Ore algebra over the ring~$R[x]$ of polynomials, where~$R$ may be either~$\bK$ or a domain with~$\bK$ as its fraction field.
This algorithm is based on recent work on desingularization for Ore operators by Chen, Jaroschek, Kauers, and Singer.
Using a basis of the contraction ideal,
we compute a completely desingularized operator for~$L$ whose leading coefficient not only
has minimal degree in~$x$ but also has minimal content. Completely desingularized operators have interesting applications
such as certifying integer sequences and checking special cases of a conjecture of Krattenthaler.

\item Desingularization in the $q$-Weyl algebra~\cite{KZ2018}. 
We give an order bound for desingularized operators, and thus derive
an algorithm for computing desingularized operators in the first $q$-Weyl
algebra.  Moreover, an algorithm is presented for computing a generating set
of the first $q$-Weyl closure of a given $q$-difference operator.  As an
application, we certify that several instances of the colored Jones polynomial
are Laurent polynomial sequences by computing the corresponding desingularized
operator.

\item Apparent singularities of D-finite systems~\cite{Yi2017}. We generalize the notions of  ordinary points and singularities
from linear ordinary differential equations to D-finite systems.
Ordinary points and apparent singularities of a D-finite system are characterized in terms of its formal power series solutions.
We also show that apparent singularities can be removed like in the univariate
case by adding suitable additional solutions to the system at hand.
Several algorithms are presented for removing and detecting apparent singularities.
In addition,  an algorithm is given for computing formal power series solutions of a D-finite system
at apparent singularities.
\end{itemize}

I plan to work on the following problems in the near future:

\begin{itemize}
\item Design algorithms for determining a generating set of a contraction ideal in the multivariate Ore algebra.

\item Develop the desingularization technique for linear Mahler equations. 

\item Study the desingularization problem for the multivariate linear difference equations with polynomial coefficients. 
\end{itemize}

\section*{Computing symbolic solutions of algebraic differential and difference equations}

An algebraic ordinary difference equation ({\AODE}) is a difference equation of the form
\[
F(x, y(x), y(x + 1), \cdots, y(x + m))=0,
\]
where $m \in \mathbb{N}$ and $F$ is a nonzero polynomial in $y(x), y(x + 1), \cdots, y(x + m)$ with coefficients in the field $\bK(x)$ of rational functions over an algebraically closed field $\bK$ of characteristic zero.
%We say that an AO{$\Delta$}E is \emph{autonomous}  if the independent variable $x$ does not appear in it explicitly.
%For computational purpose, we may choose $\bK=\bar{\mathbb{Q}}$, the field of algebraic numbers.
{\AODE}s arise naturally in various kinds of problems, such as symbolic summation, 
factorization of linear difference operators, and the 
analysis of time or space complexity of computer programs with recursive calls. 
In these and other applications, the determination of the (closed form) solutions of a given {\AODE}  is a fundamental problem of general interest. 
We can define algebraic ordinary differential equations (AODEs) analogously.  Our contributions in this area are as follows: 

\begin{itemize}
\item Rational solutions of first-order \AODEs~\cite{VZ2019}. We propose an algebraic geometric approach for studying rational solutions of first-order \AODEs. 
For an autonomous first-order \AODE, we give an upper bound for the degrees of its rational solutions, and thus derive a complete algorithm for 
computing corresponding rational solutions.

\item  Rational solutions of  algebraic ordinary differential equations of arbitrary order~\cite{VoZhang2019}. 
We first prove a sufficient condition for the existence of a bound on the degree of the possible polynomial solutions to an AODE.
An AODE satisfying this condition is called \emph{noncritical}. 
Then we prove that some common classes of low-order AODEs are noncritical.
For the rational solutions, we determine a class of AODEs, which are called \emph{maximally comparable}, 
such that the possible poles of any rational solutions are recognizable from their coefficients. 
This generalizes the well-known fact that all the rational solutions to a given linear ODE are contained in the set of zeros of the leading coefficient.
Finally, we develop an algorithm to compute all rational solutions of certain maximally comparable AODEs, 
which is applicable to $78.54\%$ of the AODEs in Kamke's collection of standard differential equations.
\end{itemize}

I plan to work on the following problems in the near future:

\begin{itemize}
\item  Design algorithms to compute polynomial and rational solutions of high-order \AODEs.
\item Compute rational solutions of non-autonomous first-order \AODEs. 
\end{itemize}	



\bibliographystyle{abbrv}
%% \bibliography{reference}
\def\cprime{$'$}
\begin{thebibliography}{10}
\bibitem{Yi2017}
S.~Chen, M.~Kauers, Z.~Li, and Y.~Zhang.
\newblock Apparent singularities of {D}-finite systems.
\newblock {\em Journal of Symbolic Computation}, in press, 2019.

\bibitem{KZ2018}
C.~Koutschan and Y.~Zhang.
\newblock Desingularization in the $q$-Weyl algebra. \\
\newblock {\em Advances in Applied Mathematics}, 97, pp.\ 80–101, 2018. 

\bibitem{VoZhang2019}
N.T.~Vo and Y.~Zhang.
\newblock {Rational Solutions of High-Order Algebraic Ordinary Differential Equations}.
\newblock {\em accepted by Journal of Systems Sciences and Complexity}, 2019.

\bibitem{VZ2019}
N.T.~Vo and Y.~Zhang.
\newblock {Rational Solutions of First-Order Algebraic Ordinary Difference Equations}.
\newblock {\em arXiv 1901.11048}, 1--25, 2019.

\bibitem{Zhang2016}
Y.~Zhang.
\newblock {\em Contraction of {O}re ideals with applications}.
\newblock In {\em Proc.\ of ISSAC'16},  413--420, New York, NY, USA, 2016, ACM.


\end{thebibliography}
%
%\bibitem{Abramov2006}
%S.~A. Abramov, M.~Barkatou, and M.~van Hoeij.
%\newblock Apparent singularities of linear difference equations with polynomial
%  coefficients.
%\newblock {\em AAECC}, 117--133, 2006.
%% 
%% \bibitem{Abramov1999}
%% S.~A. Abramov and M.~van Hoeij.
%% \newblock Desingularization of linear difference operators with polynomial coefficients.
%% \newblock In {\em Proc.\ of ISSAC'99}, 269--275, New York, NY,USA, 1999, ACM.
%% 
%
%% \bibitem{Paule1999}
%% S.~A. Abramov, P.~Paule, and M.~Petkov\v{s}ek.
%% \newblock $q$-Hypergeometric solutions of $q$-difference equations.
%% \newblock In {\em Discrete Mathematics}, 194:3--22, 1999.
%% \bibitem{Aroca2001}
%% F.~Aroca and J.~Cano.
%% \newblock Formal solutions of linear PDEs and convex polyhedra.
%% \newblock {\em J. Symb.\ Comput.}, 32:717--737, 2001.
%% 
%% \bibitem{Barkatou2015}
%% M.~A. Barkatou and S.~S. Maddah.
%% \newblock Removing apparent singularities of systems of linear differential
%%   equations with rational function coefficients.
%% \newblock In {\em Proc.\ of ISSAC'15}, 53--60, New York, NY,
%%   USA, 2015, ACM.
%% 
%% \bibitem{Weispfenning1993}
%% T.~Becker and V.~Weispfenning.
%% \newblock {\em Gr\"{o}bner bases, a computational approach to commutative
%%   algebra}.
%% \newblock Springer-Verlag, New York, USA, 1993.
%% 
%
%% \bibitem{Chen2013}
%% S.~Chen, M.~Jaroschek, M.~Kauers, and M.~F. Singer.
%% \newblock Desingularization explains order-degree curves for {O}re operators.
%% \newblock In {\em Proc.\ of ISSAC'13}, 157--164, New York, NY,
%%   USA, 2013, ACM.
%  
%\bibitem{Yi2017}
%S.~Chen, M.~Kauers, Z.~Li, and Y.~Zhang.
%\newblock Apparent singularities of {D}-finite systems.
%\newblock {\em arXiv 1705.00838}, pages 1--26, 2017.
%  
%% \bibitem{Chen2016}
%% S.~Chen, M.~Kauers, and M.~F. Singer.
%% \newblock Desingularization of {O}re operators.
%% \newblock {\em J. Symb.\ Comput.}, 74:617--626, 2016.
%
%% \bibitem{Zhang2009}
%% R.~C. Churchill and Y.~Zhang.
%% \newblock Irreducibility criteria for skew polynomials.
%% \newblock {\em Journal of Algebra}, 322:3797--3822, 2009.
%% 
%% \bibitem{Chyzak2008}
%% F.~Chyzak.
%% \newblock Mgfun Project.
%% \newblock http://algo.inria.fr/chyzak/mgfun.html.
%% \bibitem{Chyzak2016}
%% F.~Chyzak, T.~Dreyfus, P.~Dumas and M.~Mezzarobba.
%% \newblock Computing solutions of linear Mahler equations.
%% \newblock {\em arXiv 1612.05518}, pages 1--42, 2016.
%
%% \bibitem{Chyzak2010}
%% F.~Chyzak, P.~Dumas, H.~Le, J.~Martin, M.~Mishna and B.~Salvy.
%% \newblock Taming apparent singularities via {O}re closure.
%% \newblock Manuscript, 2010.
%
%% \bibitem{Salvy1998}
%% F.~Chyzak and B.~Salvy.
%% \newblock Non-commutative elimination in {O}re algebras proves multivariate
%%   identities.
%% \newblock {\em J. Symb.\ Comput.\ }, 26:187--227, 1998.
%
%\bibitem{DenefLipshitz}
%J.~Denef and L.~Lipshitz.
%\newblock Power series solutions of algebraic differential equations.
%\newblock {\em Mathematische Annalen}, 267:213--238, 1984.
%
%\bibitem{FengGao}
%R.~Feng and X.-S. Gao.
%\newblock Rational general solutions of algebraic ordinary differential
%  equations.
%\newblock In {\em Proceedings of the 2004 International Symposium on Symbolic
%  and Algebraic Computation}, ISSAC'04, pages 155--162, New York, NY, USA,
%  2004. ACM.
%
%\bibitem{FengGao06}
%R.~Feng and X.-S. Gao.
%\newblock A polynomial time algorithm for finding rational general solutions of
%  first order autonomous {ODEs}.
%\newblock {\em Journal of Symbolic Computation}, 41(7):739--762, 2006.
%
%% \bibitem{Cox2015}
%% D.~Cox, J.~Little and D.~O'Shea.
%% \newblock Ideals, varieties, and algorithms.
%% \newblock Springer, 2015.
%% 
%% \bibitem{Gessel1981}
%% I.~Gessel.
%% \newblock Two theorems on rational power series.
%% \newblock Utilitas Mathematica, 19:247--254, 1981.
%% 
%% \bibitem{Weispfenning1990}
%% A.~Kandri-Rody and V.~Weispfenning.
%% \newblock Non-commutative {G}r\"obner bases in algebras of solvable type.
%% \newblock {\em J.\ Symb.\ Comput.\ }, 9:1--26, 1990.
%% 
%
%\bibitem{GraseggerThesis}
%G.~Grasegger.
%\newblock {\em {Symbolic solutions of first-order algebraic differential
%  equations}}.
%\newblock PhD thesis, Johannes Kepler University Linz, 06 2015.
%
%
%
%\bibitem{Ince1926}
%E.~Ince.
%\newblock Ordinary differential equations.
%\newblock Dover, 1926.
%% 
%% \bibitem{Stavros2012}
%% S.~Garoufalidis and C.~Koutschan.
%% \newblock {\em Twisting $q$-holonomic sequences by complex roots of unity.}
%% \newblock In {\em Proc.\ of ISSAC'12}, 179--186, New York, NY, USA, 2012, ACM.
%
%% \bibitem{GaroufalidisKoutschan12a}
%% S.~Garoufalidis and C.~Koutschan.
%% \newblock The non-commutative $A$-polynomial of $(-2, 3, n)$ pretzel knots.
%% \newblock {\em Experimental Mathematics}, 21(3):241--251, 2012.
%% \href{http://www.koutschan.de/data/pretzel/}{http:/$\!$/www.koutschan.de/data/pretzel/}
%%% @article{GaroufalidisKoutschan12a,
%%%  author = {Stavros Garoufalidis and Christoph Koutschan},
%%%  title = {The non-commutative {$A$}-polynomial of $(-2, 3, n)$ pretzel knots},
%%%  journal = {Experimental Mathematics},
%%%  year = {2012},
%%%  volume = {21},
%%%  number = {3},
%%%  pages = {241--251},
%%%  doi = {10.1080/10586458.2012.651409},
%%%  url = {http://www.koutschan.de/data/pretzel/},
%%%  isbn_issn = {ISSN 1058-6458},
%%% }
%
%% \bibitem{GaroufalidisKoutschan13}
%% S.~Garoufalidis and C.~Koutschan.
%% \newblock Irreducibility of $q$-difference operators and the knot $7_4$.
%% \newblock {\em Algebraic \& Geometric Topology}, 13(6):3261--3286, 2013.
%
%% \bibitem{GaroufalidisLe05}
%% S.~Garoufalidis and T.~T.~Q.~L{\^e}.
%% \newblock The colored Jones function is $q$-holonomic.
%% \newblock {\em Geometry and Topology}, 9:1253--1293, 2005.
%%% @article{GaroufalidisLe05,
%%%  author = {Stavros Garoufalidis and Thang T. Q. L{\^e}},
%%%  title = {The colored {J}ones function is {$q$}-holonomic},
%%%  journal = {Geometry and Topology},
%%%  volume = {9},
%%%  year = {2005},
%%%  pages = {1253--1293 (electronic)},
%%%  isbn_issn = {ISSN 1465-3060},
%%%  doi = {10.2140/gt.2005.9.1253},
%%% }
%
%% \bibitem{GaroufalidisSun}
%% S.~Garoufalidis and X.~Sun.
%% \newblock The non-commutative A-polynomial of twist knots.
%% \newblock {\em Journal of Knot Theory and Its Ramifications}, 19:1571--1595, 2010.
%% \href{http://people.math.gatech.edu/~stavros/publications/twist.knot.data/}%
%% {http:/$\!$/people.math.gatech.edu/~stavros/publications/twist.knot.data/}
%
%% \bibitem{Kauers2007}
%% M.~Kauers. 
%% \newblock Singular.m.
%% \href{https://www.risc.jku.at/research/combinat/risc/software/Singular/index.html}%
%% {https:/$\!$/www.risc.jku.at/\linebreak[0]research/\linebreak[0]combinat/\linebreak[0]risc/\linebreak[0]software/\linebreak[0]Singular/index.html}
%
%\bibitem{Kauers2009a}
%M.~Kauers.
%\newblock Guessing handbook.
%\newblock Tech. Report 09-07, RISC Report Series, Johannes Kepler University Linz, Austria, 2009. \\
%\href{http://www.risc.jku.at/publications/download/risc_3814/demo.nb.pdf}{http:/$\!$/www.risc.jku.at/publications/download/risc\_3814/demo.nb.pdf}
%
%% \bibitem{Kauers2009b}
%% M.~Kauers. 
%% \newblock Guess package. \\
%% \newblock {{\tt http://www.risc.jku.at/research/combinat/software/ergosum/RISC/Guess.html}}.
%% 
%% \bibitem{Kolchin1973}
%% E.~Kolchin.
%% \newblock Differential algebra and algebraic groups.
%% \newblock Academic Press., New York, 1973.
%% 
%% \bibitem{Christoph2009}
%% C.~Koutschan.
%% \newblock {\em Advanced applications of the holonomic systems approach}.
%% \newblock PhD thesis, RISC-Linz, Johannes Kepler Univ., 2009.
%% 
%\bibitem{Christoph2010}
%C.~Koutschan.
%\newblock {\em HolonomicFunctions user's guide}.
%\newblock Tech. Reprot 10-01, RISC Report Series, Johannes Kepler University Linz, Austria, 2010. \\
%\href{http://www.risc.jku.at/publications/download/risc_3934/hf.pdf}{http:/$\!$/www.risc.jku.at/publications/download/risc\_3934/hf.pdf}
%
%% \bibitem{ElectronicYZ}
%% C.~Koutschan and Y.~Zhang.
%% \newblock Supplementary electronic material 1 to the article {\em Desingularization in the $q$-Weyl algebra}. \\
%% \href{https://yzhang1616.github.io/qdesing.html}{https:/$\!$/yzhang1616.github.io/qdesing.html}
%
%% \bibitem{ElectronicCK}
%% C.~Koutschan and Y.~Zhang.
%% \newblock Supplementary electronic material 2 to the article {\em Desingularization in the $q$-Weyl algebra}. \\
%% \href{http://www.koutschan.de/data/qdesing/}{http:/$\!$/www.koutschan.de/data/qdesing/}
%
%\bibitem{KZ2018}
%C.~Koutschan and Y.~Zhang.
%\newblock Desingularization in the $q$-Weyl algebra. \\
%\newblock {\em Advances in Applied Mathematics}, 97, pp.\ 80–101, 2018. 
%
%\bibitem{Kovacic}
%J.~J. Kovacic.
%\newblock An algorithm for solving second order linear homogeneous differential
%  equations.
%\newblock {\em Journal of Symbolic Computation}, 2(1):3--43, 1986.
%
%
%% 
%% \bibitem{Li2002}
%% Z.~Li, F.~Schwarz and S.~Tsarev.
%% \newblock {\em Factoring zero-dimensional ideals of linear partial differential operators.}
%% \newblock In {\em Proc.\ of ISSAC'02},  168--175, New York, NY, USA, 2002, ACM.
%% 
%% \bibitem{Li2006}
%% Z.~Li, M.~Singer, M.~Wu and D.~Zheng.
%% \newblock {\em A recursive method for determining the one-Dimensional submodules of {L}aurent-{O}re modules.}
%% \newblock In {\em Proc.\ of ISSAC'06},  220-227, New York, NY, USA, 2006, ACM.
%% 
%% \bibitem{Mahler1929}
%% K.~Mahler. 
%% \newblock Arithmetische Eigenschaften der L\"{o}sungen einer Klasse von Funktionalgleichungen.
%% \newblock {\em Mathematische Annalen}, 103 (1929), no. 1, 532.
%
%
%% \bibitem{Man1994}
%% Y.~Man and F.~Wright.
%% \newblock Fast polynomial dispersion computation and its application to indefinite summation.
%% \newblock In {\em Proc.\ of ISSAC'94}, 175--180 , New York, NY, USA, 1994, ACM.
%
%\bibitem{NgoWinkler11b}
%L.~X.~C. {Ng\^o} and F.~Winkler.
%\newblock {Rational general solutions of parametrizable AODEs.}
%\newblock {\em {Publicationes Mathematicae}}, 79(3-4):573--587, 2011.
%
%% \bibitem{Max2013}
%% M.~Jaroschek.
%% \newblock {\em Removable singularities of {O}re operators}.
%% \newblock PhD thesis, RISC-Linz, Johannes Kepler Univ., 2013.
%
%\bibitem{Riese03}
%A.~Riese.
%\newblock qMultiSum---A Package for Proving $q$-Hypergeometric Multiple Summation Identities.
%\newblock {\em Journal of Symbolic Computation}, 35:349--376, 2003.
%
%%% @article{Riese03,
%%%  author = {Axel Riese},
%%%  title = {{qMultiSum}---{A} Package for Proving q-Hypergeometric Multiple Summation Identities},
%%%  journal = {Journal of Symbolic Computation},
%%%  volume = {35},
%%%  pages = {349--376},
%%%  isbn_issn = {ISSN 0747-7171},
%%%  year = {2003},
%%% }
%
%% \bibitem{Robertz2014}
%% D.~Robertz.
%% \newblock {\em Formal algorithmic elimination for PDEs}.
%% \newblock Springer, 2014.
%
%\bibitem{SingerFormalSolutions}
%M.~F. Singer.
%\newblock Formal solutions of differential equations.
%\newblock {\em Journal of Symbolic Computation}, 10(1):59 -- 94, 1990.
%
%% \bibitem{Stanley1980}
%% R.~P.~Stanley.
%% \newblock Differentiably finite power series.
%% \newblock {\em European Journal of Combinatorics}, 1:175--188, 1980.
%
%% \bibitem{Saito1999}
%% M.~Saito, B.~Sturmfels, and N.~Takayama.
%% \newblock {\em Gr\"{o}bner deformations of hypergeometric differential equations.}
%% \newblock Springer-Verlag, New York, USA, 1999.
%
%% \bibitem{Stavros2012}
%% S.~Garoufalidis and C.~Koutschan
%% \newblock {\em Twisting $q$-holonomic sequences by complex roots of unity.}
%% \newblock In {\em Proc.\ of ISSAC'12}, 179--186, New York, NY, USA, 2012, ACM.
%
%% \bibitem{Wolfram2008}
%% S.~Wolfram.
%% \newblock Wolfram Mathematica.
%% \newblock {{\tt http://www.wolfram.com/language/}}.
%\bibitem{VoGraseggerWinkler2017}
%N.~T. Vo, G.~Grasegger, and F.~Winkler.
%\newblock Deciding the existence of rational general solutions for first-order
%  algebraic {ODEs}.
%\newblock {\em Journal of Symbolic Computation}, 2017.
%
%\bibitem{VoWinkler2015}
%N.~T. Vo and F.~Winkler.
%\newblock {Algebraic general solutions of first order algebraic ODEs}.
%\newblock In V.~P.~G. et. al., editor, {\em {Computer Algebra in Scientific
%  Computing}}, volume 9301 of {\em Lecture Notes in Computer Science}, pages
%  479--492. Springer International Publishing, 2015.
%  
%  
%
%\bibitem{VoZhang2017}
%N.T.~Vo and Y.~Zhang.
%\newblock {Laurent Series Solutions of Algebraic Ordinary Differential Equations}.
%\newblock {\em arXiv: 1709.04174}, pages 1--21,  2017.
%
%\bibitem{Zhang2016}
%Y.~Zhang.
%\newblock {\em Contraction of {O}re ideals with applications}.
%\newblock In {\em Proc.\ of ISSAC'16},  413--420, New York, NY, USA, 2016, ACM.
%
%% \bibitem{Wu1989}
%% W.~Wu.
%% \newblock {\em On the foundation of differential algebraic geometry}.
%% \newblock MM Research Preprint 3:1--29, 1989.
%% 
%% \bibitem{Zhang2017}
%% Y.~Zhang.
%% \newblock {\em Univariate contraction and multivariate desingularization of {O}re ideals.}
%% \newblock PhD thesis, Institute for Algebra, Johannes Kepler Univ., 2017. \\
%% \href{https://yzhang1616.github.io/yzhang_PhDthesis_final.pdf}{https:/$\!$/yzhang1616.github.io/yzhang\_PhDthesis\_final.pdf}
%
%\end{thebibliography}
\end{document}
