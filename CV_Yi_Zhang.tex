\documentclass[a4paper,12pt]{article}
\usepackage[utf8]{inputenc}
\usepackage{enumitem}
\usepackage[colorlinks = true,
            linkcolor = blue,
            urlcolor  = blue,
            citecolor = blue,
            anchorcolor = blue]{hyperref}
\usepackage{url}
\usepackage{graphicx,wrapfig,lipsum}
\usepackage{longtable}
\usepackage{fancyhdr}
 
\pagestyle{fancy}
\fancyhf{}
\rhead{Page \thepage}
\lhead{Yi Zhang}
\chead{Curriculum Vitae}
\rhead{\thepage}

\title{\bf{\Huge{Curriculum Vitae}}}
\author{}
\date{}

\begin{document}

\maketitle
\thispagestyle{empty}

\begin{picture}(2,2)
 \put(300,-150){\includegraphics[width=3cm]{Yi_Zhang}}
\end{picture}

\section*{\Large{Personal Data}}

\vspace{.05in}
\begin{tabular}{@{}p{1.2in}p{4in}}
Full name            & Yi Zhang \\
Date of birth        & 12.12.1988 \\
Place of birth       & Changzhou, Jiangsu Province, China \\
Nationality          & Chinese \\
Marital Status       & Single 
\end{tabular}

\section*{\Large{Contact}}

\vspace{.05in}
\begin{tabular}{@{}p{1.2in}p{4in}}
E-mail           & \href{mailto:zhangy@amss.ac.cn}{zhangy@amss.ac.cn}  \\
Address          & \href{https://www.ricam.oeaw.ac.at/}{Johann Radon Institute for Computational and Applied Mathematics (RICAM)} \\ 
                 & Austrian Academy of Sciences \\
                 & Altenbergerstra{\ss}e 69, A-4040 Linz, Austria \\
Office           & S2 0435 (Science Park II) \\                
Phone            & +43 732 2468 5235\\
Homepage         & \url{https://yzhang1616.github.io/}
\end{tabular}

\section*{\Large{Education}}

\vspace{.05in}
\begin{tabular}{@{}p{1.4in}p{4in}}
09/2013 -- 02/2017    & Ph.D. in Mathematics with distinction, 
                        \href{http://www.jku.at/algebra/content}{Institute for Algebra}, 
                        \href{http://www.jku.at/content}{Johannes Kepler University Linz}, Austria. 
                        (Co-supervisors: Prof. \href{http://www.kauers.de/}{Manuel Kauers} and 
                        Prof. \href{http://mmrc.iss.ac.cn/~zmli/}{Ziming Li})\\
09/2011 -- 07/2016    & Ph.D. in Applied Mathematics, 
                        \href{http://english.mmrc.amss.cas.cn/}{Key Laboratory of Mathematics Mechanization}, 
                        \href{http://english.amss.cas.cn/}{Academy of Mathematics and Systems Science}, 
                        \href{http://english.ucas.ac.cn/}{University of Academy of Sciences}, Beijing, China. 
                        (Co-supervisors: Prof. Manuel Kauers and Prof. Ziming Li)\\
\end{tabular}
\begin{tabular}{@{}p{1.4in}p{4in}}
09/2007 -- 07/2011    & B.Sc. in Mathematics, \href{http://math.suda.edu.cn/}{School of Mathematical Sciences}, 
                        \href{http://eng.suda.edu.cn/}{Soochow University}, Suzhou, China.  
\end{tabular} \\

\noindent I also studied as a Ph.D. student in \href{http://www.risc.jku.at/}{Research Institute for Symbolic Computation}, Johannes Kepler University Linz 
from 09/2013 to 06/2015 under the supervision of Prof. Manuel Kauers. 


\section*{\Large{Work Experience}}

\vspace{.05in}
\begin{tabular}{@{}p{1.4in}p{4in}}
03/2017 -- 02/2018    & Postdoc researcher, 
                        \href{https://www.ricam.oeaw.ac.at/}{Johann Radon Institute for Computational and Applied Mathematics} (RICAM),
                        \href{http://www.oeaw.ac.at/en/austrian-academy-of-sciences/}{Austrian Acedemy of Sciences}. 
                        (Supervisor: Dr. \href{http://www.koutschan.de/}{Christoph Koutschan})\\
\end{tabular}

\section*{\Large{Visiting Experience}}

\vspace{.05in}
\begin{tabular}{@{}p{0.8in}p{4.5in}}
May 2017               & Visiting scholar, 
                        \href{http://www.math.kobe-u.ac.jp/}{Department of Mathematics} ,
                        \href{http://www.kobe-u.ac.jp/en/}{Kobe University}, Japan. 
                        (Host researcher: Prof. \href{http://www.math.kobe-u.ac.jp/home-j/takayama-e.html}{Nobuki Takayama})\\
\end{tabular}

\section*{\Large{Awards}}

\begin{tabular}{@{}p{1.4in}p{4in}}
07/2016               & \href{https://www.sigsam.org/Awards/ISSACAwards.html}{ACM Distinguished Student Author Award at ISSAC'16}, SIGSAM, Association for Computing Machinery. \\
09/2009 -- 07/2010    & The Second Prize Scholarship of Soochow University, Suzhou, China.\\
09/2008 -- 07/2009    & The First Prize Scholarship of Soochow University, Suzhou, China. \\
09/2007 -- 07/2008    & The First Prize Scholarship of Soochow University, Suzhou, China. \\ 
09/2007 -- 07/2008    & The Zhu Jingwen Scholarship of Soochow University, Suzhou, China. \\
09/2007 -- 07/2008    & The Merit Student of Soochow Universtiy, Suzhou, China. 
\end{tabular}

\section*{\Large{Research Interests}}
{\bf Computer Algebra, Computational Algebraic Geometry, Ore Algebras, Gr\"{o}bner Bases, Algorithmic Combinatorics and Experimental Mathematics}

\section*{\Large{PhD Thesis}}
\begin{itemize}
 \item Yi Zhang. \href{https://yzhang1616.github.io/yzhang_PhDthesis_final.pdf}{{\em Univarite Contraction and Multivariate Desingularization of Ore Ideals}}. 
                PhD thesis, Institute for Algebra, Johannes Kepler University Linz, 2017.
\end{itemize}


\section*{\Large{Peer-reviewed Publication}}
\begin{itemize}
 \item Yi Zhang. {\em Contraction of Ore Ideals with Applications}. 
       In {\em Proceedings of the 2016 International Symposium on Symbolic and Algebraic Computation}, 
       pp.\ 413-420, ACM Press, 2016. DOI:\href{http://dl.acm.org/citation.cfm?id=2930890}{10.1145/2930889.2930890.} 
       (ACM Distinguished Student Author Awards)
\end{itemize}

\section*{\Large{Papers in Preperation}}
\begin{itemize}
 \item Manuel Kauers, Ziming Li and Yi Zhang. {\em Apparent Singularities of D-finite Systems}, 2016. 
\end{itemize}

\section*{\Large{Research Notes}}
\begin{itemize}
 \item Ziming Li and Yi Zhang. {\em A Note on Gr\"{o}bner Bases of Ore Polynomials over a PID}, 2016. 
 \url{https://yzhang1616.github.io/GB.pdf} 
 \item Yi Zhang. {\em Integer Vectors of a Fundamental Parallelepiped}, 2016.
\end{itemize}

\section*{\Large{Other Publication}}

\section*{\Large{Software}}
Unless otherwise stated, the software provided on this web site is free. You can redistribute it and/or modify it under the 
terms of the GNU General Public License as published by the Free Software
foundation; either version 2 of the licence, or (at your option) any later
version. The program is distributed in in the hope that they will be useful,
but without any warranty; without even the implied warranty of
merchantability or fitness for a particular purpose. 
See the \href{http://www.gnu.org/licenses/gpl.html}{GNU General Public Licence} for more details.

\section*{\Large{Talks}}
\begin{itemize}
 \item[5.] {\em Contraction of Linear Difference and Differential Operators}. Contributed talk at ISSAC'16 
 (the 41st International Symposium on Symbolic and Algebraic Computation), Wilfrid Laurier University, Waterloo, Canada, July, 2016.
 \item[4.] {\em Contraction of Linear Difference and Differential Operators}.
       Invited talk at the seminar of Center for Combinatorics, Nankai University, Tianjin, China, June, 2016.
 \item[3.] {\em An Algorithm for Contraction of an Ore Ideal}. Invited talk at the seminar of Institute of Discrete Mathematics and Geometry, 
       Vienna University of Technology, Vienna, Austria, October, 2015.
 \item[2.] {\em The Restriction Problem for D-finite Functions}. 
       Contributed talk at the Workshop on Computational and Algebraic Methods in Statistics,
       The University of Tokyo, Tokyo, Japan, March, 2015.
 \item[1.] {\em An Algorithm for Decomposing Multivariate Hypergeometric Terms}. Contributed talk at CM'13
       (the 5th National Conference of Computer Mathematics), Jilin University, Changchun, China, August, 2013.
\end{itemize}

\section*{\Large Peer-Reviewing Activities}
For each journal and conference the number of completed reviews in given in parentheses.
\begin{itemize}
 \item International Symposiums on Symbolic and Algebraic Computation (1)
 \item Journal of Symbolic Computation (2)
\end{itemize}

\section*{\Large{Programming Language}}
\vspace{.05in}
{\bf C, Matlab, Maple, Mathematica, Macaulay2 and Sage}

\section*{\Large{Hobbies and Interests}}
\begin{itemize}
 \item Sports: Table Tennis, Billiards, Tennis.
 \item Reading: Philosophy, History, Literature.
 \item Language: Chinese (native), English (fluent), German (basic).
\end{itemize}

\end{document}
